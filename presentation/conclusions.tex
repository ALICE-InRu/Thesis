\section{Conclusions}\againframe<8|handout:0>{alice}
\frame{\frametitle{Using \fullnameAlice\ framework I}
The thesis introduced a framework for learning (\alert<1>{linear}) \cdr\ -- 
using \alert<1>{\jsp} as a case-study -- with the following guidelines
\pause        
    \bi For a given problem domain, use a suitable problem \alert<2>{generator} 
    to \alert<2>{train} and \alert<2>{test} on.
\pause 
    \item Define \alert<3>{features} to grasp the essence of visited 
    \alert<3>{$k$-solutions}
\pause
\item \alert<4->{Success} is highly dependent on the preference pairs 
introduced to the system: 
\pause
    \bi \PsiSet{p} \alert<5>{reduces} the preference set \alert<5>{without 
    loss} of performance. 
\pause
    \item \alert<6>{Stepwise bias} is needed to balance time dependent 
    \PsiSet{p} in order to create \alert<6>{time independent} models. \ei
\pause 
    It is \alert<7>{non intuitive} how to go about \alert<7>{collecting} 
    training data\ei
}
\frame{\frametitle{Using \fullnameAlice\ framework II}
Continued from prev. slide
    \bi Learning \alert<1>{optimal} trajectories predominant in literature. 
    Study showed \PhiSet{\OPT} can result in \alert<1>{insufficient} knowledge.
\pause 
    \item Following \alert<2>{sub-optimal} deterministic policies, yet 
    labelling with an optimal solver, \alert<2>{improves} the guiding 
    policy.
\pause 
    \item Active update procedure using \alert<3>{DAgger} ensures sample states 
    the \alert<3>{learned model is likely to encounter} is integrated to 
    \PsiSet[\DA{i}]{p}.
\pause
    \item Instead of reusing same problem instances, extend the training set 
    with \alert<4>{new} instances for \alert<4>{quicker convergence} of 
    DAgger.
\pause 
    \item In \alert<5>{sequential} decision making, all future observations are 
    dependent on \alert<5>{previous} operations\ei
}
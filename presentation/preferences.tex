\section{Preference set}\againframe<12|handout:0>{alice}
\frame{\frametitle{Generating training data}
    \Alice\ framework for creating \dr s
    \bi \alert<1>{Linear classification} to identify good dispatches, 
    from worse ones. 
    \pause \item \alert<2>{Generate} feature set, 
    $\alert<2>{\Phi}\subset\mathcal{F}$, both from 
    \bi \alert<2>{optimal} solutions, $\vphi^o$ 
    \item \alert<2>{suboptimal} solutions, $\vphi^s$\ei 
    by exploring various \alert<2>{trajectories} within the feature-space 
    (where $\vphi^o,\vphi^s\in\mathcal{F}$).
    \pause \item Sample $\Phi$ to \alert<3>{create} training set 
    \alert<3>{$\Psi$} with rank pairs:
    \bi \alert<3>{optimal} decision, $(\vec{z}^o,y_o)=(\vphi^o-\vphi^s,+1)$ 
    \item \alert<3>{suboptimal} decision, 
    $(\vec{z}^s,y_s)=(\vphi^s-\vphi^o,-1)$\ei
    using different \alert<3>{ranking} schemes 
    (where $\vec{z}^o,\vec{z}^s\in\Psi$) 
    \pause \item \alert<4>{Sample $\Psi$} using \alert<4>{stepwise bias} for 
    time independent policy\ei
}

\againframe<handout:0>{featsize}
\frame{
    \frametitle{Sampled size of $\abs{\Psi(k)}$}
    \framesubtitle{($6\times5, N_{train}=500)$}
    \vspace{-12pt}
    \begin{center}   
        \includegraphics[width=\columnwidth]{presentation/{prefdat.p.size.6x5}.pdf}
    \end{center}    
}

\frame{
    \frametitle{Stepwise bias strategies}
    \framesubtitle{($6\times5, N_{train}=500$)}
    \includegraphics[width=\columnwidth]{presentation/{bias.CDR.6x5}.pdf}
}

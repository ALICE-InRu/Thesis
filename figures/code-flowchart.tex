\usetikzlibrary{shapes,positioning,shadows,trees}

\tikzset{
    txtComment/.style = {above, font=\tiny\sffamily, short, near end},
    isLeft/.style = {align=right,left},
    isRight/.style = {align=left,right},
}

\forestset{
    normal/.style  = {for tree={child anchor=north, parent anchor=south}},
    root/.style  = {txtLrg, fill=gray!50, medium},
    onode/.style = {txtLrg, fill=gray!25, medium},
    tnode/.style = {txtLrg, normal, fill=gray!10, short},
    emphasis/.style = {}, %tikz={\node [draw,red,thick,fit to tree]{};}},
    edge from parent/.style={arrow, edge from parent fork right}
}

\begin{figure}[p] \centering
\begin{forest}
for tree={
    s sep=6pt, % distance between siblings
    child anchor=north,
    parent anchor=south,
    l sep+=12.5pt,
    edge path={
        \noexpand\path[<-, >={latex},
        \forestoption{edge}] (!u.parent anchor) -- +(0,-5pt) -| (.child anchor) 
        \forestoption{edge label};
    }
}
[Problem Space: \cref{ch:genprobleminstances}, root
    [Optimum Data: Expert policy $\pi_\star$, tnode, medium, name=Rho,
        edge label = {node[txtComment,isLeft,long]{ Query \cite{gurobi}}},
        [Training Set: \cref{ch:gentrdat}, tnode, medium, name=TRDAT
            [Retrace Set, onode,
            edge label = {node[txtComment,isLeft]{Update $\Phi$ in 
            \cref{eq:Phi}}},
                [CDR accuracy $\xi_{\pi}$: \cref{eq:tracc:opt}, tnode, name=ACC]
                [Preference Set $\Psi$: \cref{eq:Psi:jsp}, tnode, name=PREF]
            ]
        ]
        [CMA-ES Data: \cref{ch:esmodels}, tnode, medium, name=CMAES,
        edge label = {node[txtComment,isRight]{ Dependent for 
                \cref{eq:cma:rho} not \cref{eq:cma:makespan} }} 
        ]
    ]
    [Performance Space, onode, name=Cmax
        [CDR Data: \cref{sec:CDR}, tnode, name=CDR
            [, phantom, calign with current
                [Linear policy $\pi$: \cref{eq:CDR:feat}, onode, short, 
                name=LINEAR]
                [Features: \cref{tbl:features}, onode, short, name=FEAT]
                [Schedule: \cref{ch:scheduling}, onode, short, name=JSP]
            ]
        ]
        [SDR Data: \cref{sec:SDR}, tnode, name=SDR
            [BDR Data: \cref{sec:diff:opt:bdr}, tnode]
        ]
    ]
]
\draw[arrow] (Rho) to node[txtComment,sloped] {\cref{eq:rho}}  (Cmax);
\draw[arrow] (JSP) ->  (FEAT);
\draw[arrow] (FEAT) ->  (LINEAR);
\draw[arrow] (LINEAR) to[bend right=5]  (SDR);
\draw[arrow] (LINEAR) to[bend right=15]  (CDR);
\draw[arrow] (PREF) to[bend right=30]  (LINEAR);
\draw[arrow] (CMAES) to[bend right=30] (LINEAR);
\draw[arrow] (LINEAR) to[bend right=25] (ACC);
\draw[arrow] (LINEAR) to[bend right=5] node[txtComment,sloped,midway] 
{\cref{ch:imitation}} (TRDAT);
\end{forest}
\caption[Class diagram for \Alice]{Class diagram for \Alice, \texttt{C\#} 
implementation available at 
\href{https://github.com/ALICE-InRu/Code/tree/master/csharp/ALICE}{github}}
\label{code:classdiagram}
\end{figure}
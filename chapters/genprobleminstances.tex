\HeaderQuote{If it had grown up, it would have made a dreadfully ugly child; but it makes rather a handsome pig, I think.}{Alice} 

\chapter{Problem generators}\label{ch:genprobleminstances} 

\FirstSentence{S}{ynthetic problem instances for } \JSP\ and \FSP\  will be used  throughout this dissertation. The  problem spaces are detailed in the \cref{sec:data:JSP,sec:data:FSP} for \JSP\ and \FSP, respectively. Moreover, a brief summary is given in \cref{tbl:data}.
Following the approach in \citet{Whitley}, difficult problem instances are not filtered out beforehand, although they will be specifically addressed in \cref{ch:problemstructure}. 

Although real-world instances are desirable, unfortunately they are scarce, hence in some experiments, problem instances from OR-Library maintained by \citet{ORlibrary} will be used as benchmark problems, and detailed in \cref{sec:data:orlib}. It is noted, that some of the instances are also simulated, but the majority are based on real-world instances, albeit sometimes simplified. 
%\citet{Panwalkar77} reports an extensive survey of 36 research papers in scheduling, most experiments are based on simulated data, and its verification on real-world data would be desirable, but missing. 

\section{\Jsp}\label{sec:data:JSP}
Problem instances for \JSP\ are generated stochastically by fixing the number of jobs and machines and 
discrete processing time are i.i.d. and sampled from a discrete uniform distribution. % from the interval $I=[u_1,u_2]$, i.e., $\vec{p}\sim \mathcal{U}(u_1,u_2)$. 
Two different processing times distributions were explored, namely,
\begin{description}
	\item[\Jrnd] \jrnd{n}{m} \hfill \\ where $\vec{p}\sim\mathcal{U}(1,99)$;
	\item[\Jrndn] \jrndn{n}{m} \hfill \\ where $\vec{p}\sim\mathcal{U}(45,55)$.
\end{description}
The machine ordering is a random permutation of all of the machines in the \jsp. 
For each \JSP\ class $N_{\text{train}}$  and $N_{\text{test}}$ instances were generated for training and testing, respectively. Values for $N$ are given in \cref{tbl:data}. 

Although in the case of \jrnd{n}{m}\ this may be an excessively large range for 
the uniform distribution, it is however, chosen in accordance with the 
literature \citep{Demirkol98} for creating synthesised $Jm||C_{\max}$ problem 
instances. In addition, w.r.t. the machine ordering, one could look into a 
subset of \JSP\ where the machines are partitioned into two (or more) sets, 
where all jobs must be processed on the machines from the first set (in some 
random order) before being processed on any machine in the second set, commonly 
denoted as $Jm|2\textrm{sets}|C_{\max}$ problems, but as discussed in 
\cite{orlib_swv} this family of \JSP\ is considered ``hard'' (w.r.t. relative 
error from best known solution) in comparison with the ``easy'' or 
``unchallenging'' family with the general $Jm||C_{\max}$ setup. % ath. 
%Holtsclaw96 vitnar í orlib_swv um easy-hard pælinguna
This is in stark contrast to \citet{Whitley} whose findings showed that structured $Fm||C_{\max}$ were much easier to solve than completely random structures. 
Intuitively, an inherent structure in machine ordering should be exploitable for a better performance.  However, for the sake of generality, a random structure is preferred as they correspond to difficult problem instances in the case of \JSP. Whereas, structured problem subclasses will be explored for \FSP.  

Moreover, in order to inspect the impact of any slight change within the problem spaces, two mutated versions were created based on \jrnd{n}{m}, namely, 
\begin{description}
	\item[\JrndJ] \jrndJ{n}{m} \hfill \\ where the first job, $J_1$, is always twice as long as its random counterpart, i.e.,\linebreak
	$\tilde{p}_{1a}=2\cdot p_{1a}$, where $p\in$\jrnd{n}{m}, for all $M_a\in\mathcal{M}$. 
	\item[\JrndM] \jrndM{n}{m} \hfill \\ where the first machine, $M_1$, is always twice as long as its random counterpart, i.e.,\linebreak
	$\tilde{p}_{j1}=2\cdot p_{j1}$, where $p\in$\jrnd{n}{m}, for all $J_j\in\mathcal{J}$. 
\end{description}
Therefore making job $J_1$ and machine $M_1$ bottlenecks for \jrndJ{n}{m} and \jrndM{n}{m}, respectively.

\section{\Fsp}\label{sec:data:FSP}
Problem instances for \FSP\  are generated using \citet{Whitley} problem generator\footnote{Both code, written in \texttt{C++}, and problem instances used in their experiments can be found at: \url{http://www.cs.colostate.edu/sched/generator/}}. There are two fundamental types of problem classes: non-structured versus structured.

Firstly, there are two ``conventional'' random, i.e., non-structured, problem classes for \FSP\  where processing times are i.i.d. and uniformly distributed, 
\begin{description}
	\item[\Frnd]  \frnd{n}{m} \hfill \\
	where $\vec{p}\sim\mathcal{U}(1,99)$ whose instances are equivalent to \cite{Taillard1993}\footnote{\citeauthor{Taillard1993}'s generator is available from the OR-Library.};
	\item[\Frndn]  \frndn{n}{m} \hfill \\
	where $\vec{p}\sim\mathcal{U}(45,55)$.
\end{description}
In the \JSP\ context \frnd{n}{m}\ and \frndn{n}{m}\ are analogous to \jrnd{n}{m}\ and \jrndn{n}{m}, respectively.  


Secondly, there are three structured problem classes of \FSP\  which are modelled after real-world \emph{characteristics} in \fsp\ manufacturing, namely, 
\begin{description}
	\item[\Fjc] \fjc{n}{m} \hfill \\
	job processing times are dependent on job index, however, independent of 
	machine index. Job-correlation can be of degree $0\leq\alpha\leq1$;
	\item[\Fmc]  \fmc{n}{m} \hfill \\
	job processing times are dependent on machine index, however, independent of 
	job index. Machine-correlation can be of degree $0\leq\alpha\leq1$; 
	\item[\Fmxc]  \fmxc{n}{m} \hfill \\
	job processing times are dependent on machine and job indices. Mixed-correlation can be of a degree $0\leq\alpha\leq1$.
\end{description} 
Note, for $\alpha=0.0$ the problem instances closely correspond to \frnd{n}{m}, 
hence the degree of $\alpha$ controls the transition of random to structured 
\FSP. Let's assume $\alpha=1$.

For each \FSP\  class $N_{\text{train}}$  and $N_{\text{test}}$ instances were generated for training and testing, respectively. Values for $N$ are given in \cref{tbl:data}. Moreover, an example of distribution of processing times are depicted in \cref{fig:fsp:structure}.


\begin{table}\centering
	\caption[Problem space distributions used in experimental studies.]{Problem 
	space distributions used in experimental studies. Note, problem instances are 
	synthetic and each problem space is i.i.d.}\label{tbl:data}
	{\renewcommand{\arraystretch}{1.5}
		\begin{tabular}{llcccl}\toprule
			type & name           & size ($n\times m$) & $N_{\text{train}}$ & $N_{\text{test}}$ & note                          
			\\ \midrule
			\multirow{8}{*}{\rot{\JSP}}
			& \jrnd{6}{5}   &$6\times5$ & 500 & 500 & random \\
			& \jrndn{6}{5}  &$6\times5$ & 500 & 500 & random-narrow \\
			& \jrndJ{6}{5}  &$6\times5$ & 500 & 500 & random with job variation \\
			& \jrndM{6}{5}  &$6\times5$ & 500 & 500 & random with machine variation \\
      & \jrnd{10}{10} &$10\times10$& 300 & 200 & random \\
			& \jrndn{10}{10}&$10\times10$& 300 & 200 & random-narrow \\ 
      &\jrndJ{10}{10} &$10\times10$& 300 & 200 & random with job variation\\
      &\jrndM{10}{10} &$10\times10$& 300 & 200 & random with machine variation\\
			\midrule
			\multirow{6}{*}{\rot{\FSP}}
			& \frnd{6}{5}  & $6\times5$ & 500 & 500 & random \\ 
			& \frndn{6}{5} & $6\times5$ & 500 & 500 & random-narrow \\ 
			& \fjc{6}{5}   & $6\times5$ & 500 & 500 & job-correlated \\ 
			& \fmc{6}{5}   & $6\times5$ & 500 & 500 & machine-correlated \\ 
			& \fmxc{6}{5}  & $6\times5$ & 500 & 500 & mixed-correlation \\ 
			& \frnd{10}{10}& $10\times10$ & 300 & 200 & random \\ 
			\bottomrule
		\end{tabular}}
\end{table}

\begin{figure}\centering 
	\includegraphics[width=\textwidth]{proctimes.pdf}
	\caption[Examples of distribution of job processing times for $6\times5$ 
	\FSP\  with different types of structure.]{Examples of distribution of job 
	processing times for $6\times5$ \FSP\  with different types of structure. 
	Machine indices are on the horizontal axis, job indices are colour-coded and 
	their corresponding processing times, $p_{ja}$, are on the vertical axis.}
	\label{fig:fsp:structure}
\end{figure}


\section{Benchmark problem suite}\label{sec:data:orlib}
A total of 82 and 31 benchmark problems for \JSP\ and \FSP, respectively, were 
obtained from the Operations Research Library (OR-Library) maintained by 
\citet{ORlibrary} and summarised in \cref{tbl:data:orlib}. 
Given the high problem dimensions of some problems, the optimum is not known, 
hence in those instances \cref{eq:rho} will be reporting  deviation from the 
latest best known solution (BKS) from the literature, reported by 
\citet{jsspBESTsofar} save for \Problem{swv} which can be found 
in \citet{CITATION_NEEDED}, and for \FSP\  consult \citet{CITATION_NEEDED}.
\todo{Ekki fundið BKS fyrir jsp swv and fsp OR-LIBRARY}

\subsection*{\Jsp\ OR-Library}
\citet{orlib_ft} had one of the more notorious benchmark problems for \JSP, and 
computationally expensive, however, now these instances have been solved to 
optimality. 
Similar to the synthetic \JSP\ problem spaces discussed earlier, 
\citet{orlib_abz} introduce five \JSP\ instances with a random permutation of 
machine ordering and processing times following uniform distribution,  
$\vec{p}\sim\mathcal{U}(50,100)$, for  dimensions $10\times10$ and 
$20\times15$. Likewise, \citet{orlib_yn} consists of four $20\times20$ random 
problem instances, where $\vec{p}\sim\mathcal{U}(10,50)$.
\citet{orlib_swv} introduce a set of \JSP\ problems where the job processing 
times following a uniform distribution, $\vec{p}\sim\mathcal{U}(1,100)$. There 
are a total of five problems in four dimension classes 
\begin{enumerate*}
  \item $20\times10$ (\texttt{swv01-swv05})
  \item $20\times15$ (\texttt{swv06-swv10})
  \item $50\times10$ (\texttt{swv11-swv15})
  \item $50\times10$ (\texttt{swv16-swv20})
\end{enumerate*}
Where the first three classes are considered `{hard}' and the last one as 
`{easy}'. Easy problems are ones corresponding to random machine ordering, 
whereas hard problems are partitioned in such a way the jobs must be processed 
on the first half of the machines before starting on the second half, i.e., 
$Jm|\text{2sets}|C_{\max}$.
\citet{orlib_orb} introduced ten problem instances of $10\times10$ \JSP\ where 
generated such that the machine ordering was chosen by random users in order to 
make them `difficult.' Moreover, the processing times were drawn at random, 
and the distribution that had the greater gap between its optimal value and 
standard lower bound was chosen. 

\subsection*{\Fsp\ OR-Library}
For the \FSP\ benchmarks, \citet{orlib_hel} introduces two deterministic 
instances based on `many-machine version of book-printing,' where processing 
times for $n\in\{20,100\}$ jobs and $m=10$ machines are relatively short, 
i.e., $p_{ja}\in\{0,..,9\}$. \citet{orlib_car} however, comprises of eight 
problems (of various dimension) where there is high variance in processing 
times, presumably $\vec{p}\sim\mathcal{U}(1,1000)$. 
\citet{orlib_rec} argue that completely random problem instances are unlikely 
to occur in practice, however, only the random instances used are reported in 
the OR-Library;\footnote{Only odd-numbered instances in \texttt{rec01-rec42}
  are given, since the even-numbered instances are obtained from the previous 
  instance by just reversing the processing order of each job; the optimal 
  value of each odd-numbered instance and its even-numbered counterpart is the 
  same.} 
for a total of 42 problem instances with processing times 
following a uniform distribution, $\vec{p}\sim\mathcal{U}(1,100)$, of 
dimensions varying from $20\times5$ to $75\times20$. 

\begin{table}\centering
  \caption{Benchmark problems from OR-Library used in experimental studies.}
  \label{tbl:data:orlib}
  \begin{tabular}{llrrll}\toprule
    type & name & size ($n\times m$) & $N_{\text{test}}$ & note & shorthand  \\
    \midrule \multirow{6}{*}{\rot{\JSP}}
    &\Problem{ft} & various &  3 &\citet{orlib_ft} & \texttt{ft06,ft10,ft20}\\
    &\Problem{la} & various & 40 &\citet{orlib_la} & \texttt{la01-la40}     \\
    &\Problem{abz}& various &  5 &\citet{orlib_abz}& \texttt{abz05-abz09}   \\
    &\Problem{orb}& $10\times10$& 10 &\citet{orlib_orb}& \texttt{orb01-orb10}\\
    &\Problem{swv}& various & 20 &\citet{orlib_swv}&\texttt{swv01-swv20}\\
    & \Problem{yn} & $20\times20$& 4  &\citet{orlib_yn} & \texttt{yn01-yn04}\\
    \midrule \multirow{3}{*}{\rot{\FSP}}
    &\Problem{car}& various &  8 & \citet{orlib_car} & \texttt{car1-car8} \\
    &\Problem{hel}& various &  2 & \citet{orlib_hel} & \texttt{hel1,hel2}  \\
    &\Problem{rec}& various & 21 & \citet{orlib_rec} & \texttt{rec01-rec42}\\
    \bottomrule
  \end{tabular}
\end{table}

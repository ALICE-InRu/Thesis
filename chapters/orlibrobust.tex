\HeaderQuote{The adventures first... explanations take such a dreadful 
time.}{The Gryphon} 


\chapter{OR-Library Comparison}\label{ch:orlibrobust} 

\FirstSentence{U}{ntil now only evolutionary search} models from 
\cref{ch:esmodels} have been checked for robustness using the OR-Library 
described in \cref{sec:data:orlib}. This was done to show the 
impact of choosing objective function defined in \cref{eq:cma:objfun}. 
However, there was no reference made to other models, except for demonstrating 
how far off the result was from the best known solutions (BKS).
Now we will also consider the best configurations for preference models in 
\cref{ch:prefmodels,ch:featselect,ch:imitation,ch:rollout}.
Although, all models will be using the same training data \jrnd{10}{10}.

There were a grand-total thirteen linear \cdr s applied to the OR-Library. 
A total of two models from \cref{ch:esmodels} (i.e. optimised w.r.t. either 
$\minCmax$ or $\minRho$) and eleven preference models from 
\cref{ch:prefmodels,ch:featselect,ch:rollout,ch:imitation}.
Their configurations are as follows
\begin{enumerate*}
    \item \label{pref:3phi} 3.524 from \cref{ch:featselect} with 
    \ref{bias:equal} sampling
    \item \label{pref:16aphi} 16.1 following expert policy and minimum 
    $\minCmax$ weights from \cref{ch:prefmodels}, both using either 
    \ref{bias:equal} or adjusted \ref{bias:dbl2nd}
    \item 16.1 following the perturbed leader from \cref{sec:perturbedLeader}  
    using either \ref{bias:equal} or adjusted \ref{bias:dbl2nd}
    \item \label{pref:16bphi} 16.1 following second iteration of unsupervised 
    DAgger with extended data from \cref{sec:il:active} with stepwise adjusted 
    \ref{bias:dbl2nd}
    \item \label{pref:20phi} 20.1 following expert policy and minimum 
    $\minCmax$ weights from \cref{sec:multirollout}, both using adjusted 
    \ref{bias:dbl2nd}
    \item \label{pref:24phi} 24.1 following minimum $\minCmax$ weights from 
    \cref{sec:multirollout}, using adjusted \ref{bias:dbl2nd}
\end{enumerate*} 
Note, \ref{pref:3phi} only uses three local features, \ref{pref:16aphi} to 
\ref{pref:16bphi} use all \NrFeatLocal\ local features, \ref{pref:20phi} use 
additional \phiSDRRelated\ roll-out features, and finally \ref{pref:24phi} 
includes 100 random roll-outs as well.\linebreak
Only a single configuration of \phiRNDRelated\ model was considered, as it is 
quite computationally expensive for OR-Library as some of the problem 
dimensions, $K$, is too great for 100 random roll-outs for each possible 
encountered operation. However, using only four fixed roll-outs is reasonable.
Moreover, when applying $(K-k)$-step lookahead, then it is sensible to keep 
track of the best solution found (even though they had not been specifically 
followed further). This was referred to as its \emph{fortified} solution, where 
$\rho_{\text{fort.}}\leq\rho$. To keep notation short, only 
$\rho_{\text{fort.}}$ is reported for models that incorporate any 
\phiGlobalRelated\ features. 
\Cref{tbl:comp:orlib} gives the deviation from best known solution. Note, not 
all problem instances from \cref{tbl:data:orlib} are reported as not all best 
known solutions were found. Instead 71 out of 82 and 18 out of 32 instances are 
reported for \JSP\ and \FSP, respectively.
Note, only best configuration of similar parameter settings was reported in 
\cref{tbl:comp:orlib}. However, \cref{tbl:stat:orlib} gives the frequency 
(as percentage) of how often each configuration managed achieved the best 
known solution. 
In the case of \FSP's \texttt{car1}, only the model using \phiRNDRelated\ 
features found that makespan. 
However, for \JSP's \texttt{la12} even the local-based feature models found the 
same makespan as well.
\Cref{tbl:stat:orlib} also summarises the division of model configuration that 
found the lowest makespan (which for \texttt{la12} and \texttt{car1} coincides 
with BKS). In addition the 1\%, 5\% and 10\% deviation from lowest found $\rho$ 
is reported. This can also be seen in the box-plot given in 
\cref{fig:comp:orlib}.

\begin{figure}[t]
\includegraphics[width=\textwidth]{{boxplot.ORLIB}.pdf}
\caption{Box-plot for deviation from best known solution, $\rho$, trained on
\jrnd{10}{10}}\label{fig:comp:orlib}
\end{figure}

\clearpage
{\footnotesize
\begin{longtable}{llcrr@{.}l@{}r}
\caption{Comparison results of OR-Library} 
\label{tbl:comp:orlib} \\
\toprule
Set & Name & $n \times m$ & BKS & \multicolumn{2}{c}{CMA-ES} & 
\multicolumn{1}{c}{$\rho$} \\ 
\endfirsthead
\multicolumn{6}{c}%
{\tablename\ \thetable\ -- \emph{Continued from previous page}} \\
\toprule
Set & Name & $n \times m$ & BKS & \multicolumn{2}{c}{CMA-ES} & 
\multicolumn{1}{c}{$\rho$} \\ 
\midrule
\endhead
\bottomrule\multicolumn{6}{r}{\emph{Continued on next page}} \\
\endfoot
\bottomrule
\endlastfoot
\midrule \Problem{abz}
& abz5 & 10$\times$10 & 1234 & \jrndn{6}{5} & $\minRho$ & 6.56 \\ 
& abz6 & 10$\times$10 & 943 & \jrnd{6}{5} & $\minRho$ & 11.03 \\ 
& abz7 & 20$\times$15 & 656 & \jrndJ{6}{5} & $\minCmax$ & 11.13 \\ 
& abz8 & 20$\times$15 & 665 & \jrnd{6}{5} & $\minCmax$ & 18.80 \\ 
& abz9 & 20$\times$15 & 678 & \jrnd{6}{5} & $\minRho$ & 14.45 \\ 
\midrule \Problem{ft}
& ft06 & 6$\times$6 &  55 & \jrnd{6}{5} & $\minRho$ & 7.27 \\ 
& ft10 & 10$\times$10 & 930 & \jrndM{6}{5} & $\minRho$ & 23.12 \\ 
& ft20 & 20$\times$5 & 1165 & \frnd{6}{5} & $\minRho$ & 9.53 \\ 
\midrule \Problem{la}
& la01 & 10$\times$5 & 666 & \jrndM{6}{5} & $\minRho$ & 1.80 \\ 
& la02 & 10$\times$5 & 655 & \jrnd{6}{5} & $\minCmax$ & 0.61 \\ 
& la03 & 10$\times$5 & 597 & \frnd{6}{5} & $\minRho$ & 19.43 \\ 
& la04 & 10$\times$5 & 590 & \jrnd{6}{5} & $\minRho$ & 3.56 \\ 
& la06 & 15$\times$5 & 926 & \jrnd{6}{5} & $\minRho$ & 10.15 \\ 
& la07 & 15$\times$5 & 890 & \jrndn{6}{5} & $\minRho$ & 1.57 \\ 
& la08 & 15$\times$5 & 863 & \fmxc{6}{5} & $\minRho$ & 12.98 \\ 
& la10 & 15$\times$5 & 958 & \jrnd{6}{5} & $\minRho$ & 0.10 \\ 
& la11 & 20$\times$5 & 1222 & \frnd{6}{5} & $\minRho$ & 1.55 \\ 
& la12 & 20$\times$5 & 1039 & \jrndJ{6}{5} & $\minRho$ & 0.00 \\ 
& la13 & 20$\times$5 & 1150 & \jrnd{6}{5} & $\minRho$ & 7.13 \\ 
& la15 & 20$\times$5 & 1207 & \jrnd{10}{10} & $\minCmax$ & 0.41 \\ 
& la16 & 10$\times$10 & 945 & \jrndn{6}{5} & $\minCmax$ & 5.71 \\ 
& la18 & 10$\times$10 & 848 & \jrndn{6}{5} & $\minCmax$ & 8.84 \\ 
& la19 & 10$\times$10 & 842 & \jrnd{6}{5} & $\minCmax$ & 10.93 \\ 
& la21 & 15$\times$10 & 1046 & \frnd{6}{5} & $\minRho$ & 27.92 \\ 
& la22 & 15$\times$10 & 927 & \jrndJ{6}{5} & $\minCmax$ & 12.84 \\ 
& la23 & 15$\times$10 & 1032 & \jrnd{6}{5} & $\minRho$ & 9.01 \\ 
& la24 & 15$\times$10 & 935 & \jrndJ{6}{5} & $\minRho$ & 12.30 \\ 
& la25 & 15$\times$10 & 977 & \jrndn{6}{5} & $\minCmax$ & 12.49 \\ 
& la26 & 20$\times$10 & 1218 & \jrnd{6}{5} & $\minRho$ & 16.01 \\ 
& la27 & 20$\times$10 & 1235 & \frnd{6}{5} & $\minRho$ & 12.47 \\ 
& la28 & 20$\times$10 & 1216 & \jrnd{10}{10} & $\minRho$ & 4.28 \\ 
& la29 & 20$\times$10 & 1152 & \jrnd{6}{5} & $\minRho$ & 18.92 \\ 
& la30 & 20$\times$10 & 1355 & \jrnd{10}{10} & $\minCmax$ & 2.14 \\ 
& la31 & 30$\times$10 & 1784 & \jrnd{6}{5} & $\minRho$ & 0.84 \\ 
& la32 & 30$\times$10 & 1850 & \jrndM{6}{5} & $\minCmax$ & 17.95 \\ 
& la33 & 30$\times$10 & 1719 & \jrndn{6}{5} & $\minCmax$ & 7.74 \\ 
& la34 & 30$\times$10 & 1721 & \jrnd{6}{5} & $\minCmax$ & 5.35 \\ 
& la35 & 30$\times$10 & 1888 & \fmxc{6}{5} & $\minRho$ & 4.03 \\ 
& la36 & 15$\times$15 & 1268 & \jrndJ{6}{5} & $\minCmax$ & 5.68 \\ 
& la37 & 15$\times$15 & 1397 & \jrndJ{6}{5} & $\minCmax$ & 6.30 \\ 
& la38 & 15$\times$15 & 1196 & \jrndM{6}{5} & $\minRho$ & 9.20 \\ 
& la39 & 15$\times$15 & 1233 & \jrnd{10}{10} & $\minCmax$ & 8.27 \\ 
& la40 & 15$\times$15 & 1222 & \jrndn{6}{5} & $\minRho$ & 11.78 \\ 
\midrule \Problem{orb}
& orb01 & 10$\times$10 & 1059 & \frnd{6}{5} & $\minRho$ & 8.88 \\ 
& orb02 & 10$\times$10 & 888 & \jrndM{6}{5} & $\minCmax$ & 4.84 \\ 
& orb03 & 10$\times$10 & 1005 & \jrnd{10}{10} & $\minRho$ & 12.24 \\ 
& orb04 & 10$\times$10 & 1005 & \frnd{6}{5} & $\minCmax$ & 5.87 \\ 
& orb05 & 10$\times$10 & 887 & \fmxc{6}{5} & $\minRho$ & 14.99 \\ 
& orb06 & 10$\times$10 & 1010 & \jrndn{6}{5} & $\minRho$ & 17.43 \\ 
& orb07 & 10$\times$10 & 397 & \jrnd{6}{5} & $\minRho$ & 14.36 \\ 
& orb08 & 10$\times$10 & 899 & \frnd{6}{5} & $\minCmax$ & 12.01 \\ 
& orb09 & 10$\times$10 & 934 & \jrndn{6}{5} & $\minRho$ & 10.39 \\ 
& orb10 & 10$\times$10 & 944 & \jrndn{6}{5} & $\minRho$ & 16.74 \\ 
\midrule \Problem{swv}
& swv01 & 20$\times$10 & 1407 & \fmxc{6}{5} & $\minRho$ & 21.25 \\ 
& swv02 & 20$\times$10 & 1475 & \frnd{6}{5} & $\minRho$ & 17.08 \\ 
& swv03 & 20$\times$10 & 1398 & \jrnd{6}{5} & $\minRho$ & 21.96 \\ 
& swv04 & 20$\times$10 & 1470 & \jrndM{6}{5} & $\minCmax$ & 26.46 \\ 
& swv05 & 20$\times$10 & 1424 & \jrndJ{6}{5} & $\minCmax$ & 20.01 \\ 
& swv06 & 20$\times$15 & 1675 & \frnd{6}{5} & $\minRho$ & 26.03 \\ 
& swv07 & 20$\times$15 & 1594 & \jrnd{10}{10} & $\minCmax$ & 23.34 \\ 
& swv08 & 20$\times$15 & 1755 & \frnd{6}{5} & $\minRho$ & 25.41 \\ 
& swv09 & 20$\times$15 & 1656 & \jrndM{6}{5} & $\minRho$ & 28.08 \\ 
& swv10 & 20$\times$15 & 1743 & \fmxc{6}{5} & $\minRho$ & 30.69 \\ 
& swv11 & 50$\times$10 & 2983 & \fmxc{6}{5} & $\minRho$ & 19.75 \\ 
& swv12 & 50$\times$10 & 2979 & \fmxc{6}{5} & $\minRho$ & 22.46 \\ 
& swv13 & 50$\times$10 & 3104 & \fmxc{6}{5} & $\minRho$ & 17.65 \\ 
& swv14 & 50$\times$10 & 2968 & \fmxc{6}{5} & $\minRho$ & 17.82 \\ 
& swv15 & 50$\times$10 & 2886 & \fmxc{6}{5} & $\minRho$ & 20.41 \\ 
& swv18 & 50$\times$10 & 2852 & \jrnd{6}{5} & $\minRho$ & 3.51 \\ 
& swv19 & 50$\times$10 & 2843 & \jrndJ{6}{5} & $\minRho$ & 5.49 \\ 
\midrule \Problem{yn}
& yn1 & 20$\times$20 & 884 & \jrnd{10}{10} & $\minRho$ & 10.52 \\ 
& yn2 & 20$\times$20 & 904 & \jrndn{6}{5} & $\minRho$ & 13.16 \\ 
& yn3 & 20$\times$20 & 892 & \jrndn{6}{5} & $\minRho$ & 16.26 \\ 
& yn4 & 20$\times$20 & 968 & \jrndn{6}{5} & $\minRho$ & 22.83 \\ 
\midrule \Problem{car}
& car1 & 11$\times$5 & 7038 & \fmxc{6}{5} & $\minRho$ & 5.80 \\ 
& car2 & 13$\times$4 & 7166 & \frndn{6}{5} & $\minCmax$ & 4.95 \\ 
& car3 & 12$\times$5 & 7312 & \jrndn{6}{5} & $\minRho$ & 4.23 \\ 
& car4 & 14$\times$4 & 8003 & \fmxc{6}{5} & $\minRho$ & 5.25 \\ 
& car6 & 8$\times$9 & 8505 & \frnd{6}{5} & $\minCmax$ & 8.01 \\ 
& car7 & 7$\times$7 & 6590 & \jrndn{6}{5} & $\minCmax$ & 2.49 \\ 
& car8 & 8$\times$8 & 8366 & \frnd{6}{5} & $\minRho$ & 8.39 \\ 
\midrule \Problem{hel}
& hel2 & 20$\times$10 & 136 & \fmxc{6}{5} & $\minRho$ & 9.56 \\ 
\midrule \Problem{reC}
& reC07 & 20$\times$10 & 1566 & \fmxc{6}{5} & $\minRho$ & 6.83 \\ 
& reC09 & 20$\times$10 & 1537 & \fmxc{6}{5} & $\minRho$ & 11.39 \\ 
& reC11 & 20$\times$10 & 1431 & \jrnd{6}{5} & $\minCmax$ & 10.90 \\ 
& reC13 & 20$\times$15 & 1930 & \jrndn{6}{5} & $\minRho$ & 13.21 \\ 
& reC15 & 20$\times$15 & 1950 & \jrnd{10}{10} & $\minRho$ & 9.23 \\ 
& reC17 & 20$\times$15 & 1902 & \jrnd{6}{5} & $\minCmax$ & 18.03 \\ 
& reC25 & 30$\times$15 & 2513 & \fmxc{6}{5} & $\minRho$ & 18.30 \\ 
& reC29 & 30$\times$15 & 2287 & \frnd{6}{5} & $\minRho$ & 16.53 \\ 
& reC31 & 50$\times$10 & 3045 & \fmxc{6}{5} & $\minRho$ & 16.49 \\ 
& reC33 & 50$\times$10 & 3114 & \jrnd{6}{5} & $\minRho$ & 12.65 \\ 
\end{longtable}}

\begin{table}
\noindent
\caption[Frequency of finding best makespan]{Frequency ($\%$) a model 
configuration
    found the best makespan}
\centering
{\setlength{\tabcolsep}{3pt}
\begin{tabular}{llllrrrrrr}
\toprule
& & & & & & \multicolumn{4}{c}{w.r.t. lowest found $C_{\max}$} \\ 
\cmidrule(r){7-10}
& Type & CDR & Configuration & $N$ & BKS & $=0$ & $\leq1$ & $\leq5$ & $\leq10$ 
\\ 
\midrule
\multirow{13}{*}{\rot{\JSP}}
& PREF & 3.524 & OPT.equal & 71 & 0.00 & 1.41 & 2.82 & 18.31 & 49.30 \\ 
& PREF & 16.1 & OPT.adjdbl2nd & 71 & 0.00 & 0.00 & 0.00 & 0.00 & 1.41 \\ 
& PREF & 16.1 & $\minCmax$.adjdbl2nd & 71 & 0.00 & 5.63 & 8.45 & 23.94 & 61.97 
 \\ 
& PREF & 16.1 & OPT$\epsilon$.adjdbl2nd & 71 & 1.41 & 7.04 & 7.04 & 12.68 & 
32.39 \\ 
& PREF & 16.1 & DA2UNSUP.adjdbl2nd & 71 & 1.41 & 5.63 & 7.04 & 12.68 & 42.25 
 \\ 
& PREF & 16.1 & OPT.equal & 71 & 0.00 & 0.00 & 0.00 & 0.00 & 1.41 \\ 
& PREF & 16.1 & OPT$\epsilon$.equal & 71 & 0.00 & 0.00 & 0.00 & 0.00 & 4.23 \\ 
& PREF & 16.1 & $\minCmax$.equal & 71 & 0.00 & 0.00 & 0.00 & 2.82 & 21.13 \\ 
& PREF & 20.1 & OPT.adjdbl2nd & 71 & 0.00 & 2.82 & 12.68 & 30.99 & 76.06 \\ 
& PREF & 20.1 & $\minCmax$.adjdbl2nd & 71 & 1.41 & 15.49 & 22.54 & 52.11 & 
 84.51 \\ 
& PREF & 24.1 & $\minCmax$.adjdbl2nd & 71 & 1.41 & 87.32 & 90.14 & 100.00 & 
 100.00 \\ \cmidrule(r){2-10}
 & CMA-ES & 16.1 & $\minCmax$ & 71 & 1.41 & 8.45 & 15.49 & 32.39 & 59.15 \\ 
 & CMA-ES & 16.1 & $\minRho$ & 71 & 1.41 & 9.86 & 15.49 & 33.80 & 59.15 \\ 
\midrule \multirow{13}{*}{\rot{\FSP}}
& PREF & 3.524 & OPT.equal & 18 & 0.00 & 5.56 & 5.56 & 11.11 & 55.56 \\ 
& PREF & 16.1 & OPT.adjdbl2nd & 18 & 0.00 & 0.00 & 0.00 & 0.00 & 0.00 \\ 
& PREF & 16.1 & $\minCmax$.adjdbl2nd & 18 & 0.00 & 0.00 & 0.00 & 5.56 & 55.56 
 \\ 
& PREF & 16.1 & OPT$\epsilon$.adjdbl2nd & 18 & 0.00 & 0.00 & 0.00 & 0.00 & 
11.11 \\ 
& PREF & 16.1 & DA2UNSUP.adjdbl2nd & 18 & 0.00 & 0.00 & 0.00 & 0.00 & 38.89 
 \\ 
& PREF & 16.1 & OPT.equal & 18 & 0.00 & 0.00 & 0.00 & 0.00 & 0.00 \\ 
& PREF & 16.1 & OPT$\epsilon$.equal & 18 & 0.00 & 0.00 & 0.00 & 0.00 & 0.00 \\ 
& PREF & 16.1 & $\minCmax$.equal & 18 & 0.00 & 0.00 & 0.00 & 16.67 & 38.89 \\ 
& PREF & 20.1 & OPT.adjdbl2nd & 18 & 0.00 & 0.00 & 11.11 & 16.67 & 72.22 \\ 
& PREF & 20.1 & $\minCmax$.adjdbl2nd & 18 & 0.00 & 11.11 & 27.78 & 61.11 & 
83.33 \\ 
& PREF & 24.1 & $\minCmax$.adjdbl2nd & 18 & 5.56 & 83.33 & 88.89 & 100.00 &  
100.00 \\ \cmidrule(r){2-10}
 & CMA-ES & 16.1 & $\minCmax$ & 18 & 0.00 & 0.00 & 5.56 & 22.22 & 66.67 \\ 
 & CMA-ES & 16.1 & $\minRho$ & 18 & 0.00 & 0.00 & 11.11 & 33.33 & 72.22 \\ 
\bottomrule
\end{tabular}}
\end{table}

%   Type   CDR   N  BKS best_pct   <=1    <=5   <=10
%   PREF 3.524  89 0.00     2.25  3.37  16.85  50.56
%   PREF  16.1 623 0.32     2.09  2.57   6.58  22.95
%   PREF  20.1 178 0.56     8.43 17.98  41.01  79.78
%   PREF  24.1  89 2.25    86.52 89.89 100.00 100.00
% CMA-ES  16.1 178 1.12     7.30 14.04  32.02  61.24

\section{Conclusions}
Of the thirteen linear models considered, then the one using random roll-outs 
(i.e. 24.1) was the consistently best one, which is to be expected as it has 
the most computational effort in inspecting possible solutions for each problem 
instance.
Although it did not always find the lowest makespan of the models considered 
(it did for 86.52\% instances), it was within 5\% error.
The second best configuration was using \phiSDRRelated\ features. This is also 
to be expected as it uses four fixed roll-outs. 
Notice in \cref{tbl:comp:orlib}, when 24.1 is not the best model then the 20.1 
model is best configuration (with very few exceptions). 
Of its two configurations then following minimum $\minCmax$ proved better than 
the one based on following expert policy. 
This concurs with the findings for the preference models using only just 
\phiLocalRelated\ features. 

Furthermore, the model using only three local features has a remarkably low 
$\rho$ compared to the other more sophisticated \phiLocalRelated\ preference 
models. This is probably due to the fact that during training the model has  
the \jrnd{10}{10} problem space specifically in mind. So when we test the 
trained models on completely different problem spaces then incorporating so 
many features may not necessarily be representing for those new instances. 
However, for a three-feature model, its robustness capabilities are somewhat 
more as it's focusing its efforts on representing the `essence' of \jsp\ 
instead of paying particular attention to the problem space's individuality. 

From the robustness experiments using the OR-Library benchmark suite, we see 
that the results from the training data to the corresponding test data hold 
when tested on completely different problem spaces, both with respect to data 
distribution and dimensionality.
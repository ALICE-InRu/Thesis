\HeaderQuote{I don't believe there's an atom of meaning in it.}{Alice}

\chapter{Analysing Optimal Solutions}\label{ch:analysingopt} 
%\section{Probability of choosing optimal decision}\label{sec:diff:opt:rnd}
\FirstSentence{I}{n order to create successful \dr s}, a good starting point is 
to investigate the properties of optimal solutions and hopefully be able to 
learn how to mimic such `good' behaviour. 
For this, we follow an optimal solution,\footnote{Optimal solutions are 
obtained by using a commercial software package \cite{gurobi}.} and inspect the 
evolution of its features  (defined in \cref{tbl:features}) throughout the 
dispatching process. 
Moreover, it is noted, that there are several optimal solutions available for 
each problem instance. However, it is deemed sufficient to inspect only one 
optimal trajectory per problem instance as there are $N_{\text{train}}$ 
independent instances which gives the training data variety. 

Firstly, we can observe that on a step-by-step basis there are several optimal 
dispatches to choose from. \Cref{fig:diff:opt:unique} depicts how the number of 
optimal dispatches evolve at each dispatch iteration. Note, that only one 
optimal trajectory is pursued (chosen at random), hence this is only a lower 
bound of uniqueness of optimal solutions.
As the number of possible dispatches decrease over time, \cref{fig:diff:opt} 
depicts the probability of choosing an optimal dispatch at each iteration. 

\section{Making suboptimal decisions}\label{sec:diff:opt:sub}
Looking at \cref{fig:diff:opt}, \jrnd{10}{10} has a relatively high probability 
($70\%$ and above) of choosing an optimal job. However, it is imperative to 
keep making optimal decisions, because once off the optimal track the 
consequences can be dire. To demonstrate this interaction \cref{fig:diff:case} 
depicts the worst and best case scenario of \namerho, once you've fallen off 
the optimal track. Note, that this is given that you make \emph{one} wrong 
turn. Generally, there will be many mistakes made, and then the compound 
effects of making suboptimal decisions really start adding up. 

It is interesting that for \JSP, that over time making suboptimal decisions 
make more of an impact on the resulting makespan. This is most likely due to 
the fact that if a suboptimal decision is made in the early stages, then there 
is space to rectify the situation with the subsequent dispatches. However, if 
done at a later point in time, little is to be done as the damage is already 
been inflicted upon the schedule. 
However, for \FSP, the case is the exact opposite. Under those circumstances 
it's imperative to make good decisions right from the beginning. This is due to 
the major structural differences between \jsp\ and \fsp, namely the latter 
having a homogeneous machine ordering, constricting the solution immensely. 
Luckily, this does have the added benefit of making \fsp\ less vulnerable for 
suboptimal decisions later in the decision process. 


\section{Optimality of simple priority dispatching 
rules}\label{sec:diff:opt:sdr}
The probability of optimality of the aforementioned SDRs from \cref{sec:SDR}, 
yet still maintaining our optimal trajectory, i.e., the probability of a job 
chosen by a SDR being able to yield an optimal makespan on a step by step 
basis, is depicted  in   \cref{fig:diff:opt:SDR}. Moreover, the dashed line 
represents the benchmark of randomly guessing the optimum (cf. 
\cref{fig:diff:opt}).

Now, let's bare in mind \namerho, of applying SDRs throughout the dispatching 
process (cf. box-plots of which in \cref{fig:SDR:boxplot}), then there is a 
some correspondence between high probability of stepwise optimality and low 
$\rho$. Alas, this isn't always the case, for \jrnd{10}{10} SPT always 
outperforms LPT w.r.t. stepwise optimality, however, this does not transcend to 
SPT having a lower $\rho$ value than LPT. Hence, it's not enough to just learn 
optimal behaviour, one needs to investigate what happens once we encounter 
suboptimal state spaces.

\section{Simple blended dispatching rule}\label{sec:diff:opt:bdr}
As stated before, the goal of this \lcnamecref{ch:problemstructure} is to 
utilise feature behaviour to motivate new, and \emph{hopefully} better, 
dispatching rules. 
A na\"ive approach would be creating a simple blended dispatching rule which 
would be for instance switch between two SDRs at a predetermined time point. 
Hence, going back to \cref{fig:diff:opt:SDR} a presumably good BDR for 
\jrnd{10}{10}  would be starting with SPT and then switching over to MWR at 
around time step $k=40$, where the SDRs change places in outperforming one 
another. A box-plot for \namerho, for all $10\times10$ problem spaces is 
depicted in \cref{fig:diff:boxplot:BDR}. This little manipulation between SDRs 
does outperform SPT immensely, yet doesn't manage to gain the performance edge 
of MWR, save for \frnd{10}{10}. This gives us insight that for \jsp, the 
attribute based on MWR is quite fruitful for good dispatches, whereas the same 
cannot be said about SPT -- a more sophisticated BDR is needed to improve upon 
MWR. 

A reason for this lack of performance of our proposed BDR is perhaps that by 
starting out with SPT in the beginning, it sets up the schedules in such a way 
that it's quite greedy and only takes into consideration jobs with shortest 
immediate processing times. Now, even though it is possible to find optimal 
schedules from this scenario, as \cref{fig:diff:opt:SDR} shows, the inherent 
structure is already taking place, and might make it hard to come across 
optimal moves by simple methods. Therefore it's by no means guaranteed that by 
simply swapping over to MWR will handle the situation that applying SPT has 
already created. \Cref{fig:diff:boxplot:BDR} does however, show, that by 
applying MWR instead of SPT in the latter stages, does help the schedule to be 
more compact w.r.t. SPT. However, in the case of \jrnd{10}{10} and 
\jrndn{10}{10} the fact remains that the schedules have diverged too far from 
what MWR would have been able to achieve on its own. Preferably the blended 
dispatching rule should use  best of both worlds, and outperform all of its 
inherited DRs, otherwise it goes without saying, one would simply still use the 
original DR that achieved the best results.

\begin{figure}
  \centering
  \includegraphics[width=1\linewidth]{figures/{boxplotRho.BDR.10x10}.pdf}
  \caption{Box plot for \namerho, for BDR where SPT is applied for the first 
  40\% of the dispatches, followed by MWR.}
  \label{fig:diff:boxplot:BDR}
\end{figure}

\section{Extremal feature}\label{sec:diff:opt:ext}
The SDRs we've inspected so-far are based on two features from 
\cref{tbl:features}, namely
\begin{enumerate*}
  \item \phiproc\ for SPT and LPT
  \item \phijobWrm\ for LWR and MWR 
\end{enumerate*}
By choosing the lowest value for the first SDR, and highest value for the 
latter SDR, i.e., the extremal values for those given features. Let's apply the 
same methodology from \cref{sec:diff:opt:sdr} to all varying features described 
in \cref{tbl:features}. 
\Cref{fig:diff:opt:minmax}
depict the probability of all extremal features being an optimal dispatch, with 
random guessing from \cref{fig:diff:opt} as a dashed line. 
\todo[inline]{Discuss more?}

\section{Feature evolution}\label{sec:diff:opt:evol}
In order to put the extremal features into perspective, it's worth comparing 
them with how the evolution of the features are over time, depicted in 
\cref{fig:diff:opt:evol}. 


\section{Emergence of problem difficulty}\label{sec:diff:stepwise}


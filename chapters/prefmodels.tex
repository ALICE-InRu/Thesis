\HeaderQuote{It was much pleasanter at home, when one wasn't always growing larger and smaller, and being ordered about by mice and rabbits.}{Alice} 
%\HeaderQuote{The adventures first\dots explanations take such a dreadful time.}{The Gryphon} 

\chapter{Preference Learning of CDRs}\label{ch:prefmodels} 
\FirstSentence{L}{earning models considered in this } dissertation are based on ordinal regression in which the learning task is formulated as learning preferences. In the case of scheduling, learning which operations are preferred to others. Ordinal regression has been previously presented in \cite{Ru06:PPSN}, and given in \cref{ch:ordinal} for completeness. 

\section{Ordinal regression for \jsp}
Let $\vphi_{o}\in\mathcal{F}$ denote the post-decision state when dispatching $J_o$ corresponds to an optimal schedule being built. All post-decisions states corresponding to suboptimal dispatches, $J_s$, are denoted by $\vphi_{s}\in\mathcal{F}$. One could label which feature sets were considered optimal, $\vec{z}_{o}=\vphi_{o}-\vphi_{s}$, and suboptimal, $\vec{z}_{s}=\vphi_{s}-\vphi_{o}$ by $y_o=+1$ and $y_s=-1$ respectively. 
Note, a negative example is only created as long as $J_s$ actually results in a worse makespan, i.e., $C_{\max}^{(s)}\gneq C_{\max}^{(o)}$, since there can exist situations in which more than one operation can be considered optimal.

The preference learning problem is specified by a set of preference pairs,
\begin{equation}\label{eq:Psi:jsp}
	\Psi := \bigcup_{\{\vec{x}_i\}_{i=1}^{N_{\text{train}}}}
    \condset{
        \left\{\vec{z}_o,+1\right\},\left\{\vec{z}_s,-1\right\}}{\forall (o,s) 
        \in \mathcal{O}^{(k)} \times \mathcal{S}^{(k)}}_{k=1}^K 
    \subset \Phi\times Y 
\end{equation}
where $\Phi\subset\mathcal{F}$ is the training set of $d$ features, 
$Y=\{-1,+1\}$ is the outcome space, and at dispatch $k\in\{1,..,K\}$, 
$o\in\mathcal{O}^{(k)},~s\in \mathcal{S}^{(k)}$
are optimal and suboptimal dispatches, respectively.
Note, $\mathcal{O}^{(k)}\cup\mathcal{S}^{(k)}=\mathcal{L}^{(k)}$, and 
$\mathcal{O}^{(k)}\cap\mathcal{S}^{(k)}=\emptyset$. 

For \jsp\ there are $d=\NrFeatLocal$ features (cf. the local features from 
\cref{tbl:features}), and the training set is created in the 
manner described in \cref{ch:gentrdat}.

\todoFind{Which is being used, ordinal or logistic regression?}
Logistic regression makes decisions regarding optimal dispatches and at the 
same time efficiently estimates a posteriori probabilities. When using linear 
classification model (cf. \cref{sec:ord:linpref}) for \cref{eq:CDR:feat},
then the optimal $\vec{w}^*$ obtained from the preference set can be used on 
any new data point (i.e. partial schedule), $\vchi$, and their inner product is 
proportional to probability estimate \cref{eq:prob}. 
%Similarly for non-linear classification models. 
Hence, for each job on the job-list, $J_j\in\mathcal{L}$, let $\vphi_j$ denote 
its corresponding  post-decision state. Then the job chosen to be dispatched, 
$J_{j^*}$, is the one corresponding to the highest preference estimate, i.e.,
\begin{equation}\label{eq:lin}
	J_{j^*}=\argmax_{J_j\in \mathcal{L}}\; \hat{\pi}(\vphi_j)
\end{equation}
where $\hat{\pi}(\cdot)$ is the classification model obtained by the preference 
set, $\Psi$, defined by \cref{eq:Psi:jsp}. 


\section{Interpreting linear classification models}

When using a feature space based on SDRs, the linear classification models can 
very easily be interpreted as composite dispatching rules with predetermined 
weights. 
%Similarly for non-linear classification models, however, they are harder to 
%visualise since the features have to be projected on a kernel.
Looking at the features description in \cref{tbl:features} it is possible for 
the ordinal regression to `discover' the weights $\vec{w}$ in order for 
\cref{eq:CDR:feat} corresponding applying a \sdr~from \cref{sec:SDR}. 

The optimum makespan is known for each problem instance. 
At each time step (i.e. layer of the game tree) a number of feature pairs are created. 
They consist of the features $\vphi_o$ resulting from optimal dispatches $o\in\mathcal{O}^{(k)}$, versus features $\vphi_s$ resulting from suboptimal dispatches $s\in\mathcal{S}^{(k)}$ at time $k$. 
In particular, each job is compared against another job of the job-list, 
$\mathcal{L}^{(k)}$, and if the makespan differs, i.e., $C_{\max}^{(s)}\gneq 
C_{\max}^{(o)}$, an optimal/suboptimal pair is created. 
However if the makespan would be unaltered, the pair is omitted since they give the same optimal makespan. 
This way, only features from a dispatch resulting in a suboptimal solution is labelled undesirable.

The approach taken here is to verify analytically, at each time step, by retaining the current temporal schedule as an initial state, whether it can indeed \emph{somehow} yield an optimal schedule by manipulating the remainder of the sequence. 
This also takes care of the scenario that having dispatched a job resulting in a different temporal makespan would have resulted in the same final makespan if another optimal dispatching sequence would have been chosen. 
That is to say the data generation takes into consideration when there are multiple optimal solutions to the same problem instance. 

\section{Time dependant dispatching rules}
At each dispatch iteration $k$, a number of preference pairs are created, which is then repeated for all the $N_{\text{train}}$ problem instances created. 
A separate data set is deliberately created for each dispatch iterations, as 
the initial feeling is that dispatch rules used in the beginning of the 
schedule building process may not necessarily be the same as in the middle or 
end of the schedule. As a result there are $K$ linear scheduling rules for 
solving a $n \times m$ \jsp. \todoWrite{No longer one model for all 
steps}



\section{Selecting preference pairs}\label{sec:trdat:param}
Defining the size of the preference set as $l=\abs{\Psi}$, then 
\cref{eq:prefset} gives the size of the feature training set as 
$\abs{\Phi}=\frac{1}{2}l$.
If $l$ is too large, than sampling needs to be done in order for the ordinal 
regression in \cref{ch:prefmodels} to be computationally feasible.

The strategy approached in  \cref{InRu11a} was to follow a \emph{single} 
optimal job $J_j\in\mathcal{O}^{(k)}$ (chosen at random), thus creating 
$\abs{\mathcal{O}^{(k)}}\cdot\abs{\mathcal{S}^{(k)}}$ feature pairs at each 
dispatch $k$, resulting in a training size of,
\begin{equation}\label{eq:sizePsi_b}
l =  \sum_{i=1}^{N_{\text{train}}} \left(2 \abs{\mathcal{O}^{(k)}_i}\cdot 
\abs{\mathcal{S}^{(k)}_i} \right)
\end{equation}
For the problem spaces considered there, that sort of simple sampling of the 
state space was sufficient for a favourable outcome. However for a considerably 
harder problem spaces (see \cref{ch:defdifficulty}), preliminary experiments 
were not satisfactory. 

A brute force approach was adopted to investigate the feasibility of finding 
optimal weights $\vec{w}$ for \cref{eq:CDR:feat}. 
By applying CMA-ES (discussed thoroughly in \cref{ch:esmodels}) to directly 
minimize the mean $C_{\max}$  w.r.t. the weights $\vec{w}$, gave a considerably 
more favourable result in predicting optimal versus suboptimal dispatching 
paths. 
So the question put forth is, why was the ordinal regression not able to detect 
it?
The nature of the CMA-ES is to explore suboptimal routes until it converges to 
an optimal one. 
Implying that the previous approach of only looking into one optimal route is 
not sufficient information. 
Suggesting that the training set should incorporate a more complete knowledge 
about \emph{all} possible preferences, i.e., make also the distinction between 
suboptimal and sub-suboptimal features, etc.  
This would require a Pareto ranking for the job-list, $\mathcal{L}$, which can 
be used to make the distinction to which feature sets are equivalent, better or 
worse, and to what degree (i.e. giving a weight to the preference)? 
By doing so, the training set becomes much greater, which of course would again 
need to be sampled in order to be computationally feasible to learn. 

For instance \cite{Siggi05} used decision trees to `rediscover' LPT by using 
the dispatching rule to create its training data. The limitations of using 
heuristics to label the training data is that the learning algorithm will mimic 
the original heuristic (both when it works poorly and well on the problem 
instances) and does not consider the real optimum. In order to learn new 
heuristics that can outperform existing heuristics then the training data needs 
to be correctly labelled. This drawback is confronted in 
\citep{Malik08,Russell09,Siggi10} by using an optimal scheduler, computed 
off-line. 

These aspects are the main motivation for the data generation in this 
dissertation. 
All problem instances are correctly labelled w.r.t. their optimum makespan, 
found with analytical means. %\footnote{Optimal solution were found using 
%   \cite{gurobi}, a commercial software package for solving large-scale linear 
%   optimization and a state-of-the-art solver for mixed integer programming.} 
In order to create training instances (and subsequently preference pairs) both 
a features resulting in optimal solutions are gathered (following optimal 
trajectories) and features that would have been chosen if a dispatching rule 
had been implemented (following DR trajectories). 
In the latter case, the trajectories pursued here, will be the SDRs from 
\cref{sec:SDR} as well as randomly dispatching operations.

To summarise, one needs to consider two main aspects of the generation of the 
training data
\begin{enumerate*}[label={\emph{\Roman*})},
    itemjoin={{? }}, itemjoin*={{? And }}, after={{}}]
    \item what sort of rankings should be compared during each step
    \item which path(s) should be investigated
    \begin{enumerate*}[label=\textit{\Roman{enumi}.\roman*)}, before={{? }},
        itemjoin={{? }}, itemjoin*={{? Or }}, after={{}}]
        \item Pursuing solely optimal trajectories
        \item Creating random dispatches
        \item following other means: CMA-ES computed weights, \sdr s, etc.
    \end{enumerate*}
\end{enumerate*}

\subsection{Ranking strategies}\label{sec:trdat:param:ranks}
The following ranking strategies were implemented for adding preference pairs 
to $\Psi$ defined by \cref{eq:Psi:jsp}, they were first reported in 
\cref{InRu15a},
\begin{description}
    \item[Basic ranking, $\Psi_b$,] i.e., all optimum rankings $r_1$ versus all 
    possible suboptimum rankings $r_i$, $i\in\{2,\ldots,n'\}$, preference pairs 
    are added -- same basic set-up introduced in \cref{InRu11a}. Note, 
    $\abs{\Psi_b}$ is defined in \cref{eq:sizePsi_b}.
    \item[Full subsequent rankings, $S_f$,] i.e., all possible combinations of 
    $r_i$ and $r_{i+1}$ for $i\in\{1,\ldots,n'\}$, preference pairs are added.
    \item[Partial subsequent rankings, $\Psi_p$,] i.e., sufficient set of 
    combinations of $r_i$ and $r_{i+1}$ for $i\in\{1,\ldots,n'\}$, are added to 
    the training set -- e.g. in the cases that there are more than one 
    operation with the same ranking, only one of that rank is needed to 
    compared to the subsequent rank. Note that $\Psi_p\subset \Psi_f$.
    \item[All rankings, $\Psi_a$,] denotes that all possible rankings were 
    explored, i.e.,
    $r_i$ versus $r_j$ for $i,j\in\{1,\ldots,n'\}$ and $i\neq j$, preference 
    pairs are added.
\end{description}
where $r_1>r_2>\ldots>r_{n'}$ ($n'\leq n$) are the rankings of the job-list, 
$\mathcal{L}^{(k)}$, at time step $k$.


\subsection{Experimental study}\label{sec:sec:trdat:param:expr}
To test the validity of different ranking and strategies from 
\cref{sec:trdat:param}, a training set of $N_{\text{train}}$ problem instances 
of $6\times5$ \jsp\ and \fsp\ summarised in \cref{tbl:data} (omitting the job 
and machine variations of \jsp). The size of the preference set, $\Psi$, for 
different trajectory and ranking strategies are depicted in 
\cref{fig:sizeofprefset}. Note, for now $10\times10$ problem spaces will be 
ignored, due to the extreme computational cost of correctly labelling each 
trajectory. However, in \cref{sec:pref:scalability} an optimum path will be 
pursued for said dimension.

A linear preference (PREF) model was created for each preference set, $\Psi$, 
for every problem space considered. A box-plot with \fullnamerho, is presented 
in \cref{fig:boxplot:prefset}. %\Cref{fig:track:boxplot:p1} depicts different 
%ranking strategies for a fixed trajectory, whereas \cref{fig:rank:boxplot:p1} 
%depicts different trajectory strategies for a fixed ranking. 
From the figures it is apparent there can be a performance edge gained by 
implementing a particular ranking or trajectory strategy, moreover the 
behaviour is analogous across different disciplines. 



\begin{figure}[p]
    \includegraphics[width=\textwidth]{{prefdat.size.6x5}.pdf}
    \caption[Size of preference set, $\abs{\Psi}$]{Size of preference set, 
        $l=\abs{\Psi}$, for different trajectory strategies and ranking schemes 
        (where 
        $N_{\text{train}}=500$) }
    \label{fig:size:prefset}
\end{figure}

\begin{figure}[p]
    \includegraphics[width=\textwidth]{{prefdat.boxplot.6x5}.pdf}
    \caption{Box-plot for various $\Phi$ and $\Psi$ set-up. The trajectories 
        the models are based on are depicted in white on the far right.}
    \label{fig:boxplot:prefset}
\end{figure}

\todoFind{Add CMA-ES baseline to \cref{fig:boxplot:prefset}}


\subsubsection{Ranking strategies}
There is no statistical difference between $\Psi_f$ and $\Psi_p$ 
ranking-schemes 
across all disciplines (cf. \cref{fig:track:boxplot:p1,fig:track:boxplot:p2}), 
which is expected since $\Psi_f$ is designed to contain the same preference 
information as $\Psi_f$. However neither of the Pareto ranking-schemes 
outperform 
the original $\Psi_b$ set-up from \cref{InRu11a}. The results hold for the test 
set as well. 

Combining the ranking schemes, $\Psi_a$, improves the individual 
ranking-schemes across all disciplines, except in the case of 
$\Psi_b^{opt}\big|_{\mathcal{P_1}}$ and $\Psi_b^{rnd}\big|$\!\Problem{2}, in 
which 
case there were no statistical difference. Now, for the test set, the results 
hold, however there is no statistical difference between $\Psi_b$ and $\Psi_a$ 
for most trajectories $\{\Psi^{opt},\Psi^{cma},\Psi^{rnd}\}\big|$\!\Problem{1} 
and  $\{\Psi^{opt},\Psi^{rnd}\}\big|$\!\Problem{2}. Now, whereas a smaller 
preference set is preferred, its opted to use the $\Psi^{b}$ ranking scheme 
henceforth. 

Moreover, it is noted that the learning algorithm is able to significantly 
outperform the original heuristics, MWR and CMA-ES (white), used to create the 
training data $\Psi^{mwr}$ (grey) and $\Psi^{cma}$ (yellow), respectively (cf. 
\cref{fig:track:boxplot:p1,fig:track:boxplot:p2}). For both \Problem{1} and 
\Problem{2}, linear ordinal regression models based on $\Psi^{mwr}$ are 
significantly better than $MWR$, irrespective of the ranking schemes. Whereas 
the fixed weights found via CMA-ES are only outperformed by linear ordinal 
regression models based on $\{\Psi_b^{cma},\Psi_a^{cma}\}$. This implies that 
ranking scheme needs to be selected appropriately. Result hold for the test 
data.

\subsubsection{Trajectory strategies}
Learning preference pairs from a good scheduling policies, such as $\Psi^{cma}$ 
and $\Psi^{mwr}$, gave considerably more favourable results than tracking 
optimal paths (cf. \cref{fig:track:boxplot:p1,fig:track:boxplot:p2}). 
Suboptimal routes are preferred when dealing with problem{1} (for all ranking 
schemes), however when encountering problem{2} the choice of ranking schemes 
can yield the exact opposite.

It is particularly interesting there is no statistical difference between 
$\Psi^{opt}$ and $\Psi^{rnd}$ for both $\{\Psi_{b},\Psi_{f}\}\big|$\Problem{1} 
and 
$\{\Psi_b,\Psi_f,\Psi_p\}\big|$\Problem{2} ranking-schemes. That is to say, 
tracking 
optimal dispatches gives the same performance as completely random dispatches. 
This indicates that exploring only optimal trajectories can result in a 
training set which the learning algorithm is inept to determine good dispatches 
in the circumstances when newly encountered features have diverged from the 
learned feature set labelled to optimum solutions. 

Finally, $\Psi^{all}$ gave the best combination across all disciplines. Adding 
suboptimal trajectories with the optimal trajectory gives the learning 
algorithm a greater variety of preference pairs for getting out of local minima.

\subsubsection{Following CMA-ES guided trajectory}
The rational for using the $\Psi^{cma}$ strategy was mostly due to the fact a 
linear classifier is creating the training data (using the weights found via 
CMA-ES optimisation), hence the training data created should be linearly 
separable, which in turn should boost the training accuracy for a linear 
classification learning model. However, this strategy is easily outperformed by 
the single priority based dispatching rule MWR guiding the training data 
collection, $\Psi^{mwr}$. 


\subsection{Summary and conclusion}
As the experimental results showed in \cref{sec:trdat:param:expr}, the ranking 
of optimal\footnote{Here the tasks labelled `optimal' do not necessarily yield 
    the optimum makespan (except in the case of following optimal 
    trajectories), 
    instead these are the optimal dispatches for the given partial schedule.} 
    and 
suboptimal features are of paramount importance. The subsequent rankings are 
not of much value, since they are disregarded anyway. However, the trajectories 
to create training instances have to be varied.

Unlike \citep{Siggi10,Malik08,Russell09}, learning only on optimal training 
data was not fruitful. However, inspired by the original work by 
\cite{Siggi05}, having DR guide the generation of training data (except 
correctly labelling with analytic means) gave meaningful preference pairs which 
the learning algorithm could learn. In conclusion, henceforth, the training 
data will be generate with $\Psi_{b}^{all}$ scheme.



\section{Time independent dispatching rules}\label{sec:pref:scalability}
As stated in \cref{sec:\Psi:strategies}, a separate data set is deliberately 
created for each dispatch iteration, as it is initially assumed that dispatch 
rules used in the schedule building might differ in the beginning, the middle 
or towards the end of the process. As a result there is a local linear model 
for each dispatch; a total of $K$ linear models for solving $n\times m$ 
\jsp. Now, if we were to create a global rule, then there would have to be one 
model for all dispatches iterations. The approach in \cref{InRu11a} was to take 
the mean weight for all stepwise linear models, i.e., 
$\bar{w}_i=\frac{1}{K}\sum_{k=1}^K w_i^{(k)}$ where $\vec{w}^{(k)}$ is 
the linear weight resulting from learning preference set $\Psi^{(k)}$ at 
dispatch $k$. 

A more sophisticated way, would be to create a \emph{new} linear model, where 
the preference set, $\Psi$, is the union of the preference pairs across the $K$ 
dispatches. This would amount to a substantial training set, and for $\Psi$ to 
be computationally feasible to learn, $\Psi$ has to be filtered to size 
$l_{\max}$.

\todoWrite{No longer the case, one model instead of $K$ stepwise-models}

\section{Discussion and conclusions}



\section{JOH.tex}

\subsection{Feature Selection}
The SDRs we've inspected so-far are based on two job-attributes from
\cref{tbl:jssp:feat}, namely
\begin{enumerate*}[after={{,}}]
  \item \phiproc\ for SPT and LPT 
  \item \phijobWrm\ for LWR and MWR 
\end{enumerate*}
by choosing the lowest value for SPT and LWR, and highest value for LPT and 
MWR, i.e., the extremal values for those attributes. 
There is nothing that limits us to using just only these two attributes. 

For this study we will consider all combinations of attributes using either one,
two, three or all $d$ of them, for a total of
$\nchoosek{d}{1}+\nchoosek{d}{2}+\nchoosek{d}{3}+\nchoosek{d}{d}$, i.e., total
of 697 combinations. The reason for such a limiting number of active attributes,
are due to the fact we want to keep the models simple enough for improved model
interpretability.

For each feature combination, a linear preference model is created, where 
$\Phi$ is limited to the predetermined feature combination. 
This was done with the software package from
\cite{liblinear},\footnote{Software available at
  \url{http://www.csie.ntu.edu.tw/~cjlin/liblinear}}
by training on the full preference set $\Psi$ obtained from 
\mbox{$N_{\text{train}}=300$} problem instances following the framework set up 
in \cref{sec:liblinear}. 
Note, in order to report the validation accuracy, 20\% ($N_{\text{val}}=60$) of 
the training set was set aside for validation of reporting the accuracy.

\subsection{Validation accuracy}\label{sec:CDR:acc}
As the preference set $\Psi$ has both preference pairs belonging to optimal
ranking, and subsequent rankings, it is not of primary importance to classify
\emph{all} rankings correctly, just the optimal ones. Therefore, instead of
reporting the validation accuracy based on the classification problem of the
correctly labelling the problem set $\Psi$, it's opted the validation accuracy 
is
obtained in the same manner as done in \cref{sec:opt:sdr} for SDRs, i.e., the
probability of choosing optimal decision given the resulting linear weights,
however, in this context, the mean throughout the dispatching process is
reported. \Cref{fig:stepwise_vs_classification} shows the difference between
the two measures of reporting validation accuracy. Validation accuracy based on
stepwise optimality only takes into consideration the likelihood of choosing
the optimal move at each time step. However, the classification accuracy is
also trying to correctly distinguish all subsequent rankings in addition of
choosing the optimal move, as expected that measure is considerably lower. 

\begin{figure}[th!]
  \centering
  \includegraphics[width=\linewidth]{figures/{training.accuracy}.pdf}
  \caption{Various methods of reporting validation accuracy for preference 
    learning}
  \label{fig:stepwise_vs_classification}
\end{figure}

\subsection{Pareto front}\label{sec:CDR:pareto}
When training the learning model one wants to keep the validation accuracy 
high, as that would imply a higher likelihood of making optimal decisions, 
which would in turn translate into a low final makespan. To test the validity 
of this assumptions, each of the 697 models is run on the preference set, and 
its mean $\rho$ is reported against its corresponding validation accuracy in 
\cref{fig:CDR:scatter}. The models are colour-coded w.r.t. the number of active 
features, and a line is drawn through its Pareto front. Moreover, those 
solutions are labelled with their corresponding model ID. Moreover, the Pareto 
front over all 697 models, irrespective of active feature count, is denoted 
with triangles. Moreover, their values are reported in \cref{tbl:CDR:pareto}, 
where the best objective is given in boldface. 

\begin{table}
  \caption{Mean validation accuracy and mean expected deviation from 
    optimality, $\rho$, for all CDR models on the Pareto front from 
    \cref{fig:CDR:scatter}.}\label{tbl:CDR:pareto}
  % latex table generated in R 3.1.2 by xtable 1.7-4 package
% Sun Aug 09 19:29:20 2015
\centering
\begin{tabular}{cr@{.}llllc}\toprule
    Problem & \multicolumn{2}{c}{PREF} & \multicolumn{2}{c}{Accuracy (\%)} & 
    $\rho$ (\%) & Pareto \\
    & NrFeat & Model & Optimality & Classification & & \\ 
    \midrule \multirow{11}{*}{\jrnd{10}{10}} 
  & \textbf{3} & \textbf{524} &  91.55 & 62.57 & \textbf{12.67} & 
  $\blacktriangle$ \\ 
  & 3 & 358 &  91.82 & 62.74 & 12.90 & $\blacktriangle$ \\ 
  & 3 & 355 &  91.90 & 62.71 & 12.92 & $\blacktriangle$ \\ 
  & 2 & 69 &  91.02 & 61.41 & 12.92 &  \\ 
  & 1 & 11 &  80.77 & 55.78 & 21.63 &  \\ 
  & 1 & 13 &  85.26 & 57.17 & 22.79 &  \\ 
  & \textbf{16} & \textbf{1} &  \textbf{92.24} & \textbf{63.61} & 30.47 & 
  $\blacktriangle$ \\ 
  & 2 & 111 &  91.52 & 59.69 & 32.68 &  \\ 
  & 1 & 6 &  89.85 & 58.33 & 33.08 &  \\ 
  & 1 & 3 &  89.86 & 58.34 & 33.41 &  \\ 
  & 3 & 300 &  91.91 & 60.05 & 51.87 &  \\
    \midrule \multirow{21}{*}{\jrndn{10}{10}} 
  & \textbf{3} & \textbf{281} &  86.24 & 60.34 & \textbf{12.89} & 
  $\blacktriangle$ \\ 
  & 3 & 231 &  86.52 & 58.92 & 12.98 & $\blacktriangle$ \\ 
  & 3 & 222 &  86.69 & 58.86 & 13.23 & $\blacktriangle$ \\ 
  & 2 & 68 &  86.19 & 59.27 & 13.34 &  \\ 
  & 3 & 223 &  86.73 & 58.80 & 13.44 & $\blacktriangle$ \\ 
  & 3 & 528 &  86.84 & 59.49 & 13.61 & $\blacktriangle$ \\ 
  & 2 & 52 &  86.47 & 59.16 & 13.65 &  \\ 
  & 2 & 73 &  86.55 & 59.26 & 13.67 &  \\ 
  & 3 & 159 &  86.88 & 58.87 & 13.91 & $\blacktriangle$ \\ 
  & 3 & 263 &  86.95 & 59.20 & 14.06 & $\blacktriangle$ \\ 
  & 3 & 162 &  86.92 & 58.97 & 14.06 & $\blacktriangle$ \\ 
  & 2 & 51 &  86.65 & 58.90 & 14.06 &  \\ 
  & 3 & 147 &  87.18 & 58.88 & 14.29 & $\blacktriangle$ \\ 
  & 3 & 148 &  87.45 & 59.24 & 14.79 & $\blacktriangle$ \\ 
  & 2 & 75 &  87.11 & \textbf{60.45} & 15.30 &  \\ 
  & 3 & 418 &  87.75 & 59.57 & 16.22 & $\blacktriangle$ \\ 
  & 1 & 13 &  86.22 & 58.04 & 19.21 &  \\ 
  & 2 & 91 &  87.12 & 60.17 & 19.48 &  \\ 
  & 3 & 139 &  87.81 & 59.09 & 29.00 & $\blacktriangle$ \\ 
  & 3 & 237 &  88.07 & 59.40 & 32.69 & $\blacktriangle$ \\ 
  & \textbf{16} & \textbf{1} &  \textbf{88.86} & 60.17 & 42.88 & 
  $\blacktriangle$ \\ 
    \midrule \multirow{10}{*}{\frnd{10}{10}}   
  & \textbf{3} & \textbf{539} &  95.22 & 64.97 & \textbf{16.40} & 
  $\blacktriangle$ \\ 
  & 3 & 151 &  96.06 & 64.31 & 16.75 & $\blacktriangle$ \\ 
  & 3 & 216 &  96.28 & \textbf{71.12} & 16.78 & $\blacktriangle$ \\ 
  & 2 & 94 &  92.79 & 63.12 & 16.88 &  \\ 
  & 3 & 213 &  96.30 & 71.05 & 17.22 & $\blacktriangle$ \\ 
  & 2 & 111 &  94.16 & 65.07 & 17.73 &  \\ 
  & 2 & 51 &  95.83 & 64.21 & 17.95 &  \\ 
  & 1 & 7 &  87.59 & 61.74 & 19.05 &  \\ 
  & 1 & 6 &  92.61 & 62.91 & 19.18 &  \\ 
  & \textbf{16} & \textbf{1} &  \textbf{96.67} & 70.58 & 22.50 & 
  $\blacktriangle$ \\ 
   \hline
\end{tabular}
\end{table}

\Cref{eq:jssp:linweights} showed how to interpret the linear preference models 
by their weights $\vec{w}$. \Cref{fig:CDR:weights} depicts $\vec{w}$ for all of 
the CDR models reported in \cref{tbl:CDR:pareto}. 
The weights have been normalised for clarity purposes, such that it is scaled 
to $\norm{\vec{w}}=1$, thereby giving each feature their proportional 
contribution to the preference $I_j^{\pi}$ defined by \cref{eq:CDR}. 
These weights will now be explored further, along with testing whether models 
are statistically significant to one another, using a 
Kolmogorov-Smirnov test with $\alpha=0.05$.

For \jrnd{10}{10}  there is no statistical difference between models (2.69, 
3.355, 3.358, 3.524), w.r.t. $\rho$ and the latter three w.r.t. 
accuracy. These models are therefore equivalently `best' for the problem space.
As \cref{fig:CDR:weights} shows, \phiendTime, \phijobWrm\ and \phimacWrm\ are 
similar in value, and in the case of 3.358, then \phimacFree\ has similar 
contribution as \phiendTime\ for the other models. 
Which, as standalone models are 1.6 and 1.3, respectively, and yield 
equivalent mean $\rho$ and accuracy.
As these features often coincide in \jsp\, it is justifiable to use only 
either one, as the it contains the same information as its 
counterpart.\footnote{Note, \phiendTime$~\leq~$\phimacFree, where
  \phiendTime$~=~$\phimacFree\ when $J_j$ is the latest job on $M_a$, 
  otherwise $J_j$ is placed in a previously created slot on $M_a$.}
Most likely, the equivalence of these features is indicating that the 
schedules are rarely able to dispatch in earlier slots, i.e., 
\phiendTime$~\approx~$\phimacFree. 

In addition, (2.111, 3.300) and (16.1, 3.355) are statistically insignificant 
w.r.t. validation accuracy for \jrnd{10}{10}. However, they have considerable 
performance difference w.r.t. $\rho$ ($\Delta\rho \approx 18\%$). 
So even looking at stepwise optimality by itself is very fickle, because slight 
variations can be quite dramatic to the end result. 

The solutions on the Pareto front for \jrndn{10}{10} are quite a few models
with no (or minimal) statistical difference w.r.t. $\rho$, and 
considerably more w.r.t. validation accuracy. 
Most notably are (3.281, 2.73, 2.75, 1.13), 
(note, first one has the lowest mean $\rho$) which are all statistically 
insignificant w.r.t. validation accuracy yet none w.r.t. $\rho$, with 
difference up to $\Delta\rho=6.32\%$.

For \frnd{10}{10} almost all models are statistically different w.r.t. $\rho$, 
only exception is (1.6, 1.7).
Although, w.r.t. validation accuracy, there are a few equivalent models, 
namely, (3.151, 2.51), (2.94, 1.6) and (3.216, 3.213, 16.1), with $1.2\%$, 
$2.3\%$ and $5.75\%$ difference in mean $\rho$, respectively. 

It's interesting to inspect the full model for \frnd{10}{10}, 16.1. 
Despite having similar contributions, yet statistically significantly 
different, as all the active features as (3.213, 3.216), then the (slight) 
interference from of other features, hinders the full model from achieving a 
low $\rho$. 
Only considering \phijobOps\ and \phimacOps\ with either \phiendTime\ and 
\phimacFree, boosts performance by 5.28\% and 5.72\%, respectively. 
Thereby stressing the importance of feature selection, to steer clear of 
over-fitting. Note, unlike \jrnd{10}{10}, now \phiendTime\ differs from 
\phimacFree, indicating that there are some slots created, which could be 
better utilised.
Moreover, looking at model 2.111 for \frnd{10}{10}, which has similar 
contributions as the best model, 3.539. Then introducing a third feature, 
\phimacWrm, is the key to the success of the CDR, with a boost of $\rho$ 
performance by 1.33\%. 

Note, for both \jrnd{10}{10} and \jrndn{10}{10}, model 1.13 is on the Pareto 
front. The model corresponds to feature \phijobWrm, and in both cases has a 
weight strictly greater than zero (cf. \cref{fig:CDR:weights}). Revisiting 
\cref{eq:jssp:linweights}, we observe that this implies the learning 
model was able to discover MWR as one of the Pareto solutions, and as is 
expected, there is no statistical difference to between 1.13 and MWR.

As one can see from \cref{fig:CDR:scatter}, adding additional features to 
express the linear model boosts performance in both validation accuracy and 
expected mean for $\rho$, i.e., the Pareto fronts are cascading towards more 
desirable outcome with higher number  of active features. However, there is a 
cut-off point for such improvement, as using all features is generally 
considerably worse off due to overfitting of classifying the preference set.

\begin{figure}[t]
  \centering
  \includegraphics[width=\linewidth]{figures/{pareto.rho.vs.acc}.pdf}
  \caption{Scatter plot for validation accuracy  (\%) against its 
    corresponding mean expected $\rho$ (\%) for all 697 linear models, 
    based on either one, two, three or all $d$ combinations of features.
    Pareto fronts for each active feature count based on maximum validation 
    accuracy and minimum mean expected $\rho$ (\%), and labelled with their 
    model ID. Moreover, actual Pareto front over all models is marked with 
    triangles.} \label{fig:CDR:scatter}
\end{figure}

\begin{figure}[p]
  \centering
  \includegraphics[width=\textwidth]{figures/{j.rnd}/{pareto.10x10.phi}.pdf}\\
  \includegraphics[width=\textwidth]{figures/{j.rndn}/{pareto.10x10.phi}.pdf}\\
  \includegraphics[width=\textwidth]{figures/{f.rnd}/{pareto.10x10.phi}.pdf}
  \caption{Normalised weights for CDR models from \cref{tbl:CDR:pareto}, 
    models are grouped w.r.t. its dimensionality, $d$. Note, a triangle 
    indicates a solution on the Pareto front.}\label{fig:CDR:weights}
\end{figure}

Now, let's inspect the models corresponding to the minimum mean $\rho$ and 
highest mean validation accuracy, highlighted in \cref{tbl:CDR:pareto} and 
inspect the stepwise optimality for those models in \cref{fig:CDR:opt}, again 
using probability of randomly guessing an optimal move from \cref{fig:opt:SDR} 
(denoted RND) 
as a benchmark.
As one can see for both \jrnd{10}{10} and \jrndn{10}{10}, despite having a 
higher mean validation accuracy overall, the probabilities vary significantly. 
A lower mean $\rho$ is obtained when the validation accuracy is gradually 
increasing over time, and especially during the last phase of the 
scheduling.\footnote{It's almost illegible to notice this shift directly 
  from \cref{fig:CDR:opt}, as the difference between the two best models is 
  oscillating up to only 3\% at any given step. In fact \jrndn{10}{10} has 
  the most clear difference w.r.t. classification accuracy of indicating when 
  a minimum $\rho$ model excels at choosing the preferred move.} 
Revisiting \cref{fig:case}, this trend indicates that it's likelier for the 
resulting makespan to be considerably worse off if suboptimal moves are made at 
later stages, than at earlier stages. Therefore, it's imperative to make the 
`best' decision at the `right' moment, not just look at the overall mean 
performance. 
Hence, the measure of validation accuracy as discussed in \cref{sec:CDR:acc} 
should take into consideration the impact a suboptimal move yields on a 
step-by-step basis, e.g., weighted w.r.t. a curve such as depicted in 
\cref{fig:case}.

\begin{figure}
  \centering
  \includegraphics[width=0.8\linewidth]{figures/{stepwise.10x10.FeatSelect}.pdf}
  \caption{Probability of choosing optimal move for models corresponding to 
    highest mean validation accuracy (grey) and lowest mean deviation from 
    optimality, $\rho$, (black) compared to the baseline of probability of 
    choosing an optimal move at random (dashed).}
  \label{fig:CDR:opt}
\end{figure}

Let's revert back to the original SDRs discussed in \cref{sec:opt:sdr} and 
compare the best CDR models, a box-plot for $\rho$ is depicted in 
\cref{fig:boxplot:CDR}. Firstly, there is a statistical difference between all 
models, and  clearly the CDR model corresponding to minimum mean $\rho$ value, 
is the clear winner, and outperforms the  SDRs substantially. However, the best 
model w.r.t. maximum validation accuracy, then the CDR model shows a lacklustre 
performance. In some cases it's better off, e.g., compared to LWR, yet for 
\jsp\ it doesn't surpass the performance of MWR. This implies, the learning 
model is over-fitting the training data. Results hold for the test set. 

\begin{figure}
  \includegraphics[width=1\linewidth]{figures/{boxplotRho.FeatSelect.10x10}.pdf}
  \caption{Box-plot for deviation from optimality, $\rho$, (\%) for the best 
    CDR models (cf. \cref{tbl:CDR:pareto}) and compared against the best SDRs 
    from \cref{sec:opt:sdr}, both for training and test sets.} 
  \label{fig:boxplot:CDR}
\end{figure}

\section{Conclusions}\label{sec:con}
Current literature still hold \sdr s in high regard, 
as they are simple to implement and quite efficient. 
However, they are generally taken for granted as there is clear lack of 
investigation of \emph{how} these \dr s actually work, and what 
makes them so successful (or in some cases unsuccessful)? 
For instance, of the four SDRs this study focuses on, why does MWR outperform 
so significantly for \jsp\, yet completely fail for \fsp? 
MWR seems to be able to adapt to varying distributions of processing times, 
however, manipulating the machine ordering causes MWR to break down. 
By inspecting optimal schedules, and meticulously researching what's going on, 
every step of the way of the dispatching sequence, some light is shed where 
these SDRs vary w.r.t. the problem space at hand. 
Once these simple rules are understood, then it's feasible to extrapolate the 
knowledge gained and create new \cdr s that are likely to be 
successful. 

Creating new \dr s is by no means trivial. For \jsp\ there is 
the hidden interaction between processing times and machine ordering that's 
hard to measure.
Due to this artefact, feature selection is of paramount importance, and then it 
becomes the case of not having too many features, as they are likely to hinder 
generalisation due to over-fitting in training. However, the features need to 
be explanatory enough to maintain predictive ability. 
For this reason \cref{ch:expr:CDR} was limited to up to three active features, 
as the full feature set was clearly suboptimal w.r.t. the SDRs used as a 
benchmark. 
By using features based on the SDRs, along with some additional local features 
describing the current schedule, it was possible to `discover' the SDRs when 
given only one active feature. %Although there is not much to be gained by 
%these models, they at least are a sanity check the learning models are on the 
%right track. 
Furthermore, by adding on additional features, a boost in performance was 
gained, resulting in a \cdr\ that outperformed all of the 
SDR baseline. 

When training the learning model, it's not sufficient to only optimize w.r.t. 
highest mean validation accuracy. As \cref{sec:CDR:pareto} showed, there is a 
trade-off between making the over-all best decisions versus making the right 
decision on crucial time points in the scheduling process, as \cref{fig:case} 
clearly illustrated. It is for this reason, traditional feature selection such 
as add1 and drop1 were unsuccessful in preliminary experiments, and thus 
resorting to having to exhaustively search all feature combinations.
This also opens of the question of how should validation accuracy be measured? 
Since the model is based on learning preferences, both based on optimal versus 
suboptimal, and then varying degrees of suboptimality. As we are only looking 
at the ranks in a black and white fashion, such that the makespans need to be 
strictly greater to belong to a higher rank, then it can be argued that some 
ranks should be grouped together if their makespans are sufficiently close. 
This would simplify the training set, making it (presumably) less of 
contradictions and more appropriate for linear learning. Or simply the 
validation accuracy could be weighted w.r.t. the  difference in 
makespan.
During the dispatching process, there are some pivotal times which need to be 
especially taken care off. \Cref{fig:case} showed how making suboptimal 
decisions were more of a factor during the later stages, whereas for flow-shop 
the case was exact opposite. \todo[inline]{Could discuss new sampling 
  strategies, e.g., proportional to best/worst case, optimality, etc. -- have 
  done some experiments, but not clear what strategy is best, so only equal 
  probability reported}

Despite the abundance of information gathered by following an optimal 
trajectory, the knowledge obtained is not enough by itself. Since the learning 
model isn't perfect, it is bound to make a mistake eventually. When it does, 
the model is in uncharted  territory as there is not certainty the samples 
already collected are able to explain the current situation. For this we 
propose investigating features from suboptimal trajectories as well, since the 
future observations depend on previous predictions. 
A straight forward approach would be to inspect 
the trajectories of promising SDRs or CDRs. 
In fact, it would be worth while to try out imitation learning by 
\cite{RossB10,RossGB11}, such that the learned policy following an optimal 
trajectory is used to collect training data, and the learned model is updated. 
This can be done over several iterations, with the benefit being, that the 
states that are likely to occur in practice are investigated, and as such used 
to dissuade the model from making poor choices. Alas, this comes at great 
computational cost due to the substantial amounts of states that need to be 
optimised for their correct labelling. Making it only practical for \jsp\ of 
a considerable lower dimension. 

Although this study has been structured around the \jsp\ scheduling problem, 
it is easily extended to other types of deterministic optimisation problems 
that involve sequential decision making. 
The framework presented here collects snap-shots of the state space by 
following an optimal trajectory, and verifying the resulting optimal makespan 
from each possible state. 
From which the stepwise optimality of individual features can be inspected, 
which could for instance justify omittance in feature selection. 
\todo[color=pink]{Not done, but possible} 
Moreover, by looking at the best and worst case scenario of suboptimal 
dispatches, it is possible to pinpoint vulnerable times in the scheduling 
process. 

\section{IL.TEX}

The sampling strategy for $\Psi$ in \cite{InRu11a} was \ref{bias:equal} 
and serves as a baseline. \ref{bias:opt} motivation is to give 
samples from dispatches that are less than likely to be optimal than simply at 
random (cf. \cref{fig:opt:SDR}). On the other hand, \ref{bias:bcs} and 
\ref{bias:wcs} are more focused on sampling w.r.t. the final measure, where the 
mean $\rho$ is given \emph{one} suboptimal move, otherwise it's assumed expert 
policy is followed (cf. \cref{fig:case}). 
Lastly, \ref{bias:dbl1st} and \ref{bias:dbl2nd} are very simplified versions of 
the aforementioned strategies, depending on the problem space at hand. 
\Cref{fig:bias} depicts the stepwise bias strategies for the problem spaces in 
\cref{tbl:sim}.

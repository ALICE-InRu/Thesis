\HeaderQuote{It was much pleasanter at home, when one wasn't always growing larger and smaller, and being ordered about by mice and rabbits.}{Alice} 
%\HeaderQuote{The adventures first\dots explanations take such a dreadful time.}{The Gryphon} 

\chapter{Preference Learning of CDRs}\label{ch:prefmodels} 
\FirstSentence{L}{earning models considered in this } dissertation are based on ordinal regression in which the learning task is formulated as learning preferences. In the case of scheduling, learning which operations are preferred to others. Ordinal regression has been previously presented in \cite{Ru06:PPSN}, and given in \cref{ch:ordinal} for completeness. 

\section{Ordinal regression for \jsp}
Using the training set $\{\Phi^\pi,\mathcal{Y}^\pi\}$, given in 
\cref{eq:trdat:metadata} by following some policy $\pi$, 
let $\vphi_{o}\in\Phi^\pi$ denote the post-decision state when dispatching 
job $J_o$ corresponds to an optimal schedule being built. All post-decisions 
states corresponding to suboptimal dispatches, $J_s$, are denoted by 
$\vphi_{s}\in\Phi^\pi$. 

Let's label feature sets which were considered optimal, 
$\vec{z}_{o}=\vphi_{o}-\vphi_{s}$, and suboptimal, 
$\vec{z}_{s}=\vphi_{s}-\vphi_{o}$ by $y_o=+1$ and $y_s=-1$ respectively. 
The preference learning problem is specified by a set of preference pairs,
\begin{equation}\label{eq:Psi:jsp}
	\Psi := \bigcup_{\{\vec{x}_i\}_{i=1}^{N_{\train}}}
    \condset{
        \left\{\vec{z}_o,+1\right\},\left\{\vec{z}_s,-1\right\}}{\forall 
        (J_o,J_s) 
        \in \mathcal{O}^{(k)} \times \mathcal{S}^{(k)}}_{k=1}^K 
    \subset \Phi\times Y 
\end{equation}
where 
\begin{enumerate*}
  \item $\Phi\subset\mathcal{F}$ is the training set of $d=\NrFeatLocal$ 
  features (cf. the local features from \cref{tbl:features}) 
  \item $Y=\{-1,+1\}$ is the outcome space
  \item at each dispatch $k\in\{1,\ldots,K\}$
  \item $J_o\in\mathcal{O}^{(k)},~J_s\in \mathcal{S}^{(k)}$
  are optimal and suboptimal dispatches, respectively
\end{enumerate*}

A negative example is only created as long as $J_s$ actually 
results in a worse makespan, i.e., $C_{\max}^{\pi_\star(\vchi^s)} \gneq 
C_{\max}^{\pi_\star(\vchi^o)}$, since there can exist situations in which more 
than one operation can be considered optimal. 
Hence, $\mathcal{O}^{(k)}\cup\mathcal{S}^{(k)}=\mathcal{L}^{(k)}$, and 
$\mathcal{O}^{(k)}\cap\mathcal{S}^{(k)}=\emptyset$.
If the makespan would be unaltered, the pair is omitted from $\Psi$, since they 
give the same optimal makespan. 
This way, only features from a dispatch resulting in a suboptimal solution is 
labelled undesirable.
The approach taken here is to verify analytically, at each time step, by 
retaining the current temporal schedule as an initial state, whether it can 
indeed \emph{somehow} yield an optimal schedule by manipulating the remainder 
of the sequence, i.e., $C_{\max}^{\pi_\star(\vchi^j)}$ for all 
$J_j\in\mathcal{L}^{(k)}$. 
This also takes care of the scenario that having dispatched a job resulting in 
a different temporal makespan would have resulted in the same final makespan if 
another optimal dispatching sequence would have been chosen. 
That is to say the data generation takes into consideration when there are 
multiple optimal solutions to the same problem instance. 

Since $Y=\{+1,-1\}$, we can use logistic regression, which makes decisions 
regarding optimal dispatches and at the same time efficiently estimates a 
posteriori probabilities. 
When using linear classification model (cf. \cref{sec:ord:linpref}) for 
\cref{eq:CDR:feat},then the optimal $\vec{w}^*$ obtained from the preference 
set can be used on any new data point (i.e. partial schedule), $\vchi$, and 
their inner product is proportional to probability estimate \cref{eq:prob}. 
%Similarly for non-linear classification models. 
Hence, for each job on the job-list, $J_j\in\mathcal{L}$, let $\vphi_j$ denote 
its corresponding post-decision state. Then the job chosen to be dispatched, 
$J_{j^*}$, is the one corresponding to the highest preference estimate
from \cref{eq:CDR:feat} where $\pi(\cdot)$ is the classification model obtained 
by the preference set, $\Psi$, defined by \cref{eq:Psi:jsp}. 

\section{Selecting Preference Pairs}\label{sec:trdat:param}
Defining the size of the preference set as $l=\abs{\Psi}$, then 
\cref{eq:Psi:jsp} gives the size of the feature training set as 
$\abs{\Phi}=\frac{1}{2}l$, which is given in 
\cref{fig:size:Phi:K,tbl:size:Phi:K}.
If $l$ is too large, than sampling needs to be done in order for the ordinal 
regression to be computationally feasible.

The strategy approached in  \cref{InRu11a} was to follow a \emph{single} 
optimal job $J_j\in\mathcal{O}^{(k)}$ (chosen at random), thus creating 
$\abs{\mathcal{O}^{(k)}}\cdot\abs{\mathcal{S}^{(k)}}$ feature pairs at each 
dispatch $k$, resulting in a preference set of size,
\begin{equation}\label{eq:sizePsi_b}
l =  \sum_{i=1}^{N_{\train}} \left(2 \abs{\mathcal{O}^{(k)}_i}\cdot 
\abs{\mathcal{S}^{(k)}_i} \right)
\end{equation}
For the problem spaces considered in \cref{InRu11a}, that sort of simple 
sampling of the state space was sufficient for a favourable outcome. 
However, for a considerably harder problem spaces (cf. \cref{ch:defdifficulty}) 
and not to mention increased number of jobs and machines, preliminary 
experiments were not satisfactory. 

A brute force approach was adopted to investigate the feasibility of finding 
optimal weights $\vec{w}$ for \cref{eq:CDR:feat}. 
By applying CMA-ES (discussed thoroughly in \cref{ch:esmodels}) to directly 
minimize the mean $C_{\max}$  w.r.t. the weights $\vec{w}$, gave a considerably 
more favourable result in predicting optimal versus suboptimal dispatching 
paths. 
So the question put forth is, why was the ordinal regression not able to detect 
it?
The nature of the CMA-ES is to explore suboptimal routes until it converges to 
an optimal one. 
Implying that the previous approach of only looking into one optimal route is 
not sufficient information. 
Suggesting that the preference set should incorporate a more complete knowledge 
about \emph{all} possible preferences, i.e., make also the distinction between 
suboptimal and sub-suboptimal features, etc.  
This would require a Pareto ranking for the job-list, $\mathcal{L}$, which can 
be used to make the distinction to which feature sets are equivalent, better or 
worse, and to what degree (i.e. giving a weight to the preference)? 
By doing so, the preference set becomes much greater, which of course would 
again need to be sampled in order to be computationally feasible to learn. 

For instance \cite{Siggi05} used decision trees to `rediscover' LPT by using 
the dispatching rule to create its training data. The limitations of using 
heuristics to label the training data is that the learning algorithm will mimic 
the original heuristic (both when it works poorly and well on the problem 
instances) and does not consider the real optimum. In order to learn new 
heuristics that can outperform existing heuristics then the training data needs 
to be correctly labelled. This drawback is confronted in 
\citep{Malik08,Russell09,Siggi10} by using an optimal scheduler, computed 
off-line. 

All problem instances are correctly labelled w.r.t. their optimum makespan, 
found with analytical means.\footnote{Optimal solution were found using 
  \cite{gurobi}, a commercial software package for solving large-scale linear 
  optimisation and a state-of-the-art solver for mixed integer programming.} 
The main motivation for the data generation of $\Psi$ that will be used in 
preference learning, will now need to consider the following main aspects
\begin{enumerate}[after={{}}, leftmargin=*, label={\textbf{PREF.\arabic*}}, 
ref={{PREF.\arabic*}}]
    \item Which path(s) should be investigated to collect training instances, 
    i.e., $\Phi$. Should they be features gathered resulting in
    \label{PREF:param:tracks}
    \begin{enumerate*}[label=\textit{\roman*)},before=\unskip{: }, itemjoin={{? 
    }}, itemjoin*={{, or }},after={{? }}]
      \item optimal solutions (querying expert policy $\pi_\star$)
      \item suboptimal solutions when a DR is implemented (following a fixed 
      policy $\pi$)
      \item combination of both
    \end{enumerate*}
    \item What sort of rankings should be compared during each step?
    \label{PREF:param:ranks}
    \item What sort of stepwise sampling strategy is needed for a good
    \emph{single} time independent model?
    \label{PREF:param:sampling}
\end{enumerate}
The collection of the training set $\Phi$ in \ref{PREF:param:tracks} (which is 
described in \cref{ch:gentrdat}) is of paramount of importance, as the 
subsequent preference pairs in $\Psi$ are highly dependent on the quality of 
$\Phi$. 
Since the labelling of $\Phi$ is quite computationally intensive, its 
collection should be done parsimoniously in order to not waste valuable time 
and resources. 
On the other hand, \ref{PREF:param:ranks} and \ref{PREF:param:sampling} are 
easy to inspect, once $\Phi$ has been chosen.
The following \lcnamecref{sec:trdat:param:ranks}s will try to address these 
research questions. 

\section{Scalability of \dr s}\label{sec:pref:scalability}

In \cref{InRu11a} a separate data set was deliberately created for each 
dispatch iterations, as the initial feeling is that dispatch rules used in the 
beginning of the schedule building process may not necessarily be the same as 
in the middle or end of the schedule. As a result there are $K$ linear 
scheduling rules for solving a $n \times m$ \jsp. 
Now, if we were to create a global rule, then there would have to be one 
model for all dispatches iterations. The approach in \cref{InRu11a} was to take 
the mean weight for all stepwise linear models, i.e., 
$\bar{w}_i=\frac{1}{K}\sum_{k=1}^K w_i^{(k)}$ where $\vec{w}^{(k)}$ is 
the linear weight resulting from learning preference set $\Psi^{(k)}$ at 
dispatch $k$. 

A more sophisticated way, would be to create a \emph{new} linear model, where 
the preference set, $\Psi$, is the aggregation  of all preference pairs across 
the $K$ dispatches. 
This would amount to a substantial training set, and for $\Psi$ to 
be computationally feasible to learn, $\Psi$ has to be filtered to size 
$l_{\max}$. The default set-up will be, 
\begin{equation}\label{eq:lmax}
l_{\max} := \Bigg \{ \begin{array}{rccc} 
5 \cdot 10^5 & \quad\text{for} & 10\times 10 & \text{\JSP} \\
10^5 & \quad\text{for} & 6\times 5 & \text{\JSP}
\end{array}
\end{equation}
which is roughly 60\%-70\% amount of preferences encountered from one pass of 
sampling a \mbox{$K$-stepped} trajectory using a fixed policy $\hat{\pi}$ for 
the default $N_{\train}$ (cf. \cref{tbl:size:Psi:K}). 
Sampling is done randomly, with equal probability.

\section{Ranking strategies}\label{sec:trdat:param:ranks}
First let's address \ref{PREF:param:ranks}. 
The various ranking strategies for adding preference pairs to $\Psi$ defined by 
\cref{eq:Psi:jsp} were first reported in \cref{InRu15a}, and are the following,
\begin{description}
    \item[Basic ranking, $\Psi_b$,] i.e., all optimum rankings $r_1$ versus all 
    possible suboptimum rankings $r_i$, $i\in\{2,\ldots,n'\}$, preference pairs 
    are added -- same basic set-up introduced in \cref{InRu11a}. Note, 
    $\abs{\Psi_b}$ is defined in \cref{eq:sizePsi_b}.
    \item[Full subsequent rankings, $\Psi_f$,] i.e., all possible combinations 
    of $r_i$ and $r_{i+1}$ for $i\in\{1,\ldots,n'\}$, preference pairs are 
    added.
    \item[Partial subsequent rankings, $\Psi_p$,] i.e., sufficient set of 
    combinations of $r_i$ and $r_{i+1}$ for $i\in\{1,\ldots,n'\}$, are added to 
    the training set -- e.g. in the cases that there are more than one 
    operation with the same ranking, only one of that rank is needed to 
    compared to the subsequent rank. Note that $\Psi_p\subset \Psi_f$.
    \item[All rankings, $\Psi_a$,] denotes that all possible rankings were 
    explored, i.e.,
    $r_i$ versus $r_j$ for $i,j\in\{1,\ldots,n'\}$ and $i\neq j$, preference 
    pairs are added.
\end{description}
where $r_1>r_2>\ldots>r_{n'}$ ($n'\leq n$) are the rankings of the job-list, 
$\mathcal{L}^{(k)}$, at time step $k$.
By definition the following property holds:
\begin{equation}\label{eq:Psi:size}
    \Psi_p \subset \Psi_f \subset \Psi_b \subset \Psi_a
\end{equation}

To test the validity of different ranking strategies for 
\ref{PREF:param:ranks}, 
a training set of $N_{\train}=500$ problem instances of \jrnd{6}{5} and 
\frnd{6}{5}is collected for all trajectories described in 
\cref{sec:trdat:tracks}. 
The size of the preference set, $\abs{\Psi}$, is depicted in 
\cref{fig:size:Psi:K} for each iteration $k$. 
From which, a linear preference model is created for each preference 
set, $\Psi$. A box-plot for \fullnamerho, is presented in 
\cref{fig:boxplot:prefset}. 
From the figure it is apparent there can be a performance edge gained by 
implementing a particular trajectory strategy, yet ranking scheme seems to be 
irrelevant. Moreover, the behaviour is analogous across all other 
\Problem[6\times5]{\train} in \cref{tbl:data}.

First let's restrict  the models to \Problem{6\times5}{\train}. 
There is no statistical difference between $\Psi_f$ and $\Psi_p$ 
ranking-schemes across all disciplines,
which is expected since $\Psi_f$ is designed to contain the same preference 
information as $\Psi_f$ (cf. \cref{eq:Psi:size}). 
However, neither of the Pareto ranking-schemes outperform the original $\Psi_b$ 
set-up from \cref{InRu11a}. 
The results hold for the test set as well. 
Any statistical difference between ranking schemes were for $\Psi_a$, where it 
was considered slightly lacking than some of its counterparts. 
Since a smaller preference set is preferred, its opted to use the $\Psi_{p}$ 
ranking scheme henceforth as the default set-up for \ref{PREF:param:ranks}. 

Moving on to higher dimension, results for \jrnd{10}{10} were similar to 
\Problem[6\times5]{\train}. Only exception begin that ranking schemes 
showed difference in performance when using \PhiSet{OPT}, where \PsiSet[OPT]{p} 
come on top. Strengthening our previous choice of \PsiSet{p} as 
standard ranking scheme.


\begin{figure}[p]
  \includegraphics[width=\textwidth]{{prefdat.size.6x5}.pdf}
  \vspace*{-30pt}
  \caption[Size of preference set, $\abs{\Psi}$]{Size of 
  \Problem[6\times5]{\train} preference set, $l=\abs{\Psi}$, for 
  different trajectory strategies and ranking schemes (where 
  $N_{\train}=500$) }
  \label{fig:size:Psi:K}
\end{figure}

\begin{figure}[p]
  \includegraphics[width=\textwidth]{{prefdat.boxplot.6x5}.pdf}
  \vspace*{-30pt}
  \caption{Box-plot for various $\Phi$ and $\Psi$ set-up using
    \Problem[6\times5]{\train}. The trajectories 
    the models are based on are depicted in white on the far right.}
  \label{fig:boxplot:prefset}
\end{figure}

\section{Trajectory strategies}\label{sec:trdat:param:tracks:passive}

We'd like to inspect which trajectory is the best to use for $\Psi$. 
\Cref{InRu15a} only considered \jrnd{6}{5} and \jrndn{6}{5}, 
however, results for \Problem[6\times5]{\train} and 
\jrnd{10}{10} are currently available.\footnote{Additional problem spaces can 
    be found in \shiny: Preference Models $>$ Trajectories \& ranks.}
Models from \cref{fig:boxplot:prefset} are limited to the ones corresponding to 
$\Psi_p$. 
\Cref{fig:size:Psi.p:K} jointly illustrates the size of the preference set 
used, i.e., $\abs{\Psi_p}$ from \cref{fig:size:Psi:K}.
\Cref{tbl:size:Psi:K} reports the total amount of preferences for all $K$ 
dispatches.

\Cref{tbl:param:tracks} reports the relative ordering of trajectories, 
ordered w.r.t. their mean \namerho, and their size of preference set, i.e., 
$\abs{\Psi_p}$.
Models that are statistically better are denoted by `$\succ$' otherwise 
considered equivalent. 

For most problem spaces \PsiSet{LPT}{p} was the worst trajectory to pursue. 
Looking back at \cref{fig:size:Phi:K}, then even though \PhiSet{LPT} was not 
the trajectory with the least features, the amount of equivalent features 
w.r.t. $C_{\max}$ are far too many to make a meaningful preference set out of 
it. It's only for \jrndn{6}{5} that there is another trajectory with fewer 
preferences, namely \PsiSet{LWR}{p} (cf. \cref{fig:size:Psi:K}), and in that 
case LWR is the worst model.
Model that come on top, are those that have a varied $\Psi$. However, 
aggregating features from all trajectories is not a good idea, as the 
preference set then becomes too varied for a satisfactory result. 

\begin{table}[p]\centering
\caption[Total number of preferences in $\Psi_p$ for all $K$ steps]{Total 
number of preferences in $l=\abs{\Psi_p}$ for all $K$ steps. Note `--' denotes 
not available.}
\label{tbl:size:Psi:K}
{\setlength{\tabcolsep}{2pt} \scriptsize
\begin{tabular}{lcccccccccccc}
  \toprule 
  Track 
&\multicolumn{9}{c}{\Problem[6\times5]{\text{train}}, $N_{\text{train}}=500$} 
&\multicolumn{3}{c}{\Problem[10\times10]{\text{train}}, $N_{\text{train}}=300$} 
\\
  & $j.rnd$ & $j.rndn$ & $j.rnd,J_1$& $j.rnd,M_1$ & $f.rnd$ & $f.rndn$ & 
  $f.jc$ & $f.mc$ & $f.mxc$ & $j.rnd$ & $j.rndn$ & $f.rnd$ \\ 
  \midrule
  SPT & 73926 & 68410 & 74416 & 65150 & 79388 & 70808 & 68956 & 89788 & 92036 
  & 285912 & -- & -- \\ 
  LPT & 43456 & 58540 & 28498 & 34136 & 36162 & 54684 & 11548 & 23260 & 17308 
  & 151444 & -- & -- \\ 
  LWR & 46580 & 46306 & 32326 & 41554 & 64226 & 68628 & 69124 & 40150 & 40110 
  & 163546 & -- & -- \\ 
  MWR & 83756 & 102092 & 53246 & 62056 & 87376 & 111708 & 106226 & 65882 & 
  64692 & 370104 & -- & -- \\ 
  RND & 72824 & 80358 & 52210 & 61670 & 77148 & 77080 & 64550 & 55288 & 55398 
  & 313346 & -- & -- \\ 
  OPT & 100910 & 111736 & 79404 & 90948 & 95388 & 93036 & 81306 & 79836 & 78440 
  & 453662 & 470522 & 299952 \\ 
  $\minRho$ & 93006 & 111068 & 64050 & 89504 & 77142 & 63120 & 45404 & 36608 & 
  74556 & 427032 & -- & -- \\ 
  $\minCmax$ & 108390 & 111346 & 73168 & 95920 & 83058 & 61992 & 47412 & 35484 
  & 36052 & 432650 & -- & -- \\ \midrule
  ALL & 622848 & 689856 & 457318 & 540938 & 599888 & 601056 & 494526 & 426296 & 
  458592 & 2595758 & 470522 & 299952 \\
  \bottomrule
\end{tabular}}
\end{table}

\begin{table}
  \caption{Relative ordering w.r.t. mean $\rho$ for trajectories in 
    \cref{sec:trdat:tracks}} \label{tbl:param:tracks}
  \begin{tabular}{lc}
    \toprule
    Problem & Ordering of trajectories \\ \midrule
    \jrnd{6}{5} & $\minCmax$ $\succ$ $\minRho$ $\succ$ ALL $\equiv$ SPT 
    $\equiv$ MWR $\equiv$ RND $\succ$ OPT $\succ$ LWR $\succ$ LPT \\
    \jrndn{6}{5} & $\minCmax$ $\equiv$ $\minRho$ $\equiv$ MWR $\equiv$ RND 
    $\equiv$ ALL $\equiv$ SPT $\succ$ LPT $\succ$ LWR $\succ$ OPT \\
    \jrndJ{6}{5} & $\minCmax$ $\succ$ $\minRho$ $\succ$ SPT $\succ$ ALL 
    $\equiv$ RND $\succ$ OPT $\equiv$ LWR $\equiv$ MWR $\succ$ LPT \\
    \jrndM{6}{5} & $\minCmax$ $\equiv$ $\minRho$ $\succ$ SPT $\equiv$ ALL 
    $\equiv$ RND $\succ$ LWR $\succ$ OPT $\succ$ MWR $\succ$ LPT\\
    \frnd{6}{5} & SPT $\equiv$ $\minCmax$ $\equiv$ OPT $\equiv$ $\minRho$ 
    $\succ$ ALL $\succ$ LWR $\succ$ MWR $\equiv$ RND $\equiv$ LPT \\
    \frndn{6}{5} & $\minRho$ $\equiv$ $\minCmax$ $\equiv$ RND $\equiv$ LWR 
    $\equiv$ ALL $\succ$ SPT $\equiv$ LPT $\equiv$ OPT $\succ$ MWR\\
    \fjc{6}{5} & OPT $\succ$ SPT $\equiv$ LWR $\succ$ RND $\equiv$ MWR $\succ$ 
    LPT $\equiv$ $\minCmax$ $\equiv$ $\minRho$ $\succ$ ALL\\
    \fmc{6}{5} &$\minRho$ $\succ$ LWR $\succ$ $\minCmax$ $\equiv$ MWR $\succ$ 
    OPT $\equiv$ LPT $\equiv$ RND $\equiv$ SPT $\succ$ ALL \\
    \fmxc{6}{5} & RND $\equiv$ OPT $\equiv$ $\minRho$ 
    $\equiv$ SPT $\equiv$ ALL $\equiv$ MWR $\succ$ LWR $\succ$ $\minCmax$ 
    $\succ$ LPT\\
    \jrnd{10}{10} & MWR $\succ$ RND $\succ$ SPT $\equiv$ OPT $\succ$ ALL 
    $\equiv$ LPT $\equiv$ LWR \\
    \bottomrule
  \end{tabular}
\end{table}


\begin{figure}[t]
  \includegraphics[width=\textwidth]{{prefdat.p.size.6x5}.pdf}
  \vspace*{-30pt}
  \caption[Size of preference set, $\abs{\Psi_p}$]{Size of 
    \Problem[6\times5]{\train} preference set, $l=\abs{\Psi_p}$, for 
    different trajectory strategies}
  \label{fig:size:Psi.p:K}
\end{figure}

Learning preference pairs from a good scheduling policies, such as 
$\Phi^{\minCmax},~\Phi^{\minRho}$ and \PhiSet{MWR}, gave considerably more 
favourable results than tracking optimal paths, save for \fjc{6}{5} where the 
ordering is reversed. Generally, suboptimal routes are preferred. 
However, even though LWR is a better policy than MWR for \FSP, then 
\PhiSet{LWR} is a worse candidate than e.g. \PhiSet{MWR}, but as discussed 
before, it's due to the lack of varied dispatches for the trajectory.

It is particularly interesting there is statistical difference between 
\PhiSet{OPT} and \PhiSet{RND}, where the latter had improved performance for 
all \JSP\ problem spaces. In those cases, tracking optimal dispatches gives 
worse performance as pursuing completely random dispatches. 
This indicates that exploring only expert policy can result in a 
training set which the learning algorithm is inept to determine good dispatches 
in the circumstances when newly encountered features have diverged from the 
learned feature set labelled to optimum solutions. 

Generally, adding suboptimal trajectories with the expert policy, i.e., 
\PhiSet{ALL}, gives the learning algorithm a greater variety of preference 
pairs for getting out of local minima. However, for some problem spaces, e.g., 
\frnd{6}{5} and \fmc{6}{5} then additional suboptimal solutions that are too 
diverse yield a worse outcome than \PhiSet{OPT} would achieve on its own.

%   Problem Dimension   SPT   LPT   LWR   MWR   RND ES.rho ES.Cmax
%     j.rnd       6x5 Track Track Track Track Track    SDR     SDR
%    j.rndn       6x5 Track Track Track Track Track   SAME    SAME
%  j.rnd,J1       6x5 Track Track Track   SDR Track    SDR     SDR
%  j.rnd,M1       6x5 Track  SAME Track   SDR Track    SDR     SDR
%     f.rnd       6x5 Track Track Track Track Track    SDR     SDR
%    f.rndn       6x5 Track Track Track Track Track    SDR     SDR
%      f.jc       6x5 Track Track Track Track Track    SDR     SDR
%      f.mc       6x5 Track Track   SDR Track Track    SDR     SDR
%     f.mxc       6x5 Track Track   SDR Track Track    SDR     SDR
%     j.rnd     10x10 Track Track Track  SAME Track    SDR     SDR

\pagebreak
Comparing $\Psi^{\angles{\text{DR}}}$ to its corresponding DR used to 
guide its collection, then usually the preference model outperformed the DR it 
was trying to mimic. The exceptions being 
\begin{enumerate*}
  \item MWR for \jrndJ{6}{5} and \jrndM{6}{5} (and \jrnd{10}{10} was 
  statistically insignificant)
  \item LWR for \fmc{6}{5} and \fmxc{6}{5}
  \item LPT was statistically insignificant for \jrndM{6}{5}
  \item $\minCmax$ and $\minRho$ for all problem spaces, save for \jrndn{6}{5} 
  which was statistically insignificant
\end{enumerate*}
Revisiting \cref{fig:diff:case:SDR}, then when $\Psi^{\pi}$ 
succeeds its original policy $\pi$, it implies the learning model was able to 
steer the learned policy towards $\zeta_{\min}^{\pi}$. 
In fact, its improvement is proportional to its spread\footnote{Consult \shiny:
    Optimality $>$ Best and worst case scenario.}  
from $\zeta_{\mu}^{\pi}$ to $\zeta_{\min}^{\pi}$ or $\zeta_{\max}^{\pi}$.
Therefore, a good preference set based on $\Phi^\pi$ not only has to have a low 
$\zeta_{\mu}^\pi$ to mimic, but also the policy $\pi$ needs to be sufficiently 
different from $\zeta_{\min}^{\pi}$ and $\zeta_{\max}^{\pi}$ for adequate 
learning. That is why $\Phi^{\CMAES}$ strategies 
were not good enough for preference learning, as their 
$\zeta^\pi_{\any}$ spread was the lowest compared to the other fixed DRs.

The rational for using the $\Phi^{\CMAES}$ strategies
was mostly due to the fact a linear classifier is creating the training data 
(using the weights found via CMA-ES optimisation in \cref{eq:cma:objfun}), 
hence the training data created should be linearly separable, which in turn 
should boost the training accuracy for a linear classification learning model. 
However, these strategies is not outperforming the original DR used in guiding 
the training data collection. 

As the experimental results showed, that unlike 
\citep{Siggi10,Malik08,Russell09}, learning only on optimal training 
data was not fruitful. However, inspired by the original work by 
\cite{Siggi05}, having DR guide the generation of training data (except 
correctly labelling with analytic means) gives meaningful preference pairs 
which the learning algorithm could learn. 

\section{Feature Selection}\label{sec:pref:featselect}

We know from \cref{ch:esmodels} there exists linear weights $\vec{w}$ for 
\cref{eq:CDR:feat} found with evolutionary optimisation that achieve a lower 
\namerho, than any of the aforementioned preference models have been able to 
learn. This goes to show that the $d=\NrFeatLocal$ features are `enough' -- 
meaning there is not a need for defining new ones just yet. 
However, the optimal weights for \cref{eq:cma:objfun} were quite erratic (cf. 
\cref{fig:cma:wei}). 
Perhaps the features from $\Phi^{\CMAES}$ are 
contradictory, and therefore not suitable for preference learning. 
Furthermore, the SDRs we've inspected so-far are based on two job-attributes 
from \cref{tbl:features}, namely
\begin{enumerate*}[after={{,}}]
  \item \phiproc\ for SPT and LPT 
  \item \phijobWrm\ for LWR and MWR 
\end{enumerate*}
by choosing the lowest value for SPT and LWR, and highest value for LPT and 
MWR, i.e., the extremal values for those attributes. 
These SDRs achieve a remarkably low $\rho$, so perhaps not that many additional 
features are needed to achieve a competitive result.

For this study we will consider all combinations of 
feature attributes using either one, two, three or all $d=\NrFeatLocal$ of 
them, for a total of
\begin{equation}
{d \choose 1}+{d \choose 2}+{d \choose 3}+{d \choose d} = 697
\end{equation}
The reason for such a limiting number of active features, are due to the fact 
we want to keep the models simple enough for improved model interpretability.

For each feature combination, a linear preference model is created, where 
$\Psi_p$ is limited to the predetermined feature combination. 
This was done for all \Problem[10\times10]{\train} in \cref{tbl:data}, 
each consisting of $N_{\train}=300$ problem instances.
Moreover, in order to report the validation accuracy, 20\% 
(i.e. $N_{\text{val}}=60$) of the training set was set aside for 
reporting the accuracy.

\subsection{Validation accuracy}\label{sec:CDR:acc}
As the preference set $\Psi_p$ has both preference pairs belonging to optimal
ranking, and subsequent rankings, it is not of primary importance to classify
\emph{all} rankings correctly, just the optimal ones. Therefore, instead of
reporting the validation accuracy based on the classification problem of the
correctly labelling the entire problem set $\Psi_p$, it's opted that the 
validation accuracy is obtained using \cref{eq:tracc:opt}, namely the 
probability of choosing an optimal decision given the resulting linear 
weights.\footnote{Due to the superabundant number of models then calculating 
    the \emph{preferable} $\xi_{\pi}$ from \cref{eq:tracc:track} is not viable.}
However, in this context, the mean throughout the dispatching process is
reported, i.e., $\frac{1}{K}\sum_{k=k'}^K \xi^{\star}_{\pi}(k')$. 
\Cref{fig:stepwise_vs_classification} shows the difference between
the two measures of reporting validation accuracy. 

Validation accuracy based on $\xi^{\star}_{\pi}$ only takes into consideration 
the likelihood of choosing the optimal move at each time step. However, the 
classification accuracy is also trying to correctly distinguish all subsequent 
rankings in addition of choosing the optimal move, as expected that measure is 
considerably lower. 

\begin{figure}[p]
  \centering
  \includegraphics[width=\linewidth]{figures/{training.accuracy}.pdf}
  \caption{Various methods of reporting validation accuracy for preference 
    learning}
  \label{fig:stepwise_vs_classification}
\end{figure}

\subsection{Pareto front}\label{sec:CDR:pareto}
When training the learning model one wants to keep the validation accuracy 
high, as that would imply a higher likelihood of making optimal decisions, 
which would in turn translate into a low final makespan. To test the validity 
of this assumptions, each of the 697 models is run on the preference set, and 
its mean $\rho$ is reported against its corresponding validation accuracy in 
\cref{fig:CDR:scatter}. The models are colour-coded w.r.t. the number of active 
features, and a line is drawn through its Pareto front. Moreover, those 
solutions are labelled with their corresponding model ID. Moreover, the Pareto 
front over all 697 models, irrespective of active feature count, is denoted 
with triangles. Moreover, their values are reported in \cref{tbl:CDR:pareto}, 
where the best objective is given in boldface. 

\begin{table}[p]
  \caption{Mean validation accuracy and mean expected deviation from 
    optimality, $\rho$, for all CDR models on the Pareto front from 
    \cref{fig:CDR:scatter}.}\label{tbl:CDR:pareto}
  % latex table generated in R 3.1.2 by xtable 1.7-4 package
% Sun Aug 09 19:29:20 2015
\centering
\begin{tabular}{cr@{.}llllc}\toprule
    Problem & \multicolumn{2}{c}{PREF} & \multicolumn{2}{c}{Accuracy (\%)} & 
    $\rho$ (\%) & Pareto \\
    & NrFeat & Model & Optimality & Classification & & \\ 
    \midrule \multirow{11}{*}{\jrnd{10}{10}} 
  & \textbf{3} & \textbf{524} &  91.55 & 62.57 & \textbf{12.67} & 
  $\blacktriangle$ \\ 
  & 3 & 358 &  91.82 & 62.74 & 12.90 & $\blacktriangle$ \\ 
  & 3 & 355 &  91.90 & 62.71 & 12.92 & $\blacktriangle$ \\ 
  & 2 & 69 &  91.02 & 61.41 & 12.92 &  \\ 
  & 1 & 11 &  80.77 & 55.78 & 21.63 &  \\ 
  & 1 & 13 &  85.26 & 57.17 & 22.79 &  \\ 
  & \textbf{16} & \textbf{1} &  \textbf{92.24} & \textbf{63.61} & 30.47 & 
  $\blacktriangle$ \\ 
  & 2 & 111 &  91.52 & 59.69 & 32.68 &  \\ 
  & 1 & 6 &  89.85 & 58.33 & 33.08 &  \\ 
  & 1 & 3 &  89.86 & 58.34 & 33.41 &  \\ 
  & 3 & 300 &  91.91 & 60.05 & 51.87 &  \\
    \midrule \multirow{21}{*}{\jrndn{10}{10}} 
  & \textbf{3} & \textbf{281} &  86.24 & 60.34 & \textbf{12.89} & 
  $\blacktriangle$ \\ 
  & 3 & 231 &  86.52 & 58.92 & 12.98 & $\blacktriangle$ \\ 
  & 3 & 222 &  86.69 & 58.86 & 13.23 & $\blacktriangle$ \\ 
  & 2 & 68 &  86.19 & 59.27 & 13.34 &  \\ 
  & 3 & 223 &  86.73 & 58.80 & 13.44 & $\blacktriangle$ \\ 
  & 3 & 528 &  86.84 & 59.49 & 13.61 & $\blacktriangle$ \\ 
  & 2 & 52 &  86.47 & 59.16 & 13.65 &  \\ 
  & 2 & 73 &  86.55 & 59.26 & 13.67 &  \\ 
  & 3 & 159 &  86.88 & 58.87 & 13.91 & $\blacktriangle$ \\ 
  & 3 & 263 &  86.95 & 59.20 & 14.06 & $\blacktriangle$ \\ 
  & 3 & 162 &  86.92 & 58.97 & 14.06 & $\blacktriangle$ \\ 
  & 2 & 51 &  86.65 & 58.90 & 14.06 &  \\ 
  & 3 & 147 &  87.18 & 58.88 & 14.29 & $\blacktriangle$ \\ 
  & 3 & 148 &  87.45 & 59.24 & 14.79 & $\blacktriangle$ \\ 
  & 2 & 75 &  87.11 & \textbf{60.45} & 15.30 &  \\ 
  & 3 & 418 &  87.75 & 59.57 & 16.22 & $\blacktriangle$ \\ 
  & 1 & 13 &  86.22 & 58.04 & 19.21 &  \\ 
  & 2 & 91 &  87.12 & 60.17 & 19.48 &  \\ 
  & 3 & 139 &  87.81 & 59.09 & 29.00 & $\blacktriangle$ \\ 
  & 3 & 237 &  88.07 & 59.40 & 32.69 & $\blacktriangle$ \\ 
  & \textbf{16} & \textbf{1} &  \textbf{88.86} & 60.17 & 42.88 & 
  $\blacktriangle$ \\ 
    \midrule \multirow{10}{*}{\frnd{10}{10}}   
  & \textbf{3} & \textbf{539} &  95.22 & 64.97 & \textbf{16.40} & 
  $\blacktriangle$ \\ 
  & 3 & 151 &  96.06 & 64.31 & 16.75 & $\blacktriangle$ \\ 
  & 3 & 216 &  96.28 & \textbf{71.12} & 16.78 & $\blacktriangle$ \\ 
  & 2 & 94 &  92.79 & 63.12 & 16.88 &  \\ 
  & 3 & 213 &  96.30 & 71.05 & 17.22 & $\blacktriangle$ \\ 
  & 2 & 111 &  94.16 & 65.07 & 17.73 &  \\ 
  & 2 & 51 &  95.83 & 64.21 & 17.95 &  \\ 
  & 1 & 7 &  87.59 & 61.74 & 19.05 &  \\ 
  & 1 & 6 &  92.61 & 62.91 & 19.18 &  \\ 
  & \textbf{16} & \textbf{1} &  \textbf{96.67} & 70.58 & 22.50 & 
  $\blacktriangle$ \\ 
   \hline
\end{tabular}
\end{table}

\Cref{fig:CDR:weights} depicts $\vec{w}$ for all of the learned CDR models 
reported in \cref{tbl:CDR:pareto}. 
The weights have been normalised for clarity purposes, such that it is scaled 
to $\norm{\vec{w}}=1$, thereby giving each feature their proportional 
contribution to the preference $I_j^{\pi}$ defined by \cref{eq:CDR}. 
These weights will now be explored further, along with testing whether models 
are statistically significant to one another, using a 
Kolmogorov-Smirnov test with $\alpha=0.05$.

For \jrnd{10}{10}  there is no statistical difference between models (2.69, 
3.355, 3.358, 3.524), w.r.t. $\rho$ and the latter three w.r.t. 
accuracy. These models are therefore equivalently `best' for the problem space.
As \cref{fig:CDR:weights} shows, \phiendTime, \phijobWrm\ and \phimacWrm\ are 
similar in value, and in the case of 3.358, then \phimacFree\ has similar 
contribution as \phiendTime\ for the other models. 
Which, as standalone models are 1.6 and 1.3, respectively, and yield 
equivalent mean $\rho$ and accuracy.
As these features often coincide in \jsp\, it is justifiable to use only 
either one, as the it contains the same information as its 
counterpart.\footnote{Note, \phiendTime$~\leq~$\phimacFree, where
  \phiendTime$~=~$\phimacFree\ when $J_j$ is the latest job on $M_a$, 
  otherwise $J_j$ is placed in a previously created slot on $M_a$.}
Most likely, the equivalence of these features is indicating that the 
schedules are rarely able to dispatch in earlier slots, i.e., 
\phiendTime$~\approx~$\phimacFree. 

In addition, (2.111, 3.300) and (16.1, 3.355) are statistically insignificant 
w.r.t. validation accuracy for \jrnd{10}{10}. However, they have considerable 
performance difference w.r.t. $\rho$ ($\Delta\rho \approx 18\%$). 
So even looking at stepwise optimality, $\xi_{\pi}^{\star}$ by itself is very 
fickle, because slight variations can be quite dramatic to the end result. 

The solutions on the Pareto front for \jrndn{10}{10} are quite a few models
with no (or minimal) statistical difference w.r.t. $\rho$, and 
considerably more w.r.t. validation accuracy. 
Most notably are (3.281, 2.73, 2.75, 1.13)
(note, first one has the lowest mean $\rho$), which are all statistically 
insignificant w.r.t. validation accuracy yet none w.r.t. $\rho$, with 
difference up to $\Delta\rho=6.32\%$.

For \frnd{10}{10} almost all models are statistically different w.r.t. $\rho$, 
only exception is (1.6, 1.7).
Although, w.r.t. validation accuracy, there are a few equivalent models, 
namely, (3.151, 2.51), (2.94, 1.6) and (3.216, 3.213, 16.1), with $1.2\%$, 
$2.3\%$ and $5.75\%$ difference in mean $\rho$, respectively. 

It's interesting to inspect the full model for \frnd{10}{10}, 16.1. 
Despite having similar contributions, yet statistically significantly 
different, as all the active features as (3.213, 3.216), then the (slight) 
interference from of other features, hinders the full model from achieving a 
low $\rho$. 
Only considering \phijobOps\ and \phimacOps\ with either \phiendTime\ and 
\phimacFree, boosts performance by 5.28\% and 5.72\%, respectively. 
Thereby stressing the importance of feature selection, to steer clear of 
over-fitting. Note, unlike \jrnd{10}{10}, now \phiendTime\ differs from 
\phimacFree, indicating that there are some slots created, which could be 
better utilised.
Moreover, looking at model 2.111 for \frnd{10}{10}, which has similar 
contributions as the best model, 3.539. Then introducing a third feature, 
\phimacWrm, is the key to the success of the CDR, with a boost of $\rho$ 
performance by 1.33\%. 

Note, for both \jrnd{10}{10} and \jrndn{10}{10}, model 1.13 is on the Pareto 
front. The model corresponds to feature \phijobWrm, and in both cases has a 
weight strictly greater than zero (cf. \cref{fig:CDR:weights}). Revisiting 
\cref{eq:CDR:SDR}, we observe that this implies the learning 
model was able to discover MWR as one of the Pareto solutions, and as is 
expected, there is no statistical difference to between 1.13 and MWR.

As one can see from \cref{fig:CDR:scatter}, adding additional features to 
express the linear model boosts performance in both validation accuracy and 
expected mean for $\rho$, i.e., the Pareto fronts are cascading towards more 
desirable outcome with higher number  of active features. However, there is a 
cut-off point for such improvement, as using all features is generally 
considerably worse off due to overfitting of classifying the preference set.

\begin{figure}[t]
  \centering
  \includegraphics[width=\linewidth]{figures/{pareto.rho.vs.acc}.pdf}
  \caption{Scatter plot for validation accuracy  (\%) against its 
    corresponding mean expected $\rho$ (\%) for all 697 linear models, 
    based on either one, two, three or all $d$ combinations of features.
    Pareto fronts for each active feature count based on maximum validation 
    accuracy and minimum mean expected $\rho$ (\%), and labelled with their 
    model ID. Moreover, actual Pareto front over all models is marked with 
    triangles.} \label{fig:CDR:scatter}
\end{figure}

\begin{figure}[p]
  \centering
  \includegraphics[width=\textwidth]{figures/{j.rnd}/{pareto.10x10.phi}.pdf}\\
  \includegraphics[width=\textwidth]{figures/{j.rndn}/{pareto.10x10.phi}.pdf}\\
  \includegraphics[width=\textwidth]{figures/{f.rnd}/{pareto.10x10.phi}.pdf}
  \caption{Normalised weights for CDR models from \cref{tbl:CDR:pareto}, 
    models are grouped w.r.t. its dimensionality, $d$. Note, a triangle 
    indicates a solution on the Pareto front.}\label{fig:CDR:weights}
\end{figure}

Now, let's inspect the models corresponding to the minimum mean $\rho$ and 
highest mean validation accuracy, highlighted in \cref{tbl:CDR:pareto} and 
inspect the stepwise optimality for those models in \cref{fig:CDR:opt}, again 
using probability of randomly guessing an optimal move (i.e. 
$\xi^{\star}_{\RND}$ from \cref{fig:diff:opt:rnd}) as a benchmark.
As one can see for both \jrnd{10}{10} and \jrndn{10}{10}, despite having a 
higher mean validation accuracy overall, the probabilities vary significantly. 
A lower mean $\rho$ is obtained when the validation accuracy is gradually 
increasing over time, and especially during the last phase of the 
scheduling.\footnote{It's almost illegible to notice this shift directly 
  from \cref{fig:CDR:opt}, as the difference between the two best models is 
  oscillating up to only 3\% at any given step. In fact \jrndn{10}{10} has 
  the most clear difference w.r.t. classification accuracy of indicating when 
  a minimum $\rho$ model excels at choosing the preferred move.} 
Revisiting \cref{fig:case}, this trend indicates that it's likelier for the 
resulting makespan to be considerably worse off if suboptimal moves are made at 
later stages, than at earlier stages. Therefore, it's imperative to make the 
`best' decision at the `right' moment, not just look at the overall mean 
performance. 
Hence, the measure of validation accuracy as discussed in \cref{sec:CDR:acc} 
should take into consideration the impact a suboptimal move yields on a 
step-by-step basis, e.g., \cref{eq:tracc:opt} is weighted w.r.t. a curve such 
as \cref{eq:bwc:opt}.

\begin{figure}
  \centering
  \includegraphics[width=0.8\linewidth]{figures/{stepwise.10x10.FeatSelect}.pdf}
  \caption{Probability of choosing optimal move for models corresponding to 
    highest mean validation accuracy (grey) and lowest mean deviation from 
    optimality, $\rho$, (black) compared to the baseline of probability of 
    choosing an optimal move at random (dashed).}
  \label{fig:CDR:opt}
\end{figure}

Let's revert back to the original SDRs discussed in \cref{sec:opt:sdr} and 
compare the best CDR models, a box-plot for $\rho$ is depicted in 
\cref{fig:boxplot:CDR}. Firstly, there is a statistical difference between all 
models, and  clearly the CDR model corresponding to minimum mean $\rho$ value, 
is the clear winner, and outperforms the  SDRs substantially. However, the best 
model w.r.t. maximum validation accuracy, then the CDR model shows a lacklustre 
performance. In some cases it's better off, e.g., compared to LWR, yet for 
\jsp\ it doesn't surpass the performance of MWR. This implies, the learning 
model is over-fitting the training data. Results hold for the test set. 

\begin{figure}
  \includegraphics[width=1\linewidth]{figures/{boxplotRho.FeatSelect.10x10}.pdf}
  \caption{Box-plot for deviation from optimality, $\rho$, (\%) for the best 
    CDR models (cf. \cref{tbl:CDR:pareto}) and compared against the best SDRs 
    from \cref{sec:opt:sdr}, both for training and test sets.} 
  \label{fig:boxplot:CDR}
\end{figure}

\subsection{Conclusions}

When training the learning model, it's not sufficient to only optimize w.r.t. 
highest mean validation accuracy. As \cref{sec:CDR:pareto} showed, there is a 
trade-off between making the over-all best decisions versus making the right 
decision on crucial time points in the scheduling process, as \cref{fig:case} 
clearly illustrated. It is for this reason, traditional feature selection such 
as add1 and drop1 were unsuccessful in preliminary experiments, and thus 
resorting to having to exhaustively search all feature combinations.
This also opens of the question of how should validation accuracy be measured? 
Since the model is based on learning preferences, both based on optimal versus 
suboptimal, and then varying degrees of suboptimality. As we are only looking 
at the ranks in a black and white fashion, such that the makespans need to be 
strictly greater to belong to a higher rank, then it can be argued that some 
ranks should be grouped together if their makespans are sufficiently close. 
This would simplify the training set, making it (presumably) less of 
contradictions and more appropriate for linear learning. Or simply the 
validation accuracy could be weighted w.r.t. the  difference in 
makespan.
During the dispatching process, there are some pivotal times which need to be 
especially taken care off. \Cref{fig:case} showed how making suboptimal 
decisions were more of a factor during the later stages, whereas for flow-shop 
the case was exact opposite. \todo[inline]{Could discuss new sampling 
  strategies, e.g., proportional to best/worst case, optimality, etc. -- have 
  done some experiments, but not clear what strategy is best, so only equal 
  probability reported}

Despite the abundance of information gathered by following an optimal 
trajectory, the knowledge obtained is not enough by itself. Since the learning 
model isn't perfect, it is bound to make a mistake eventually. When it does, 
the model is in uncharted  territory as there is not certainty the samples 
already collected are able to explain the current situation. For this we 
propose investigating features from suboptimal trajectories as well, since the 
future observations depend on previous predictions. 
A straight forward approach would be to inspect 
the trajectories of promising SDRs or CDRs. 
In fact, it would be worth while to try out imitation learning by 
\cite{RossB10,RossGB11}, such that the learned policy following an optimal 
trajectory is used to collect training data, and the learned model is updated. 
This can be done over several iterations, with the benefit being, that the 
states that are likely to occur in practice are investigated, and as such used 
to dissuade the model from making poor choices. Alas, this comes at great 
computational cost due to the substantial amounts of states that need to be 
optimised for their correct labelling. Making it only practical for \jsp\ of 
a considerable lower dimension. 

Although this study has been structured around the \jsp\ scheduling problem, 
it is easily extended to other types of deterministic optimisation problems 
that involve sequential decision making. 
The framework presented here collects snap-shots of the state space by 
following an optimal trajectory, and verifying the resulting optimal makespan 
from each possible state. 
From which the stepwise optimality of individual features can be inspected, 
which could for instance justify omittance in feature selection. 
\todo[color=pink]{Not done, but possible} 
Moreover, by looking at the best and worst case scenario of suboptimal 
dispatches, it is possible to pinpoint vulnerable times in the scheduling 
process. 


\HeaderQuote{It was much pleasanter at home, when one wasn't always growing larger and smaller, and being ordered about by mice and rabbits.}{Alice} 
%\HeaderQuote{The adventures first\dots explanations take such a dreadful time.}{The Gryphon} 

\chapter{Preference Learning of CDRs}\label{ch:prefmodels} 
\FirstSentence{L}{earning models considered in this } dissertation are based on ordinal regression in which the learning task is formulated as learning preferences. In the case of scheduling, learning which operations are preferred to others. Ordinal regression has been previously presented in \cite{Ru06:PPSN}, and given in \cref{ch:ordinal} for completeness. 

\section{Ordinal regression for \jsp}
Let $\vphi_{o}\in\mathcal{F}$ denote the post-decision state when dispatching $J_o$ corresponds to an optimal schedule being built. All post-decisions states corresponding to suboptimal dispatches, $J_s$, are denoted by $\vphi_{s}\in\mathcal{F}$. One could label which feature sets were considered optimal, $\vec{z}_{o}=\vphi_{o}-\vphi_{s}$, and suboptimal, $\vec{z}_{s}=\vphi_{s}-\vphi_{o}$ by $y_o=+1$ and $y_s=-1$ respectively. 
Note, a negative example is only created as long as $J_s$ actually results in a worse makespan, i.e., $C_{\max}^{(s)}\gneq C_{\max}^{(o)}$, since there can exist situations in which more than one operation can be considered optimal.

The preference learning problem is specified by a set of preference pairs,
\begin{equation}\label{eq:S:jsp}
	S := \bigcup_{\{\vec{x}_i\}_{i=1}^{N_{\text{train}}}}
    \condset{
        \left\{\vec{z}_o,+1\right\},\left\{\vec{z}_s,-1\right\}}{\forall (o,s) 
        \in \mathcal{O}^{(k)} \times \mathcal{S}^{(k)}}_{k=1}^K 
    \subset \Phi\times Y 
\end{equation}
where $\Phi\subset\mathcal{F}$ is the training set of $d$ features, 
$Y=\{-1,+1\}$ is the outcome space, and at dispatch $k\in\{1,..,K\}$, 
$o\in\mathcal{O}^{(k)},~s\in \mathcal{S}^{(k)}$
are optimal and suboptimal dispatches, respectively.
Note, $\mathcal{O}^{(k)}\cup\mathcal{S}^{(k)}=\mathcal{L}^{(k)}$, and 
$\mathcal{O}^{(k)}\cap\mathcal{S}^{(k)}=\emptyset$. 

For \jsp\ there are $d=\NrFeatTotal$ features (cf. the step-by-step varying features from \cref{tbl:features}), and the training set is created in the manner described in \cref{sec:gentrainingdata}.

\todoFind{Which is being used, ordinal or logistic regression?}
Logistic regression makes decisions regarding optimal dispatches and at the same time efficiently estimates a posteriori probabilities. When using linear classification model (cf. \cref{sec:ord:linpref}), i.e., 
\begin{equation}\label{eq:jsp:linweights}
	\hat{\pi}(\vchi)=\sum_{i=1}^d 
	w_i\phi_i(\vchi)=\inner{\vec{w}}{\vphi(\vchi)},
\end{equation}
the optimal $\vec{w}^*$ obtained from the preference set can be used on any new 
data point (i.e. partial schedule), $\vchi$, and their inner product is 
proportional to probability estimate \cref{eq:prob}. 
%Similarly for non-linear classification models. 
Hence, for each job on the job-list, $J_j\in\mathcal{L}$, let $\vphi_j$ denote 
its corresponding  post-decision state. Then the job chosen to be dispatched, 
$J_{j^*}$, is the one corresponding to the highest preference estimate, i.e.,
\begin{equation}\label{eq:lin}
	J_{j^*}=\argmax_{J_j\in \mathcal{L}}\; \hat{\pi}(\vphi_j)
\end{equation}
where $\hat{\pi}(\cdot)$ is the classification model obtained by the preference 
set, $S$, defined by \cref{eq:S:jsp}. 


\section{Interpreting linear classification models}

When using a feature space based on SDRs, the linear classification models can 
very easily be interpreted as composite dispatching rules with predetermined 
weights. 
%Similarly for non-linear classification models, however, they are harder to 
%visualise since the features have to be projected on a kernel.
Looking at the features description in \cref{tbl:features} it is possible for 
the ordinal regression to `discover' the weights $\vec{w}$ in order for 
\cref{eq:jsp:linweights} corresponding applying a \sdr~from \cref{sec:SDR}. 

The optimum makespan is known for each problem instance. 
At each time step (i.e. layer of the game tree) a number of feature pairs are created. 
They consist of the features $\vphi_o$ resulting from optimal dispatches $o\in\mathcal{O}^{(k)}$, versus features $\vphi_s$ resulting from suboptimal dispatches $s\in\mathcal{S}^{(k)}$ at time $k$. 
In particular, each job is compared against another job of the job-list, 
$\mathcal{L}^{(k)}$, and if the makespan differs, i.e., $C_{\max}^{(s)}\gneq 
C_{\max}^{(o)}$, an optimal/suboptimal pair is created. 
However if the makespan would be unaltered, the pair is omitted since they give the same optimal makespan. 
This way, only features from a dispatch resulting in a suboptimal solution is labelled undesirable.

The approach taken here is to verify analytically, at each time step, by retaining the current temporal schedule as an initial state, whether it can indeed \emph{somehow} yield an optimal schedule by manipulating the remainder of the sequence. 
This also takes care of the scenario that having dispatched a job resulting in a different temporal makespan would have resulted in the same final makespan if another optimal dispatching sequence would have been chosen. 
That is to say the data generation takes into consideration when there are multiple optimal solutions to the same problem instance. 

\section{Time dependant dispatching rules}
At each dispatch iteration $k$, a number of preference pairs are created, which is then repeated for all the $N_{\text{train}}$ problem instances created. 
A separate data set is deliberately created for each dispatch iterations, as 
the initial feeling is that dispatch rules used in the beginning of the 
schedule building process may not necessarily be the same as in the middle or 
end of the schedule. As a result there are $K$ linear scheduling rules for 
solving a $n \times m$ \jsp. \todoWrite{No longer one model for all 
steps}

\section{Selecting preference pairs}\label{sec:S:strategies}
Defining the size of the preference set as $l=\abs{S}$, then \cref{eq:S:jsp} gives the size of the feature training set as $\abs{\Phi}=\frac{1}{2}l$.
If $l$ is too large, than sampling needs to be done in order for the ordinal regression to be computationally feasible.

%Due to the nature of the sequence representation, the earlier stages of the 
%dispatching are more or less equivalent (and thus irrelevant), hence it is 
%appropriate to follow some random optimal path to begin with and then go after 
%some (if not all possible) optimal paths until completion at step $K$. 

The strategy approached in  \cref{InRu11a} was to follow some optimal job 
$J_j\in\mathcal{O}^{(k)}$ (chosen at random), thus creating 
$\abs{\mathcal{O}^{(k)}}\cdot\abs{\mathcal{S}^{(k)}}$ feature pairs at each 
dispatch $k$, resulting in a training size of,
\begin{equation}\label{eq:sizeS_b}
	l =  \sum_{i=1}^{N_{\text{train}}} \left(2 \abs{\mathcal{O}^{(k)}_i}\cdot \abs{\mathcal{S}^{(k)}_i} \right)
\end{equation}
For the problem spaces considered there, that sort of simple sampling of the state space was sufficient for a favourable outcome. However for a considerably harder problem spaces (see \cref{ch:problemstructure}), preliminary experiments were not satisfactory. 

A brute force approach was adopted to investigate the feasibility of finding optimal weights $\vec{w}$ for \cref{eq:jsp:linweights}. 
By applying CMA-ES (discussed thoroughly in \cref{ch:esmodels}) to directly minimize the mean $C_{\max}$  w.r.t. the weights $\vec{w}$, gave a considerably more favourable result in predicting optimal versus suboptimal dispatching paths. 
So the question put forth is, why was the ordinal regression not able to detect it?
The nature of the CMA-ES is to explore suboptimal routes until it converges to an optimal one. 
Implying that the previous approach of only looking into one optimal route is not sufficient information. 
Suggesting that the training set should incorporate a more complete knowledge about \emph{all} possible preferences, i.e., make also the distinction between suboptimal and sub-suboptimal features, etc.  
This would require a Pareto ranking for the job-list, $\mathcal{L}$, which can 
be used to make the distinction to which feature sets are equivalent, better or 
worse, and to what degree (i.e. giving a weight to the preference)? 
By doing so, the training set becomes much greater, which of course would again need to be sampled in order to be computationally feasible to learn. 

For instance \cite{Siggi05} used decision trees to `rediscover' LPT by using the dispatching rule to create its training data. The limitations of using heuristics to label the training data is that the learning algorithm will mimic the original heuristic (both when it works poorly and well on the problem instances) and does not consider the real optimum. In order to learn new heuristics that can outperform existing heuristics then the training data needs to be correctly labelled. This drawback is confronted in \citep{Malik08,Russell09,Siggi10} by using an optimal scheduler, computed off-line. 

These aspects are the main motivation for the data generation in this dissertation. 
All problem instances are correctly labelled w.r.t. their optimum makespan, found with analytical means.\footnote{Optimal solution were found using \cite{gurobi}, a commercial software package for solving large-scale linear optimization and a state-of-the-art solver for mixed integer programming.} 
In order to create training instances (and subsequently preference pairs) both a features resulting in optimal solutions are gathered (following optimal trajectories) and features that would have been chosen if a dispatching rule had been implemented (following DR trajectories). 
In the latter case, the trajectories pursued here, will be the SDRs from \cref{sec:SDR} as well as randomly dispatching operations.

To summarise, one needs to consider two main aspects of the generation of the 
training data
\begin{enumerate*}[label={\emph{\Roman*})},
        itemjoin={{? }}, itemjoin*={{? And }}, after={{}}]
    \item what sort of rankings should be compared during each step
    \item which path(s) should be investigated
    \begin{enumerate*}[label=\textit{\Roman{enumi}.\roman*)}, before={{? }},
        itemjoin={{? }}, itemjoin*={{? Or }}, after={{}}]
    \item Pursuing solely optimal trajectories
    \item Creating random dispatches
    \item following other means: CMA-ES computed weights, \sdr s, etc.
    \end{enumerate*}
\end{enumerate*}

\subsection{Ranking strategies}
The following ranking strategies were implemented for adding preference pairs 
to $S$ defined by \cref{eq:S:jsp}, they were first reported in \cref{InRu15a},
\begin{description}
	\item[Basic ranking, $S_b$,] i.e., all optimum rankings $r_1$ versus all 
	possible suboptimum rankings $r_i$, $i\in\{2,\ldots,n'\}$, preference pairs 
	are added -- same basic set-up introduced in \cref{InRu11a}. Note, $|S_b|$ 
	is defined in \cref{eq:sizeS_b}.
	\item[Full subsequent rankings, $S_f$,] i.e., all possible combinations of $r_i$ and $r_{i+1}$ for $i\in\{1,\ldots,n'\}$, preference pairs are added.
	\item[Partial subsequent rankings, $S_p$,] i.e., sufficient set of combinations of $r_i$ and $r_{i+1}$ for $i\in\{1,\ldots,n'\}$, are added to the training set -- e.g. in the cases that there are more than one operation with the same ranking, only one of that rank is needed to compared to the subsequent rank. Note that $S_p\subset S_f$.
	\item[All rankings, $S_a$,] denotes that all possible rankings were explored, i.e.,
	$r_i$ versus $r_j$ for $i,j\in\{1,\ldots,n'\}$ and $i\neq j$, preference pairs are added.
\end{description}
where $r_1>r_2>\ldots>r_{n'}$ ($n'\leq n$) are the rankings of the job-list, 
$\mathcal{L}^{(k)}$, at time step $k$.


\subsection{Trajectory strategies}
The following trajectory strategies were explored for adding preference pairs to $S$ defined by \cref{eq:S:jsp},
\begin{description}
	\item[Optimum trajectory, $\Phi^{OPT}$,] at each dispatch some (random) 
    optimal task is dispatched. This is also referred to following the expert 
    policy, $\pi_\star$.	\item[SPT trajectory, $\Phi^{SPT}$,] at each 
    dispatch 
    the task 
	corresponding to shortest processing time is dispatched, i.e., following 
	\sdr~SPT.
	\item[LPT trajectory, $\Phi^{LPT}$,] at each dispatch the task 
	corresponding to largest processing time is dispatched, i.e., following 
	\sdr~LPT.
	\item[LWR trajectory, $\Phi^{LWR}$,] at each dispatch the task 
	corresponding to least work remaining is dispatched, i.e., following 
	\sdr~LWR.
	\item[MWR trajectory, $\Phi^{MWR}$,] at each dispatch the task 
	corresponding to most work remaining is dispatched, i.e., following 
	\sdr~MWR.
	\item[Random trajectory, $\Phi^{RND}$,] at each dispatch some random task 
	is dispatched.
	\item[CMA-ES trajectory, $\Phi^{CMA}$,] at each dispatch the task 
	corresponding to highest priority, computed with fixed weights $\vec{w}$, 
	which were obtained by optimising the mean for \fullnamerho, with CMA-ES. 
	\item[All trajectories, $\Phi^{ALL}$,] denotes all aforementioned 
	trajectories were explored, i.e., 
	\begin{equation}
		\Phi^{ALL}:=\condset{\Phi^{A}}{\forall A \in \{OPT,CMA,MWR,LWR,RND\}}.
	\end{equation}
\end{description}
In the case of the $\Phi^{\langle SDR \rangle}$ and $\Phi^{CMA}$ trajectories 
it is sufficient to explore each trajectory exactly once for each problem 
instance. Whereas, for $\Phi^{OPT}$ and $\Phi^{RND}$ there can be several 
trajectories worth exploring, however, only one is chosen (at random). It is 
noted that since the number of problem instances, $N_{\text{train}}$, is 
relatively large, it is deemed sufficient to explore one trajectory for each 
instance, in those cases as well.

These trajectory strategies were initially introduced in \cref{InRu15a}, save 
for $S^{CMA}$ and the SDR-based ones, however the latter are currently 
addressed since, e.g., LWR is considered more favourable for \fsp\ rather than 
MWR (cf. \cref{ch:problemstructure}). 

\subsection{Experimental study}\label{sec:expr:locallin}
To test the validity of different ranking and strategies from \cref{sec:S:strategies}, a training set of $N_{\text{train}}$ problem instances of $6\times5$ \jsp\ and \fsp\ summarised in \cref{tbl:data} (omitting the job and machine variations of \jsp). The size of the preference set, $S$, for different trajectory and ranking strategies are depicted in \cref{fig:sizeofprefset}. Note, for now $10\times10$ problem spaces will be ignored, due to the extreme computational cost of correctly labelling each trajectory. However, in \cref{sec:pref:scalability} an optimum path will be pursued for said dimension.

A linear preference (PREF) model was created for each preference set, $S$, for every problem space considered. A box-plot with \fullnamerho, is presented in \cref{fig:boxplot:gentrdat}. %\Cref{fig:track:boxplot:p1} depicts different ranking strategies for a fixed trajectory, whereas \cref{fig:rank:boxplot:p1} depicts different trajectory strategies for a fixed ranking. 
From the figures it is apparent there can be a performance edge gained by implementing a particular ranking or trajectory strategy, moreover the behaviour is analogous across different disciplines. 


\begin{figure}
	\includegraphics[width=\textwidth]{{trdat.size.6x5}.pdf}
	\caption[Size of training set, $\abs{\Phi}$]{Size of training set, $\abs{\Phi}$, for different trajectory strategies ($N_{\text{train}}=500$)}
	\label{fig:size:prefset}
\end{figure}


\begin{figure}
	\includegraphics[width=\textwidth]{{prefdat.size.6x5}.pdf}
	\caption[Size of preference set, $\abs{S}$]{Size of preference set, $l=\abs{S}$, for different trajectory strategies and ranking schemes (where $N_{\text{train}}=500$) }
	\label{fig:size:trdat}
\end{figure}

\subsubsection{Ranking strategies}
There is no statistical difference between $S_f$ and $S_p$ ranking-schemes 
across all disciplines (cf. \cref{fig:track:boxplot:p1,fig:track:boxplot:p2}), 
which is expected since $S_f$ is designed to contain the same preference 
information as $S_f$. However neither of the Pareto ranking-schemes outperform 
the original $S_b$ set-up from \cref{InRu11a}. The results hold for the test 
set as well. 

Combining the ranking schemes, $S_{all}$, improves the individual ranking-schemes across all disciplines, except in the case of $S_b^{opt}\big|_{\mathcal{P_1}}$ and $S_b^{rnd}\big|$\!\Problem{2}, in which case there were no statistical difference. Now, for the test set, the results hold, however there is no statistical difference between $S_b$ and $S_{all}$ for most trajectories $\{S^{opt},S^{cma},S^{rnd}\}\big|$\!\Problem{1} and  $\{S^{opt},S^{rnd}\}\big|$\!\Problem{2}. Now, whereas a smaller preference set is preferred, its opted to use the $S^{b}$ ranking scheme henceforth. 

Moreover, it is noted that the learning algorithm is able to significantly outperform the original heuristics, MWR and CMA-ES (white), used to create the training data $S^{mwr}$ (grey) and $S^{cma}$ (yellow), respectively (cf. \cref{fig:track:boxplot:p1,fig:track:boxplot:p2}). For both \Problem{1} and \Problem{2}, linear ordinal regression models based on $S^{mwr}$ are significantly better than $MWR$, irrespective of the ranking schemes. Whereas the fixed weights found via CMA-ES are only outperformed by linear ordinal regression models based on $\{S_b^{cma},S_{all}^{cma}\}$. This implies that ranking scheme needs to be selected appropriately. Result hold for the test data.

\subsubsection{Trajectory strategies}
Learning preference pairs from a good scheduling policies, such as $S^{cma}$ and $S^{mwr}$, gave considerably more favourable results than tracking optimal paths (cf. \cref{fig:track:boxplot:p1,fig:track:boxplot:p2}). Suboptimal routes are preferred when dealing with problem{1} (for all ranking schemes), however when encountering problem{2} the choice of ranking schemes can yield the exact opposite.

It is particularly interesting there is no statistical difference between $S^{opt}$ and $S^{rnd}$ for both $\{S_{b},S_{f}\}\big|$\Problem{1} and $\{S_b,S_f,S_p\}\big|$\Problem{2} ranking-schemes. That is to say, tracking optimal dispatches gives the same performance as completely random dispatches. This indicates that exploring only optimal trajectories can result in a training set which the learning algorithm is inept to determine good dispatches in the circumstances when newly encountered features have diverged from the learned feature set labelled to optimum solutions. 

Finally, $S^{all}$ gave the best combination across all disciplines. Adding suboptimal trajectories with the optimal trajectory gives the learning algorithm a greater variety of preference pairs for getting out of local minima.

\subsubsection{Following CMA-ES guided trajectory}
The rational for using the $S^{cma}$ strategy was mostly due to the fact a linear classifier is creating the training data (using the weights found via CMA-ES optimisation), hence the training data created should be linearly separable, which in turn should boost the training accuracy for a linear classification learning model. However, this strategy is easily outperformed by the single priority based dispatching rule MWR guiding the training data collection, $S^{mwr}$. 


\subsection{Summary and conclusion }
As the experimental results showed in \cref{sec:expr:locallin}, the ranking of optimal\footnote{Here the tasks labelled `optimal' do not necessarily yield the optimum makespan (except in the case of following optimal trajectories), instead these are the optimal dispatches for the given partial schedule.} and suboptimal features are of paramount importance. The subsequent rankings are not of much value, since they are disregarded anyway. However, the trajectories to create training instances have to be varied.

Unlike \citep{Siggi10,Malik08,Russell09}, learning only on optimal training data was not fruitful. However, inspired by the original work by \cite{Siggi05}, having DR guide the generation of training data (except correctly labelling with analytic means) gave meaningful preference pairs which the learning algorithm could learn. In conclusion, henceforth, the training data will be generate with $S_{b}^{all}$ scheme.


\section{Time independent dispatching rules}\label{sec:pref:scalability}
As stated in \cref{sec:S:strategies}, a separate data set is deliberately 
created for each dispatch iteration, as it is initially assumed that dispatch 
rules used in the schedule building might differ in the beginning, the middle 
or towards the end of the process. As a result there is a local linear model 
for each dispatch; a total of $K$ linear models for solving $n\times m$ 
\jsp. Now, if we were to create a global rule, then there would have to be one 
model for all dispatches iterations. The approach in \cref{InRu11a} was to take 
the mean weight for all stepwise linear models, i.e., 
$\bar{w}_i=\frac{1}{K}\sum_{k=1}^K w_i^{(k)}$ where $\vec{w}^{(k)}$ is 
the linear weight resulting from learning preference set $S^{(k)}$ at dispatch 
$k$. 

A more sophisticated way, would be to create a \emph{new} linear model, where 
the preference set, $S$, is the union of the preference pairs across the $K$ 
dispatches. This would amount to a substantial training set, and for $S$ to be 
computationally feasible to learn, $S$ has to be filtered to size $l_{\max}$.

\todoWrite{No longer the case, one model instead of $K$ stepwise-models}

\section{Discussion and conclusions}
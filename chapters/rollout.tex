\HeaderQuote{Now, here, you see, it takes all the running you can do, to 
    keep in the same place. If you want to get somewhere else, you must run at 
    least twice as fast as that!}{The Queen} 

\chapter{Pilot Model}\label{ch:rollout}

\FirstSentence{R}{oll-out algorithms, also known as Pilot Method} 
\citep{Bertsekas97,Duin99}, for combinatorial optimisation aim to improve 
performance by sequential application of a pilot heuristic which completes the 
remaining $(K-k)$ steps. Roll-outs for \JSP\ have been conducted by 
\citet{Ru12}. 
Continuing with that work, \citet{Geirsson2012} compares several pilot 
heuristics, e.g., \emph{Randomly Chosen Dispatch Rules} which is similar 
to \phiSDRRelated\ (but here one roll-out per fixed SDR). 
The motivation being that SDR-based roll-outs are of higher quality than 
random ones which require less computational budget. 
However, \citeauthor{Geirsson2012} notes that performance w.r.t. traditional 
random roll-outs is statistically insignificant and not worth the overhead of 
implementing various SDRs beforehand. 

\citeauthor{Geirsson2012} reworks the roll-out algorithm as an 
$\abs{\mathcal{L}}$-armed bandit,\footnote{In probability theory, the
    multi-armed bandit problem \citep{badit:book} describes a gambler at a row 
    of slot machines who has to decide which machines to play, i.e., pull its 
    lever, in  order to maximise his rewards, that are specific to each 
    machine. The gambler also has to decide how many times to play each machine 
    and in which order to play them. The gambler's actions are referred to as 
    \emph{pilot-heuristic}.} 
i.e., each job of the job-list are the levers. Since the best job, $j^*$, to 
dispatch at step $k$, is not known beforehand, all available jobs are 
evaluated using roll-outs. 
As a result, using the features \phiRNDRelated, the weights $\vec{w}$ yield the 
deterministic pilot heuristic. Although in \citeauthor{Geirsson2012}'s work, 
other statistics were used for guidance, e.g., quartile and octile. 

\pagebreak
\emph{Remark}: the roll-outs considered in \cref{tbl:features}, are with a 
relatively frugal budget, only 100 roll-outs per lever is considered -- all 
evenly distributed between levers. 
However, using the multi-armed bandit paradigm, it's possible to allocate 
roll-outs originating from the job-list with bias towards more promising levers.

Note, in the case of random roll-outs (namely \phiRNDRelated) then the final makespan resulting in the pursued trajectory might not necessarily be the best final makespan found during the dispatching process, this is reported as its `fortified' result, denoted $\rho_{\text{fort.}}$. 

\section{Single feature roll-outs}\label{sec:singlerollout}

A model based on each of the extremal (i.e. minimum or maximum value) values 
for \phiGlobalRelated\ was created. The three best models for each problem 
space is reported, namely minimum values for $\phiSPT$, $\phiRNDmean$ and 
$\phiRNDmin$. 
Box-plot for \namerho, is depicted in \cref{fig:rollout1:boxplot}, and its 
main statistics are given in \cref{tbl:rollout1:boxplot}.
In all cases, the fortified makespan was significantly better than the final 
makespan of the pursued trajectory, save for \frnd{6}{5} using minimum 
$\phiRNDmin$, which was statistically insignificant.

Revisiting \cref{fig:SDR:boxplot}, then SPT is never the best SDR for any of 
the problem spaces (cf. \cref{fig:SDR:boxplot}).
However, choosing the minimum SPT from every possible operation onwards 
gives the best result of \phiSDRRelated. This twist in SPT application boosts 
performance by 
\begin{enumerate*}
    \item $\Delta\rho\approx -39\%$ for \jrnd{6}{5}
    \item $\Delta\rho\approx -26\%$ for \frnd{6}{5}
    \item $\Delta\rho\approx -43\%$ for \jrnd{10}{10}
\end{enumerate*}
Bearing \cref{fig:diff:case:OPT:10x10} in mind, then notice how 
$\xi^{\star}_{-\phiSPT}$ differs from $\xi^{\star}_{-\phiproc}$. This implies 
that it's better to \emph{not} choose the job SPT would `normally' pick, in the 
initial stages. As we saw from \cref{fig:diff:boxplot:BDR} then pursuing SPT 
drastically derailed $\rho$ performance if it ventured off the optimal 
trajectory. 
However, SPT $\xi^{\star}_{-\phiproc}$ shows that on average SPT is a policy that 
is likely to be optimal. 
Therefore making SPT roll-outs with $\phiSPT$, namely, repeatedly applying a 
$(K-k)$-lookahead for $-\phiproc$. 
Then $\phiSPT$ manages to overcome the shortcomings of pursuing $-\phiproc$ on 
its own for only a 1-step lookahead.

Regarding random roll-outs, the greedy $\phiRNDmin$ came out on top for 
\Problem[6\times5]{\train}, whereas for \jrnd{10}{10} then a better fortified 
mean result was achieved by following $\phiRNDmean$.

\begin{figure}[p]\centering 
    \includegraphics[width=\textwidth]{ALL/{boxplot.single.rollout.ALL}.pdf}
    \caption[Box-plot for single roll-out features]{Box-plot for \namerho, for 
    top three single extremal values for \phiGlobalRelated\ roll-out using 
    \Problem{\train}. SPT shown for reference on the far right.}
    \label{fig:rollout1:boxplot}
\end{figure}

\begin{table}[p]
  \caption{Main statistics for top three single extremal values for 
  \phiGlobalRelated\ rollout using \Problem{\train}}
  \label{tbl:rollout1:boxplot}
  \centering
  {\small
  \begin{tabular}{lr@{.}c@{ }c@{ }lrrrrrr}
 \toprule
 & $\vphi$ & Ext. & NrFeat &  & Min. & 1st Qu. & Median & Mean & 
 3rd Qu. & Max. \\ 
 \midrule \multirow{5}{*}{\jrnd{6}{5}} 
 &$\phiSPT$&min & 1 & $\rho$ & 0.00 & 2.15 & 4.86 & 5.94 & 8.98 & 27.90 \\ 
 &$\phiRNDmean$&min & 1 & $\rho$ & 0.00 & 1.43 & 4.00 & 4.83 & 7.16 & 23.47 \\ 
 &$\phiRNDmean$&min & 1 & $\rho_{\text{fort.}}$ & 0.00 & 0.00 & 0.00 & 0.90 & 
 1.28 & 10.69 \\ 
 &$\phiRNDmin$&min & 1 & $\rho$ & 0.00 & 0.00 & 0.00 & 1.14 & 1.76 & 10.31 \\ 
 &$\phiRNDmin$&min & 1 & $\rho_{\text{fort.}}$ & 0.00 & 0.00 & 0.00 & 
 \textbf{0.78} & 0.94 & 8.46 \\ 
 \midrule \multirow{5}{*}{\frnd{6}{5}} 
 &$\phiSPT$&min & 1 & $\rho$ & 0.00 & 3.15 & 7.15 & 7.71 & 11.19 & 30.24 \\ 
 &$\phiRNDmean$&min & 1 & $\rho$ & 0.00 & 1.38 & 3.42 & 4.06 & 6.32 & 15.12 \\ 
 &$\phiRNDmean$&min & 1 & $\rho_{\text{fort.}}$ & 0.00 & 0.00 & 0.37 & 
 \textbf{1.02} & 1.57 & 7.53 \\ 
 &$\phiRNDmin$&min & 1 & $\rho$ & 0.00 & 0.00 & 0.66 & 1.51 & 2.25 & 11.26 \\ 
 &$\phiRNDmin$&min & 1 & $\rho_{\text{fort.}}$ & 0.00 & 0.00 & 0.39 & 1.23 & 
 1.87 & 11.26 \\ 
 \midrule \multirow{5}{*}{\jrnd{10}{10}} 
 &$\phiSPT$&min & 1 & $\rho$ & 1.85 & 8.99 & 11.67 & 11.82 & 14.12 & 26.09 \\ 
 &$\phiRNDmean$&min & 1 & $\rho$ & 2.13 & 7.33 & 9.49 & 9.73 & 11.76 & 19.53 
 \\ 
 &$\phiRNDmean$&min & 1 & $\rho_{\text{fort.}}$ & 0.00 & 4.46 & 6.01 & 
 \textbf{5.97} & 7.46 & 13.32 \\ 
 &$\phiRNDmin$&min & 1 & $\rho$ & 0.25 & 6.37 & 8.72 & 8.90 & 10.98 & 20.10 \\ 
 &$\phiRNDmin$&min & 1 & $\rho_{\text{fort.}}$ & 0.00 & 5.47 & 7.33 & 7.24 & 
 9.07 & 16.38 \\
 \bottomrule
 \end{tabular}}
\end{table}

\section{Multi feature roll-outs}\label{sec:multirollout}
When using random roll-outs there are many strategies to choose which job is 
the most promising for future roll-outs. 
For this reason, let's consider the preference models from \cref{ch:prefmodels} 
with additional features \phiGlobalRelated\ as the weights can now be 
considered as its deterministic pilot heuristic. 

\pagebreak
If we only use \phiSDRRelated\ it requires 4 deterministic $(K-k)$ step 
roll-outs at each time step. Whereas, introducing \phiRNDRelated\ costs an 
additional 100 random roll-outs for each time step.
Therefore, we'll consider both using only the first \NrFeatSDR\ features due to it's 
relatively low computational budget, and also the computationally intensive 
full model of \NrFeatGlobal\ features.

The experimental set-up will consider the stepwise sampling biases from 
\cref{sec:pref:bias}
\begin{enumerate*}
    \item \ref{bias:equal} (i.e. equal probability) 
    \item adjusted \ref{bias:dbl2nd} (i.e. double emphasis on second half)
\end{enumerate*}
Furthermore, the training data will be using either the expert policy or 
following the weights obtained from minimising w.r.t. $\minCmax$ as defined in 
\cref{eq:cma:makespan}. Both trajectories were detailed in 
\cref{sec:trdat:param:tracks:passive}.
Box-plot for \namerho, is depicted in \cref{fig:rollout:boxplot}, and its main 
statistics are given in \cref{tbl:rollout:boxplot} 

\begin{figure}[t!]\centering 
    \includegraphics[width=\textwidth]{ALL/{boxplot.multi.rollout.ALL}.pdf}
    \vspace{-6pt}
    \caption[Box-plot for multi roll-out features]{Box-plot for \namerho, using 
        roll-out features. Corresponding models only using \phiLocalRelated\ 
        features shown for reference on the far left.}
        \label{fig:rollout:boxplot}
\end{figure}

\begin{table}[ht]
\caption[Main statistics for \phiGlobalRelated\ rollout preference models]{
    Main statistics for \phiGlobalRelated\ rollout preference models using 
    \Problem{\train}}
\label{tbl:rollout:boxplot}
\centering
{\footnotesize \setlength{\tabcolsep}{3pt}
\begin{tabular}{lcc@{ }c@{ }lrrrrrr}
\toprule
& Bias & Track & NrFeat & & Min. & 1st Qu. & Median & Mean & 3rd Qu. 
& Max. \\ 
  \midrule \multirow{16}{*}{\jrnd{6}{5}}
& equal & OPT & 20 & $\rho$ & 0.00 & 2.60 & 5.87 & 6.53 & 9.39 & 21.34 \\ 
& equal & OPT & 20 & $\rho_{\text{fort.}}$ & 0.00 & 0.95 & 3.43 & 4.08 & 6.43 & 
16.22 \\ 
& equal & OPT & 24 & $\rho$ & 0.00 & 0.00 & 0.64 & 1.35 & 1.94 & 10.10 \\ 
& equal & OPT & 24 & $\rho_{\text{fort.}}$ & 0.00 & 0.00 & 0.00 & 0.76 & 0.93 & 
8.03 \\ 
& adjdbl2nd & OPT & 20 & $\rho$ & 0.00 & 1.95 & 4.62 & 5.65 & 8.39 & 27.22 \\ 
& adjdbl2nd & OPT & 20 & $\rho_{\text{fort.}}$ & 0.00 & 0.83 & 3.64 & 4.29 & 
6.92 & 18.50 \\ 
& adjdbl2nd & OPT & 24 & $\rho$ & 0.00 & 0.00 & 0.00 & 1.26 & 1.64 & 14.18 \\ 
& adjdbl2nd & OPT & 24 & $\rho_{\text{fort.}}$ & 0.00 & 0.00 & 0.00 & 
\textbf{0.71} & 0.88 & 
11.16 \\ 
& equal & $\minCmax$ & 20 & $\rho$ & 0.00 & 1.55 & 4.35 & 5.07 & 7.80 & 27.22 
\\ 
& equal & $\minCmax$ & 20 & $\rho_{\text{fort.}}$ & 0.00 & 0.46 & 3.03 & 3.71 & 
5.81 & 18.50 \\ 
& equal & $\minCmax$ & 24 & $\rho$ & 0.00 & 0.00 & 0.55 & 1.29 & 1.92 & 12.42 
\\ 
& equal & $\minCmax$ & 24 & $\rho_{\text{fort.}}$ & 0.00 & 0.00 & 0.00 & 0.80 & 
0.98 & 9.89 \\ 
& adjdbl2nd & $\minCmax$ & 20 & $\rho$ & 0.00 & 1.57 & 4.50 & 5.22 & 7.86 & 
27.22 \\ 
& adjdbl2nd & $\minCmax$ & 20 & $\rho_{\text{fort.}}$ & 0.00 & 0.82 & 3.52 & 
4.12 & 6.39 & 17.50 
\\ 
& adjdbl2nd & $\minCmax$ & 24 & $\rho$ & 0.00 & 0.00 & 0.25 & 1.31 & 2.04 & 
9.11 
\\ 
& adjdbl2nd & $\minCmax$ & 24 & $\rho_{\text{fort.}}$ & 0.00 & 0.00 & 0.00 & 
0.82 & 1.14 & 8.03 
\\ 
\midrule \multirow{16}{*}{\frnd{6}{5}}    
& equal & OPT & 20 & $\rho$ & 0.00 & 2.21 & 5.12 & 5.94 & 8.71 & 29.12 \\ 
& equal & OPT & 20 & $\rho_{\text{fort.}}$ & 0.00 & 0.85 & 3.09 & 3.71 & 5.83 & 
16.29 \\ 
& equal & OPT & 24 & $\rho$ & 0.00 & 0.00 & 0.76 & 1.39 & 2.19 & 9.28 \\ 
& equal & OPT & 24 & $\rho_{\text{fort.}}$ & 0.00 & 0.00 & 0.00 & \textbf{0.84} 
& 1.26 & 8.06 \\ 
& adjdbl2nd & OPT & 20 & $\rho$ & 0.00 & 2.55 & 5.07 & 5.88 & 8.42 & 29.12 \\ 
& adjdbl2nd & OPT & 20 & $\rho_{\text{fort.}}$ & 0.00 & 1.19 & 3.23 & 3.91 & 
5.94 & 16.29 \\ 
& adjdbl2nd & OPT & 24 & $\rho$ & 0.00 & 0.00 & 0.71 & 1.39 & 2.13 & 10.04 \\ 
& adjdbl2nd & OPT & 24 & $\rho_{\text{fort.}}$ & 0.00 & 0.00 & 0.19 & 1.04 & 
1.55 & 8.67 \\ 
& equal & $\minCmax$ & 20 & $\rho$ & 0.00 & 2.42 & 5.33 & 5.97 & 8.32 & 28.38 
\\ 
& equal & $\minCmax$ & 20 & $\rho_{\text{fort.}}$ & 0.00 & 0.82 & 2.99 & 3.56 & 
5.63 & 16.29 \\ 
& equal & $\minCmax$ & 24 & $\rho$ & 0.00 & 0.00 & 0.73 & 1.46 & 2.27 & 9.59 \\ 
& equal & $\minCmax$ & 24 & $\rho_{\text{fort.}}$ & 0.00 & 0.00 & 0.00 & 0.96 & 
1.45 & 8.46 \\ 
& adjdbl2nd & $\minCmax$ & 20 & $\rho$ & 0.00 & 2.77 & 5.74 & 6.44 & 9.20 & 
29.12 \\ 
& adjdbl2nd & $\minCmax$ & 20 & $\rho_{\text{fort.}}$ & 0.00 & 1.17 & 3.29 & 
3.93 & 5.93 & 16.29 
\\ 
& adjdbl2nd & $\minCmax$ & 24 & $\rho$ & 0.00 & 0.00 & 0.63 & 1.42 & 2.45 & 
10.52 \\ 
& adjdbl2nd & $\minCmax$ & 24 & $\rho_{\text{fort.}}$ & 0.00 & 0.00 & 0.17 & 
1.03 & 1.63 & 9.75 
\\ 
\midrule \multirow{16}{*}{\jrnd{10}{10}}
& equal & OPT & 20 & $\rho$ & 1.89 & 14.99 & 19.41 & 20.06 & 24.64 & 45.14 \\ 
& equal & OPT & 20 & $\rho_{\text{fort.}}$ & 1.89 & 8.92 & 11.36 & 11.36 & 
13.94 & 21.29 \\ 
& equal & OPT & 24 & $\rho$ & 0.87 & 7.30 & 9.72 & 10.15 & 12.92 & 23.35 \\ 
& equal & OPT & 24 & $\rho_{\text{fort.}}$ & 0.00 & 4.21 & 5.66 & 5.79 & 7.20 & 
13.49 \\ 
& adjdbl2nd & OPT & 20 & $\rho$ & 5.73 & 12.39 & 15.96 & 16.23 & 19.54 & 37.12 
\\ 
& adjdbl2nd & OPT & 20 & $\rho_{\text{fort.}}$ & 3.93 & 8.96 & 11.24 & 11.40 & 
13.57 & 22.30 \\ 
& adjdbl2nd & OPT & 24 & $\rho$ & 0.34 & 6.18 & 8.19 & 8.59 & 10.43 & 22.14 \\ 
& adjdbl2nd & OPT & 24 & $\rho_{\text{fort.}}$ & 0.00 & 4.08 & 5.79 & 5.79 & 
7.15 & 12.79 \\ 
& equal & $\minCmax$ & 20 & $\rho$ & 2.76 & 9.13 & 12.16 & 12.26 & 15.14 & 
34.60 
\\ 
& equal & $\minCmax$ & 20 & $\rho_{\text{fort.}}$ & 1.89 & 6.85 & 9.12 & 9.28 & 
11.23 & 20.07 \\ 
& equal & $\minCmax$ & 24 & $\rho$ & 0.62 & 5.24 & 7.09 & 7.33 & 8.90 & 19.43 
\\ 
& equal & $\minCmax$ & 24 & $\rho_{\text{fort.}}$ & 0.24 & 3.54 & 4.91 & 4.93 & 
6.31 & 12.38 \\ 
& adjdbl2nd & $\minCmax$ & 24 & $\rho$ & 0.63 & 4.44 & 6.44 & 6.58 & 8.39 & 
21.36 \\ 
& adjdbl2nd & $\minCmax$ & 24 & $\rho_{\text{fort.}}$ & 0.57 & 3.31 & 5.06 & 
\textbf{4.92} & 
6.35 & 10.93 \\ 
\bottomrule
\end{tabular}}
\end{table}

First off, there was no statistical difference between stepwise sampling 
strategy. Exceptions for \jrnd{6}{5} and \jrnd{10}{10} being 20.1.OPT and 
24.1.OPT, favouring adjusted \ref{bias:dbl2nd}, same as \cref{sec:pref:bias} 
previously showed for 16.1 models.
However, w.r.t. its fortified result there was no significant difference any 
more.

Furthermore, the choice of trajectory starts to become irrelevant when using 
roll-out features. Most configuration had no significant difference. However, 
$\minCmax$ was preferred over expert policy for
\begin{enumerate*}
    \item \jrnd{6}{5} w.r.t. \ref{bias:equal} using \NrFeatSDR\ features
    \item \jrnd{10}{10} for all configurations
\end{enumerate*}
This agrees with the results from \cref{ch:prefmodels} for 16.1 models.

As expected, when more computational budget for $\vphi$ is allocated, the 
quality of the preference model increases, namely (median based on all 
configurations)
\begin{enumerate*}
    \item for \jrnd{6}{5} improved $\Delta\rho\approx-4.3\%$ from 16.1 to 20.1, 
    and $\Delta\rho\approx-5.9\%$ from 20.1 to 24.1 
    \item for \frnd{6}{5} improved $\Delta\rho\approx-3.3\%$ from 16.1 to 20.1, 
    and $\Delta\rho\approx-3.8\%$ from 20.1 to 24.1 
    \item for \jrnd{10}{10} improved $\Delta\rho\approx-5.7\%$ from 16.1 to 
    20.1, and $\Delta\rho\approx-5.6\%$ from 20.1 to 24.1 
\end{enumerate*}

The best configuration, namely following $\minCmax$ with adjusted stepwise bias 
\ref{bias:dbl2nd}, is depicted with the CMA-ES obtained weights in 
\cref{fig:rollout:Cmax:boxplot}. The local 16.1.ES.Cmax model was statistically 
insignificant from the baseline CMA-ES obtained weights. From the figure, we can 
see how the models significantly improve with an increased number of roll-outs.

\begin{figure}[t]\centering 
    \includegraphics[width=\textwidth]{ALL/{boxplot.multi.rollout.6x5}.pdf}
    \includegraphics[width=\textwidth]{{j.rnd}/{boxplot.multi.rollout.10x10}.pdf}
    \caption[Box-plot for multi roll-out featuers based on $\minCmax$]{
        Box-plot for \namerho, using roll-out features for $\minCmax$ 
        trajectory. Directly optimised CMA-ES model shown for reference on 
        the far right.}
    \label{fig:rollout:Cmax:boxplot}
\end{figure}

By using preference models to create a deterministic pilot heuristic it's 
possible to improve the mean \namerho. Especially if we consider using 
\phiSDRRelated\ compared to the best single based roll-out, namely minimum 
$\phiSPT$, then the improvement (all were significant) for adjusted 
\ref{bias:dbl2nd} following $\minCmax$ was 
\begin{enumerate*}
    \item 5.2\% compared to 5.9\% for \jrnd{6}{5}
    \item 6.4\% compared to 7.7\% for \frnd{6}{5} 
    \item 10.1\% compared to 11.8\% for \jrnd{10}{10} 
\end{enumerate*}
When \phiRNDRelated\ were added, then the results for mean 
$\{\rho,\rho_{\text{fort.}}\}$ were as follows
\begin{enumerate*}
    \item $\{1.3\%,0.8\%\}$ compared to $\{1.1\%,0.8\%\}$ using 
    $\phiRNDmin$ for \jrnd{6}{5} 
    \item $\{1.4\%,1.0\%\}$ compared to $\{1.5\%,1.2\%\}$ using 
    $\phiRNDmin$ for \frnd{6}{5} 
    \item $\{6.4\%,4.7\%\}$ compared to $\{9.7\%,6.0\}\%$ using 
    $\phiRNDmean$ for \jrnd{10}{10} 
\end{enumerate*}
Regarding \Problem[6\times5]{\train}, there was no significant difference to 
minimum $\phiRNDmin$. 
Whereas, for \jrnd{10}{10}, the preference model paid off, with an improvement 
of $\Delta\rho\approx-3.3\%$ and $\Delta\rho_{\text{fort.}}\approx-1.3\%$.
This is also the case where the greedy approach of following minimum 
$\phiRNDmin$ was unsuccessful. Note, both dimensions had the same computational 
budget of 100 random roll-outs for each dispatch. 
In the case for \Problem[6\times5]{\train} the 100 roll-outs is quite enough. 
Notice in \cref{fig:rollout:boxplot} that the minimum is very often found (the 
3rd quartile is often 0) and mean $\rho_{\text{fort.}}\leq1.5\%$.
However, as the possibilities for operations grow exponentially with increased 
dimensionality, then for \jrnd{10}{10} we notice that we need to be more 
mindful how we allocate our $(K-k)$ roll-outs to achieve a good performance.

\section{Conclusions}

Revisiting \cref{fig:diff:extr:features}, then $(K-k)$-step lookahead (i.e. 
\phiGlobalRelated) gave consistently the best (single) indicators for finding 
good solutions. 
This makes sense, as they're designed to measure end-performance, which is 
something that the initial 1-step attributes (i.e. \phiLocalRelated) are 
struggling to measure.

By incorporating roll-outs then \cref{eq:CDR:feat} can be considered as a 
deterministic pilot heuristic, which we could learn via preference models from 
\cref{ch:prefmodels}. However, currently they're not feasible for direct 
optimisation as was done in \cref{ch:esmodels} as that would require too many 
costly function evaluations.

Although, for low dimension \jsp\ (i.e. \Problem[6\times5]{\train}) the learned 
deterministic policy did not statistically improve performance, as it was 
equally adequate to pursue minimum $\phiRNDmin$ on its own.
However, going to a higher dimension, as was done for \jrnd{10}{10}, then finally 
we're able to reap the fruit of one's labour.

Unfortunately, \phiRNDRelated\ are not practical features for high dimensional 
data due to excessive computational cost. 
Nevertheless, bearing \cref{fig:diff:case:OPT:10x10} in mind, it 
might be sufficient to lookahead only a few steps at key times in the 
dispatching process. For instance, let the computational budget for 
\frnd{10}{10} roll-outs be full $K$-solutions in the beginning phases as 
that's when the problem space is most susceptible to bad moves. Then gradually 
decrease to only a few step lookahead, as \fsp\ is then relatively stable.
Conversely, for \jrnd{10}{10}, start with a few step lookahead, and then 
expand the horizon as time goes by. Alternatively, when there aren't that many 
dispatches left, it might be worth developing a hybrid approach where the 
remaining dispatches from that point are optimised with some exact methods. 

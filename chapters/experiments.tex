\HeaderQuote{The adventures first... explanations take such a dreadful time.}{The Gryphon} %Alice's Adventures in Wonderland:


\chapter{Experiments }\label{ch:experiments} 
\todoWrite{Compare CMA-ES to PREF models}

\FirstSentence{T}{here's something to be said} for having a good opening line. Morbi commodo, ipsum sed pharetra gravida, orci  $x = 1/\alpha$ magna rhoncus neque, id pulvinar odio lorem non turpis. Nullam sit amet enim. Suspendisse id velit vitae ligula volutpat condimentum. Aliquam erat volutpat. Sed quis velit. Nulla facilisi. Nulla libero. Vivamus pharetra posuere sapien. Nam consectetuer. Sed aliquam, nunc eget euismod ullamcorper, lectus nunc ullamcorper orci, fermentum bibendum enim nibh eget ipsum. Donec porttitor ligula eu dolor. Maecenas vitae nulla consequat libero cursus venenatis. Nam magna enim, accumsan eu, blandit sed, blandit a, eros.
$$\zeta = \frac{1039}{\pi}$$

\clearpage

\Cref{jsp:methods} summarise the main techniques applied to solve \JSP. The 
figure is based on Fig. 1 from \citet{Jain99}, however, updated to reflect the 
previous work relevant to this dissertation.

Some notes
\begin{description}
    \item[Local Search] \cite{LocalSearch} divide approximations methods to 
    \begin{enumerate}
        \item iterative methods, which starts with some initial feasible 
        solution, and its neighbourhood is searched for one with lower cost. If 
        such a solution is found, the algorithm is
        continued from there; otherwise, a local minimum has been found
        \item constructive methods, construct a complete schedule and apply 
        local search to partial schedules on the way
    \end{enumerate}
    \item[Greedy randomized adaptive search procedure] (GRASP) for \JSP\ by 
    \cite{GRASP}
    \item[Beam search] by \cite{BeamSearch} is an adaptation of the branch and 
    bound method in which only some nodes are evaluated in the search tree. At 
    any level, only the promising nodes are kept for further branching and 
    remaining nodes are pruned off permanently.
    \item[Insertion Algorithm] \cite{InsertionAlg} 
\end{description}


\clearpage
%http://tex.stackexchange.com/questions/244742/work-breakdown-structure-wbs-horizontally

\usetikzlibrary{shapes,positioning,shadows,trees}

\tikzset{arrow/.style = {->,>={latex}, draw=black},
        txtLrg/.style  = {draw, text width=2cm, align=center, %drop shadow, 
            font=\bfseries\scriptsize\sffamily, rectangle, thin},
        txtSml/.style  = {txtLrg, align=left, 
            font=\tiny\sffamily},
        txtComment/.style = {font=\tiny\sffamily, xshift=-0.4cm, near start,
            text width=4cm, rotate=90, align=left},
        txtRef/.style= {font=\tiny\sffamily, midway, above, sloped},
        hybrid/.style = {-,draw=black, dashed},
        based/.style = {arrow, dotted},
        veryshort/.style = {text width=3em},
        short/.style = {text width=4.25em},
        medium/.style = {text width=7em},
        long/.style = {text width=7em},
        verylong/.style = {text width=8em},
}

\forestset{
    normal/.style  = {for tree={child anchor=west, parent anchor=east}},
    rotated/.style = {for tree={child anchor=north, parent anchor=south},
        rotate=90},
    unrotate/.style = {normal, rotate=-90},
    root/.style  = {txtLrg, rotated, fill=gray!50, medium},
    onode/.style = {txtLrg, rotated, fill=gray!25, medium},
    tnode/.style = {txtSml, normal, fill=gray!10, long},
    emphasis/.style = {}, %tikz={\node [draw,red,thick,fit to tree]{};}},
    edge from parent/.style={arrow, edge from parent fork right}
}

\begin{figure}[p] \centering
\begin{forest}
for tree={
    s sep=3pt, % distance between siblings
    grow=east,
    growth parent anchor=east,
    edge path={\noexpand\path[\forestoption{edge},->, >={latex}] 
        (!u.parent anchor) -- +(5pt,0pt) |- (.child anchor)
        \forestoption{edge label};}
}
[\JSP, root
    [Approximation, root, edge label = {node[txtComment, left]{
            Approximations methods, or heuristics, are generally time 
            efficient, but do not necessarily attain the global optimum.}},
        [General algorithms (iterative methods), onode, verylong
            [Artificial intelligence, onode
                [Machine learning, onode,  
                    [Roll-out / Pilot method, tnode]
                    [Reinforcement learning, tnode]
                    [Decision tree, tnode]
                    [Imitation learning, onode, emphasis
                        [Active, onode, veryshort
                            [DAgger, tnode, short]
                        ]
                        [Passive, onode, veryshort
                            [Follow the expert, tnode, short]
                            [Perturbed leader, tnode, short]
                            [Follow heuristics, tnode, short]
                        ]
                    ]
                ]
                [Constraint satisfaction,tnode]
%                [Expert systems,tnode]
                [Ant colony optimisation,tnode, name=ACO]
                [Artificial neural network,tnode, name=ANN]
            ]
            [Local search, onode
                [Reinsertion algorithms, tnode]
                [Threshold alg., onode, long, unrotate
                    [Simulated annealing, tnode, name=SA]
                    [Threshold accepting, tnode
%                        [Great Deluge alg. \& record-to-record travel, tnode]
                    ]
                    [Iterative development, tnode]
                ]
%                [Problem space methods, onode
%                    [Problem and heuristic space, tnode]
%                    [GRASP, tnode]
%                ]
                [Large step optimisation, tnode]
                [Tabu Search, tnode, name=TS]
                [Evolutionary computation,tnode
                    [Genetic local search, tnode]
                ]
                [Genetic algorithms, tnode, name=GA
                    [Genetic programming, tnode, name=GP]
                ]
                [Variable depth search, tnode, name=VDS]
            ]
        ]
        [Tailored algorithms (constructive methods), onode, verylong
            [Bottleneck based heuristics, onode, long, unrotate
                [Shifting bottleneck procedure, tnode, short, name=SBP
                    [Beam search, tnode, short, name=BS]
                ]
            ]
            [Insertion algorithm, tnode, veryshort]
            [Priority DR, onode, short, emphasis
                [SDR, onode, short
                    [SPT, tnode, veryshort]
                    [LPT, tnode, veryshort]
                    [LWR, tnode, veryshort]
                    [MWR, tnode, veryshort]
                ]
                [CDR, tnode, name=CDR, veryshort]
            ] 
         ]
    ]
    [Optimisation, root, edge label = {node[txtComment, right]{
            Exact methods guarantee an optimal solution, although for NP-hard 
            problems they are intractable for high dimensionality.}},
        [Efficient methods, tnode, short] 
        [Enumerative methods, onode, 
            [Branch \& Bound, tnode, name=BB]
            [Mathematical, onode
                [Surrogate duality, tnode] 
                [Lagrangian relaxation, tnode] 
                [Decomposition methods, tnode] 
                [Integer linear programming, tnode] 
                [Mixed integer lin.prog., tnode] 
            ]
        ]
    ]
]
\draw[arrow] (ANN) -| (SA);
\draw[arrow] (VDS) -| (SBP);
%\draw[based,out=east,in=north] (BB) to node[txtRef] {\cite{BeamSearch}} (BS);
%\draw[hybrid,out=east,in=east] (GA) to (TS);
%\draw[based,out=east,in=east]  (ACO) to node[txtRef] {\cite{Korytkowski13}} 
%(CDR) ;
%\draw[based,out=east,in=east] (GP) to (CDR);
\end{forest}
\caption[Various methods for solving \JSP]{Various methods for solving \JSP\ 
\citep[based on Fig. 1 from][]{Jain99}}\label{jsp:methods}
\end{figure}
\begin{table}[p] \centering
    \caption{Summary of experimental designs in \namecref{InRu11a}s.}
    \label{papers:summary}
    \subcaptionbox{Model parameters\label{papers:models}}{
    \begin{minipage}{\fullwidth}\centering
    {\setlength{\tabcolsep}{3pt}
    \begin{tabular}{c l c l c}\toprule
        Paper & Problem & Model & Model parameters 
        & $\abs{\text{Model}}$\footnote{Models are either stepwise (i.e. 
            total of $K$ models) or fixed throughout the dispatching 
            process.} \\ \midrule
        \ref{InRu11a} & \JSP & PREF & 
        ~~$\Phi^{\text{OPT}}, \Psi_b$ & $K$ \\
        \ref{InRu11b} & $\R$-functions & CMA-ES & 
        ~~surrogate sampling strategies & 1\\
        \ref{InRu12} & \JSP& SDR & 
        ~~MWR & 1 \\
        \ref{InRu14} & \JSP, \FSP & CMA-ES & 
        ~~$\min \Exp{C_{\max}},\min \Exp{\rho}$ & 1 \\
        \ref{InRu15a} & \JSP & PREF & 
        $\!\begin{array}{l}
            \condset{\Phi^\pi}{\pi\in\{\text{OPT},\text{SDR},\text{ALL}\}}\\
            \condset{\Psi_r}{r\in\{b,f,p\}}
        \end{array}$
        & $K$\\
        \ref{InRu15b} & \JSP, \FSP & PREF & 
        ~~$\Phi^{\text{OPT}},\Psi_p$ & 1\\
        \ref{InRu15c} & \JSP, \FSP & PREF & 
        $\!\begin{array}{l}
            \condset{\Phi^\pi}{\pi\in\{\text{OPT},\text{OPT}\epsilon,\text{DA}i\}}\\
            \Psi_p
        \end{array}$
        & 1 \\
        \bottomrule
    \end{tabular}}
    \end{minipage}}
    \\
    \subcaptionbox{\JSP\ and  \FSP\ problems spaces\label{papers:problems}}{
    \begin{minipage}{\fullwidth}\centering
    {\setlength{\tabcolsep}{3pt}
    \begin{tabular}{c l l l l}\toprule
      Paper & Problem & $I=[u_1,u_2]$\footnote{Processing times are uniformly
        distributed from an interval $I=[u_1,u_2]$, i.e., 
        $\vec{p}\sim\mathcal{U}(u_1,u_2)$.} & size ($n\times m$) & name \\ 
        \midrule
      \ref{InRu11a} & \JSP & $[1,100], [50,100]$ & $6\times6$ & j.rnd, j.rndn\\
      \ref{InRu12} & \JSP & $[1,200]$ & $6\times 6$ & j.rnd \\
      \ref{InRu14} & \JSP, \FSP & $[1,99],[45,55]$ & $6\times5,10\times10$ & 
      j.rnd, j.rndn, f.rnd, f.rndn, f.jc\\
      \ref{InRu15a} & \JSP & $[1,99],[45,55]$ & $6\times5$ & j.rnd, j.rndn\\
      \ref{InRu15b} & \JSP, \FSP & $[1,99],[45,55]$ & $10\times10$ & 
      j.rnd, j.rndn, f.rnd\\
      \ref{InRu15c} & \JSP, \FSP & $[1,99],[45,55]$ & $6\times5,10\times10$ & 
      j.rnd, j.rndn, f.rnd\\
      \bottomrule
    \end{tabular}}
    \end{minipage}}
\end{table}

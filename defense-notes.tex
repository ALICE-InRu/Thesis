\documentclass[20pt,a4paper]{extarticle}

\usepackage{xcolor}
\usepackage{pdfpages}
\usepackage{multirow}

%\renewcommand{\rho}{\varrho}
\renewcommand{\vec}[1]{\mathbf{#1}}
\newcommand{\vphi}{{\mathbf{\phi}}}
\newcommand{\vchi}{{\mathbf{\chi}}}
\newcommand{\vsigma}{{\mathbf{\sigma}}}
\newcommand{\mat}[1]{\mathbf{#1}}
\newcommand{\R}{{\mathbb R}}
\newcommand{\Exp}[1]{{\mathbb E}\left[#1\right]}
\newcommand{\Prob}[1]{{\mathcal P}\left(#1\right)}
\newcommand{\strng}[1]{{\mbox{\tt #1}}}
\newcommand{\inner}[2]{\big<{#1}\cdot{#2}\big>}
\newcommand{\abs}[1]{\lvert#1\rvert}
\newcommand{\norm}[1]{\lVert#1\rVert}
\newcommand{\argmax}{\mathop{\rm argmax}}
\newcommand{\argmin}{\mathop{\rm argmin}}
\newcommand{\nchoosek}[2]{\tiny \left(\begin{array}{c}#1\\#2\end{array}\right)}
\newcommand{\condset}[2]{\left\{#1\;\middle|\;#2\right\}}
\newcommand{\limit}[3]{\lim_{#2}#1=#3}
\newcommand{\bigOh}[1]{{O}\left(#1\right)}

% percentage relative deviation from optimality
\newcommand{\Namerho}{Deviation from optimality, $\rho$}
\newcommand{\namerho}{deviation from optimality, $\rho$}
\newcommand{\fullnamerho}{\namerho, defined by~\cref{eq:rho}}
\newcommand{\Problem}[2][ ]{$\mathcal{P}_{#2}^{#1}$}
\newcommand{\jrnd}[2]{\Problem[#1 \times #2]{j.rnd}}
\newcommand{\jrndJ}[2]{\Problem[#1 \times #2]{j.rnd,J_1}}
\newcommand{\jrndM}[2]{\Problem[#1 \times #2]{j.rnd,M_1}}
\newcommand{\jrndn}[2]{\Problem[#1 \times #2]{j.rndn}}
\newcommand{\frnd}[2]{\Problem[#1 \times #2]{f.rnd}}
\newcommand{\frndn}[2]{\Problem[#1 \times #2]{f.rndn}}
\newcommand{\fjc}[2]{\Problem[#1 \times #2]{f.jc}}
\newcommand{\fmc}[2]{\Problem[#1 \times #2]{f.mc}}
\newcommand{\fmxc}[2]{\Problem[#1 \times #2]{f.mxc}}

\newcommand{\dr}{dispatching rule}
\newcommand{\cdr}{composite priority \dr}
\newcommand{\sdr}{single priority \dr}
\newcommand{\Fsp}{Flow-shop}
\newcommand{\Jsp}{Job-shop}
\newcommand{\fsp}{flow-shop}
\newcommand{\jsp}{job-shop}
\newcommand{\FSP}{FSP}
\newcommand{\JSP}{JSP}
\newcommand{\Jrnd}{\JSP~random}
\newcommand{\JrndJ}{\JSP~random with job variation}
\newcommand{\JrndM}{\JSP~random with machine variation}
\newcommand{\Jrndn}{\JSP~random-narrow}
\newcommand{\Frnd}{\FSP~random}
\newcommand{\Frndn}{\FSP~random narrow}
\newcommand{\Fjc}{\FSP~job-correlated}
\newcommand{\Fmc}{\FSP~machine-correlated}
\newcommand{\Fmxc}{\FSP~mixed-correlated}

% job-related
\newcommand{\phiproc}{$\phi_1$}
\newcommand{\phistartTime}{$\phi_2$}
\newcommand{\phiendTime}{$\phi_3$}
\newcommand{\phiarrivalTime}{$\phi_4$}
\newcommand{\phiwait}{$\phi_5$}
\newcommand{\phijobTotProcTime}{$\phi_6$}
\newcommand{\phijobWrm}{$\phi_7$}
\newcommand{\phijobOps}{$\phi_8$}
\newcommand{\phiJobRelated}{\phiproc\!-\phijobOps}
% mac-related
\newcommand{\phimacFree}{$\phi_{9}$}
\newcommand{\phimacTotProcTime}{$\phi_{10}$}
\newcommand{\phimacWrm}{$\phi_{11}$}
\newcommand{\phimacOps}{$\phi_{12}$}
\newcommand{\phireducedSlack}{$\phi_{13}$} 
\newcommand{\phimacSlack}{$\phi_{14}$}
\newcommand{\phiallSlack}{$\phi_{15}$}
\newcommand{\phiSlackRelated}{\phireducedSlack\!-\phiallSlack}
\newcommand{\phimakespan}{$\phi_{16}$}
\newcommand{\phiMacRelated}{\phimacFree\!-\phimakespan}
% global
\newcommand{\phiSPT}{$\phi_{17}$}
\newcommand{\phiLPT}{$\phi_{18}$}
\newcommand{\phiLWR}{$\phi_{19}$}
\newcommand{\phiMWR}{$\phi_{20}$}
\newcommand{\phiSDRRelated}{\phiSPT\!-\phiMWR}
\newcommand{\phiRND}{\phi_{\text{RND}}}
\newcommand{\phiRNDmean}{$\phi_{21}$}
\newcommand{\phiRNDstd}{$\phi_{22}$}
\newcommand{\phiRNDmin}{$\phi_{23}$}
\newcommand{\phiRNDmax}{$\phi_{24}$}
\newcommand{\phiRNDRelated}{\phiRNDmean\!-\phiRNDmax}
\newcommand{\phiGlobalRelated}{\phiMWR\!-\phiRNDmax}
\newcommand{\phiLocalRelated}{\phiproc\!-\phimakespan}
% feature count
\newcommand{\NrFeatLocal}{16}
\newcommand{\NrFeatGlobal}{4}
\newcommand{\NrFeatTotal}{24}

\hyphenpenalty=800
\hbadness=2500

\hyphenation{heur-ist-ics}
\hyphenation{algo-rithm}

\usepackage{amsmath,amssymb}

\usepackage{tabularx}
\usepackage{multirow}
\usepackage{rotating}
\newcommand{\rot}[1]{\begin{sideways}#1\end{sideways}}

\usepackage[inline]{enumitem}
\setlist[enumerate,1]{
    label=\textit{\roman*)},
    before=\unskip{: }, 
    itemjoin={{; }}, itemjoin*={{, and }},
    after={{. }}
    }

\usepackage{booktabs} % \toprule \midrule \bottomrule

\usepackage{comment}

%\usepackage[retainorgcmds]{IEEEtrantools}

\usepackage{aliascnt}
\usepackage[capitalise,nameinlink]{cleveref} % must come last! 
% provide singular and plural names of the categories "paper"
\crefname{paper}{Paper}{Papers}
\Crefname{paper}{Paper}{Papers}

% Nomenclature
\usepackage{nomencl}
\makenomenclature
\renewcommand\nomgroup[1]{
  \ifthenelse{\equal{#1}{F}}{\item[\textbf{Rice's Framework}]}{
    \ifthenelse{\equal{#1}{J}}{\item[\textbf{\Jsp\ Scheduling}]}{ 
      \ifthenelse{\equal{#1}{O}}{\item[\textbf{Ordinal Regression}]}{ 
        \ifthenelse{\equal{#1}{S}}{\item[\textbf{Surrogate Modelling}]}{ 
          \ifthenelse{\equal{#1}{X}}{\item[\textbf{Experimental Settings}]}{ 
            \ifthenelse{\equal{#1}{Y}}{\item[\textbf{Subscripts and 
                Superscripts}]}{
              \ifthenelse{\equal{#1}{Z}}{\item[\textbf{Acronyms}]}{ 
              }
            }
          }
        }
      }
    }
  }
}
\usepackage{xifthen}

\usepackage{algorithm}
\usepackage{algpseudocode}
\makeatletter
\def\BState{\State\hskip-\ALG@thistlm}
\makeatother
\algnewcommand\algorithmicto{~\textbf{to}~}
\algnewcommand\algorithmicand{~\textbf{and}~}
\algrenewtext{For}[3]%
{\algorithmicfor\ $#1 \gets #2 \algorithmicto\ #3$~\algorithmicdo}
\usepackage{ifthen}
\usepackage{float}
\usepackage{morefloats}

%\usepackage{footnote}
%\usepackage[bottom,perpage,symbol*]{scripts/footmisc}

\newcommand{\tcr}[1]{{\color{red} #1}}
\newcommand{\tcb}[1]{{\color{blue} #1}}
\newcommand{\tcg}[1]{{\color{green} #1}}
\newcommand{\Alice}{Alice}
\newcommand{\fullnameAlice}{Adaptive Learning Intelligent Composite rulEs}

\usepackage[retainorgcmds]{packages/IEEEtrantools}

%\usetikzlibrary{angles} % gametree

\setlist[description]{leftmargin=1cm,labelindent=1cm}
\usepackage[a4paper,top=12pt,bottom=.8cm,right=2cm,left=2cm]{geometry}

\hyphenpenalty=10000

\usepackage{enumitem}
\setlist{leftmargin=0mm}
\newcommand{\bi}{\begin{itemize}\item }
\newcommand{\ei}{\end{itemize}}
\newcommand{\be}{\begin{enumerate}\item }
\newcommand{\ee}{\end{enumerate}}

\newcommand{\forward}{{\color{red}\textbf{forwards}}~}
\newcommand{\backward}{{\color{red}\textbf{backwards}}~}

%opening
\title{\fullnameAlice}
\author{Helga Ingimundard\'{o}ttir}
\date{June 30, 2016}

\newcommand{\printpage}[1]{%
    \clearpage
    \begin{figure}[t!]\centering
    \fbox{\includegraphics[page=#1,scale=0.75]{handout.pdf}}
    \end{figure}
}
\newcommand{\printpages}[2]{
    \clearpage
    \begin{figure}[t!]\centering
    \fbox{\includegraphics[page=#1,scale=0.5]{handout.pdf}}
    \fbox{\includegraphics[page=#2,scale=0.5]{handout.pdf}}
    \end{figure}
}
\newcommand{\printfullpages}[2]{
    \clearpage
    \begin{figure}[t!]\centering
        \fbox{\includegraphics[page=#1,scale=1]{handout.pdf}}
        \fbox{\includegraphics[page=#2,scale=1]{handout.pdf}}
    \end{figure}
}
\begin{document}
\printpage{1}
\bi I would like to start by thanking the Faculty of Industrial Engineering, 
    Mechanical Engineering and Computer Science for accepting my thesis for 
    defence
\ei
\begin{tabular}{cll}
    \textbf{A} & \multicolumn{2}{l}{\textbf{Analysis} or footprints}\\
    \textbf{L} & \multicolumn{2}{l}{Preference \textbf{Learning}}\\
    \textbf{I} & \multicolumn{2}{l}{\textbf{Iterative} feedback improvements}\\
    \textbf{C} & \textbf{Consecutive} & \multirow{2}{*}{$\Bigg\}$ 
    \parbox{8cm}{the 
            search operator\\in algorithm space}}\\
    \textbf{E} & \textbf{Executions}& 
\end{tabular}

\printpage{2}
\bi I will start with a brief introduction
    \item The motivation is to train an optimisation algorithm using data
    \item This can be done for an arbitrary problem domain
    \forward
    \item The main contribution of the thesis is towards a better understanding 
    of how this training should be constructed
    \item The quality of the training data will rewarded with a better learned 
    algorithm
\ei 

\printpage{3}
\bi The thesis presents a framework call \Alice\ to go about this
    \item It is built on Rice's framework for algorithm selection from 1976
    \item Here is it presented as a flow chart, emphasised with pink blocks. 
    The additional blocks are for clarification of what's going on
    \forward \be We start with the problem space -- this could be some 
    arbitrary problem domain
    \forward \item Generally we will work with simulated data, so this will 
    limit 
    the problem space to a subspace of instances
    \forward \item Then we collect features. Features are used to describe the 
    quality of a solution
    \forward \item From there we look at the algorithm space. This could be any 
    kind algorithm you would want to test for your problem space. It could be 
    for example a machine learning technique or evolutionary strategy.
    \forward \item So take an algorithm and apply it to the features resulting 
    for a given set of instances and we get a performance
    \forward \item Now we can infer with the interaction of an algorithm and 
    problem space using footprints in instance space (also known as landmarking)
    This could for explain why an algorithm is successful or unsuccessful for 
    these types of problems
    \ee 
    \item Adding on to Rice's framework \forward the following represents a 
    framework for algorithm learning.
    The additional block are for the learning phase in the flow chart
    \item We would start in the same fashion as before \forward x3
    \item But from the collected features \forward we will create preferences  
    that will serve as input \forward for the preference learning
    \item The outcome of which is \forward an algorithm, and we can continue 
    \forward our framework as before and \forward redo our analysis.
    \item From here we can use the footprints analysis to iteratively improve  
    the learning model via feedback to the preference set
    \item This has been a brief overview, and now I will address each block in 
    more detail. 
\ei 

\printpage{4}
\bi Starting with Problem Space (OVERVIEW) \forward The thesis uses \jsp\ 
scheduling problem as a case study, and I will explain using an hypothetical 
example based around \forward The Mad Hatter's Tea Party
    \item Here we can see four guest: Alice, March Hare, Dourmouse and the host 
    Mad Hatter
    \item All of them have to perform 5 different activities and the time it 
    takes can very between guests 
    \forward \item So this can be considered as a 4-job 5-machine \jsp\ subject 
    to the constraints
    \item \forward none of the guests can multi task and each must follow a 
    predetermined order for their tasks (e.g. March Hare insists on doing them 
    alphabetically) and they would rather wait than breaking habit 
    \item \forward and that a machine can handle at most one job at a time 
    (e.g. 
    two can't simultaneously pour from the same teapot )
    \item We know Alice needs to see the Red Queen ASAP however she's polite 
    and won't leave until everyone has finished their activity
    \item \forward so the objective is to schedule the jobs so as to minimise 
    the 
    maximum completion time (so-called makespan) -- when Alice can leave
\ei

\printpage{5}
\bi We arrived at the tea party with a clean slate.
    \item The disjunctive grasp has a swim-lane for each guest that yields the 
    predetermined ordering, so we go from left to right -- but we can hop 
    between lanes.
    \item We need to visit all of the nodes (guest performing an activity) in 
    the graph with a Hamiltonian path (a total of 20 operations). 
    \item What we need to do know is to select from the available jobs 
    (grey in DG) which one should we select next based on which one of 
    these jobs has the highest priority -- this is essentially a DR
    \item The Gantt chart then translates the dispatched operations into 
    physical time or \textit{seconds} (eg at what o'clock should Alice 
    start/finish pouring tea)
    \item \forward Here we are further along the dispatching seq. and we can 
    see 
    the partial schedule to emerge. 
    \item Solids are already chosen (pink in DG) and dashed are potential 
    outcomes for the available jobs (Notice Alice is finished, so only 3 
    op. currently remain)
    \item \forward We do these executions consecutively until we've visited all 
    nodes in the DG. So this is an example of a completed solution.
    \item Here Mad Hatter is the last to finish, so the makespan is at 548 
    seconds (~9min). 
    \item Notice that Alice finished much earlier and there is a bottleneck for 
    taking turns in speaking ($M_5$), is there some other way to arrange 
    these tasks so Alice could leave sooner?
\ei 

\printpage{6}    
\bi Here the top four are the outcomes for our Tea-Party using commonly used 
    \sdr\ from the literature. They are simple to use yet surprisingly 
    effective. They are called SPT, LPT, LWR and MWR. 
    \item Previous slides showed SPT being constructed with just over 9min, but 
    MWR would have taken $7 \frac{1}{2}$min
    \item A random solution is shown in the lower left-hand-corner (8min)
    \item However an optimal solution (obtained by expert policy or 
    \textit{narrator}) would have been for example the one shown in the lower 
    right hand corner, with Alice leaving in 7min
    \item What we want to now learn is a deterministic policy (or heuristic) 
    that closely resembles the expert policy in the right-hand-corner
    (from these trials, MWR is the best heuristic policy)
\ei 
\printpage{7}
\bi Moving on, given the problem space \jsp\ we will use simulated data 
    so this is the subspace of instances (OVERVIEW) \forward 
    \item The thesis address various problem generators
    \item There are 5 types of processing time distributions (eg seconds it 
    takes Alice to pour a cup of tea)
    \item There are also 2 variations of machine ordering -- if it distinct 
    between all jobs it is called \jsp\ or if it is the same for all jobs it is 
    called \fsp\
    \item Moreover, there are 2 problem sizes explored $6\times5$ that require 
    30 operations and $10\times10$ requiring 100 operations.  
    \item \forward However, only the highlighted subspaces will be explored in 
    this presentation
\ei 
\printpage{8}
\bi Now we can start looking into features for \jsp\\ (OVERVIEW) \forward
    \item Features are a way to measure the quality of our solution which is 
    being built
    \item Here we can see 16 one-step lookahead features and 8 rollout 
    features, but the presentation will only consider one-step lookahead
    \item They are inspired from commonly used \sdr\ 
    \item \forward $\phi_1$ corresponds to shortest/longest proc. time
    \item \forward $\phi_7$ corresponds to least/most work remaining
\ei 
\printpage{9}
\bi We can sample features from $k$-solutions using some policy to guide a 
trajectory through feature space
    \item \forward that could obtained from an expert policy -- so inspecting 
    the 
    features encountered while creating an optimal solution
    \item \forward or we can use some deterministic \sdr\ like SPT \forward or 
    LPT 
    \forward LWR \\ \forward MWR \forward or some random trajectory.
    \item \forward instead of following \sdr\ we could also use some more 
    sophisticated heuristic, such as \cdr\ found by optimising with 
    evolutionary search 
    \item \forward and we can aggregate features across different trajectory 
    policies 
\ei 
\printpage{10}
\bi Remark from our Tea-party example, that Alice finished all of her   
    activities quite early ($k<10$), so even though there are 4-jobs, they can 
    number of available operations we can collect features from diminishes as 
    the dispatching sequence progresses
    \item Here we can see the amount of features encountered evolving with the 
    dispatching step for the aforementioned trajectories 
\ei 
\printpage{11}
\bi Moving on to Algorithm space (OVERVIEW) \forward 
    \item There have been various methods used to solve \jsp\ -- here we can 
    see a tree representation of the nature of those techniques based on the 
    work from Jain and Meeran in 1999
    \item I've highlighted the methods addressed in the thesis:
    \item Branch and bound is the optimisation algorithm I use for labelling my 
    features
    \item \sdr\ have already been described (ie SPT, LPT, LWR, MWR) 
    \item and \cdr\ is a combination of those simpler rules
    \item I used evolutionary search to find appropriate weights for a CDR
    \item Roll-outs were used to design more comprehensive features with 
    respect to the schedule's overall performance (\phiGlobalRelated)
    \item \forward My contribution lies in the \textit{new branch} called 
    imitation learning that to ES learns weights for CDR -- this will be 
    address more thoroughly later
\ei 
\printpage{12}
\bi Performance space (OVERVIEW) \forward
    \item is the loss function that describes the discrepancy between the 
    predicted value using a certain policy compared to the true optimum outcome 
    obtained using expert advice 
    \item therefore the lower the mean for the deviation of optimality the 
    better the policy 
\ei  
\printpage{13}
\bi Regarding footprints in instance space I will start by inferring the     
    interaction of the \sdr s from before to the previously discussed problem 
    generators \\(OVERVIEW) \forward 
    \item Here we can see a box-plot for \namerho\
    \item for \jsp\ it is best to use MWR of these rules as it will give me the 
    lowest mean of deviation
    \item on the other hand its counterpart LWR is much better for \fsp
    \item Here we can see that despite the same processing time distributions 
    the change in machine ordering definition we see MWR is better than LWR for 
    \jsp\ and vice versa for \fsp
    \item \forward Result hold when going to a higher dimensionality 
    \item So this is only based on results from the final dispatching step, but 
    we would like to know when in the dispatching sequence this performance 
    discrepancy starts to occur
\ei 
\printpage{14}
\bi Here we are looking at a $10\times 10$ problem spaces and inspecting given  
    an optimal trajectory how many optimal moves are available at any given 
    dispatching step. As we already saw, the number of total moves (optimal and 
    suboptimal) diminish as $k$ increases 
    \item \forward so here we have the number of moves translated in terms of 
    probability of choosing optimal move -- basically what are the odds you 
    would at random select the optimal move (given you do everything else 
    correct)
\ei 
\printpages{15}{16}
\bi Now let's look at the odds of selecting the optimal move using
    a deterministic policy, such as a \sdr\
    \item Here we see for \jsp\ that SPT is the most likely policy to select 
    the optimal move, or at least until step $k=40$ when MWR becomes the better 
    option
    \item Similar thing are going on for \fsp\ except SPT starts of best until 
    $k=10$ where LWR surpasses in probability 
    \item Based on this information, I want to create a \dr\ that starts with 
    SPT until step 40 and then switches to uses MWR from there on out
    \item \forward I call this a blended \dr 
    \item This switching of policies does improve the upon SPT immensely, 
    however it still doesn't manage to achieve the same greatness MWR did on 
    its own. 
    \item Now why is that? \backward So instead of looking at the probability 
    of selecting the optimal move, let's look at what happens if you make a 
    sub-optimal move \forward
\ei 
\printpage{17}
\bi Here see the impact in terms of \namerho, (so not probability of optimality)
    if we make one misstep -- everything else is done to optimality 
    \item The lower bound is then the best case scenario and upper bound is the 
    worst case scenario -- with the dashed line as the mean impact 
    \item Notice that for \jsp\ it isn't imperative to make a bad decision in 
    the initial phases of dispatching, it's more important to focus on doing 
    the right thing towards the end
    \item The opposite is true for \fsp, where the absolute first moves are 
    critical towards final outcome. 
    \item This can be explained by the nature of \fsp\ where having the same 
    machine ordering constrict the search space in such a way it's hard to 
    recover from previous mistakes. Whereas the randomness in machine ordering 
    \jsp\ natively takes care of that 
\ei 
\printpages{18}{19}
\bi In sequential decision making, all future observations are dependent on 
    previous operations, therefore since we are bound to make mistakes, 
    analysis based on optimal solutions is no longer valid. 
    \item Therefore let's update the measures based on the probability of 
    choosing an optimal move for a given \sdr\ during its dispatching 
    process    
    \item Here we see the compound effects of making suboptimal decisions
    \item Notice that for $k<15$ SPT is above the MWR probability, and this is 
    why switching here is good decision -- although not statically significant 
    from MWR (but as we remember the impact on $\rho$ is the least during that 
    phase)
    \item \forward Analogous to before (shown in yellow) here we see the impact 
    of not following our proposed policy (whose $\rho$ evolution is dashed 
    line). So the band between that line and the lower bound is the possible 
    improvement we could learn from that trajectory 
    \item In fact a BDR switch at $k=40$ never had a chance of 
    improving SPT as the temporal makespan was already higher than the final 
    makespan for MWR and had 60 operation left to dispatch
\ei 
\printpage{20}
\bi With the information obtained by our footprint analysis we can start 
    generating training data that will serve as input in preference learning 
    (OVERVIEW) \forward
    \item The objective will be identifying good dispatching/features from 
    worse ones
    \item \forward From the feature set we have both optimal and suboptimal 
    solutions --this could be the state that results in Alice pouring tea vs. 
    Dormouse singing twinkle-twinkle little star 
    \item \forward By using branch and bound I have determined that Alice is 
    the preferred option of the two, so I can create the preference pair 
    $\{\vphi_{\textrm{A}} - \vphi_{\textrm{D}},+1\}$ and 
    $\{\vphi_{\textrm{D}} - \vphi_{\textrm{A}},-1\}$. 
    The combination of all optimal/suboptimal pairs creates the preference set 
    $\Psi$. And there are various ranking schemes of how to select the minimum 
    mount of pairs without loss of information
    \item Furthermore, since we will eventually like to create a policy that 
    could work for the OR-Library (which can be of a much greater size then I 
    am currently training on) I like to make a dispatching step independent 
    model.
    \item Remember the amount of features encountered for a given 
    trajectory  
\ei 
\printpages{21}{22}
\bi Here we see the resulting amount of preference pairs 
    collected 
    \item Bearing in mind for \jsp\ the impact w.r.t. $\rho$ was the greater 
    influenced during the latter stages of dispatching the emphasis in learning 
    should be there also -- however there are fewer samples to learn from, so 
    we need to balance that out with a stepwise bias strategy 
    \item \forward Here are some ad-hoc strategies for stepwise sampling based 
    on the footprints analysis
\ei 
\printpages{23}{24}
\bi Preference learning is a methodology in order to create a mapping for pairs 
    to ranks, such that optimal feature is preferred to a suboptimal feature if 
    and only if the function evaluation of a optimal feature is higher than the 
    function evaluation of a suboptimal feature
    \item \forward In the thesis I only consider the mapping to be a linear 
    function, so the learning algorithm is optimising with respect to weights 
    $\vec{w}$ using the preference set
    \item the resulting model is effectively a \cdr
    \item \forward Note this is only a linear approximation used to capture the 
    complex underlying dynamics -- you could easily substitute the inner 
    product to kernel methods or a more state-of-the-art algorithm 
    \item \forward Revisiting the algorithm space again, the focus is now on 
    the imitation learning branch which is divided into two sub-branches: 
    Passive and Active
\ei 
\printpage{25}
\bi Passive imitation learning only needs one collection of training set 
    \item \forward I can start by learning on \PsiSet[\textrm{OPT}]{p} -- that 
    is prediction using expert advice
    \item \forward or I can perturb that trajectory with an $\epsilon$ 
    likelihood
    \item \forward or I could training data from some deterministic policy
    \item In all of these cases I'm always labelling the using expert advice 
    (gurobi), is only a matter of where does my feature set come from
    \item \forward Here are box-plots for top two strategies, and we see that 
    by introducing suboptimal states spaces to the training set a lower $\rho$ 
    is achieved -- implying learning for optimal trajectories is insufficient, 
    at least in the case of a linear approximation function it is unattainable 
    to learn those optimal state spaces
    \item Furthermore, using more problem instances (but always same sized 
    $|$\PsiSet[\textrm{OPT}]{p}$|=l_{\max}$, but more variety of problem 
    instances) gives a worse result
\ei 
\printpage{26}
\bi Instead of using some heuristics (here a SDRs chosen ad-hoc) to guide your 
    training data a more sophisticated approach would be to use active 
    imitation leaning, that in an iterative fashion 
    \item \forward I would create a model using Dataset Aggregation (DAgger)
    that essentially continues with the learned policy using expert advice 
    (from prev. slide) -- this would be iteration 0
    \item I would then collect the features encountered by my learned model and 
    retrain -- that way the learned model is more likely to encounter features 
    that it has learned
    \item and there can be a combination of the learned model from the previous 
    iteration and the expert, where $\beta$ controls how you let go of the 
    expert and solely rely on learned policies (fixed-supervised-unsupervised)
    \item \forward In the case of DAgger performance becomes (slightly) better 
    with each iteration of DAgger, however for after the 2nd iteration you need 
    to use new problem instances get more variety of new features (in contrast 
    to PIL).    
\ei 
\printpage{27}
\bi You don't have to have a sampling bias strategy to get a good result 
    using DAgger as it's is automatically inbuilt in DAgger to sample the 
    space. However, if you do use a good sampling bias convergence is quicker
    \item On the left we have two references (not imitation learning). MWR was 
    the best SDR for \jsp\ -- and AIL and PIL approaches were significantly 
    better
    \item CMA-ES is a brute force approach where an optimisation can $~$week in 
    computational effort -- where an iteration of DAgger could take less than a 
    day -- nevertheless the brute force approach still managed to find a better 
    weights than AIL 
\ei 
\printfullpages{28}{29}
\printpages{30}{31}
\bi Invite guests to PhD party!
\ei 

\end{document}

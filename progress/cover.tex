%%%%%%%%%%%%%%%%%%%%%%%%%%%%%%%%%%%%%%%%%
% tungufoss Cover letter
% XeLaTeX Template
% Version 1.0 (30/01/15)
%
% This template was based on Friggeri Resume/CV 
% by Adrien Friggeri https://github.com/afriggeri/CV
% and updated by Helga Ingimundardottir https://github.com/tungufoss/cv
%
% License:
% CC BY-NC-SA 3.0 (http://creativecommons.org/licenses/by-nc-sa/3.0/)
%
% Important notes:
% This template needs to be compiled with XeLaTeX and the bibliography, if used,
% needs to be compiled with biber rather than bibtex.
%
%%%%%%%%%%%%%%%%%%%%%%%%%%%%%%%%%%%%%%%%%

\documentclass{cover} 
%----------------------------------------------------------------------------------------
%	YOUR NAME AND CONTACT INFORMATION
%----------------------------------------------------------------------------------------

\usepackage[utf8]{inputenc}
\usepackage[T1]{fontenc}
\usepackage[icelandic]{babel}
	
\signature{Helga Ingimundardóttir}
\addrfrom{% Address
    Kögunarhæð 1\\ 
    210 Garðabæ
}	
\phonefrom{(+354) 865-1341} % Phone number
\emailfrom{hei2@hi.is} % Email address
	
%----------------------------------------------------------------------------------------
%	ADDRESSEE AND GREETING/CLOSING
%----------------------------------------------------------------------------------------
	
\greetto{} % Greeting text
\closeline{Virðingarfyllst,} % Closing text
	
\nameto{} % Addressee of the letter above the to address
	
\addrto{% To address
    Iðnaðarverkfræði-, vélaverkfræði- og tölvunarfræðideild\\
    Verkfræði- og náttúruvísindasvið\\
    Háskóli Íslands\\
    Reykjavík
}	
%----------------------------------------------------------------------------------------
\pagestyle{plain}
\firstName{Helga}
\lastName{Ingimundardóttir}
\workTitle{doktorsnemi í reikniverkfræði}	

\newcommand{\mycaption}[1]{{\bfseries #1:}\\}
\usepackage{url}
\usepackage{booktabs}
\begin{document}
  
\myletter{	
%----------------------------------------------------------------------------------------
%	LETTER CONTENT
%----------------------------------------------------------------------------------------
Ég heiti Helga Ingimundardóttir og er doktorsefni í reikniverkfræði við 
Iðnaðarverkfræði-, vélaverkfræði- og tölvunarfræðideild.
Ég hóf doktorsnám mitt við deildina haustið 2009 og hefur Tómas Philip 
Rúnarsson, prófessor við iðnaðarverkfræðideild, verið megin leiðbeinandi minn.
Auk hans, þá skipar doktorsnefnd mín einnig: Gunnar Stefánsson, prófessor við 
stærðfræðiskor, og Michele Sébag, PhD sem starfar sem forstöðukona \emph{Équipe 
A-O -- Laboratoire de Recherche en Informatique}, CNRS, í París, Frakklandi. 
Þann 20. september 2014 var formlegur fundur með allri doktorsnefndinni þar sem 
ég lýsti rannsóknarefni mínu. Þessi fundur telst vera ,,munnlega prófið`` um 
miðbik námsins.

Rétt er að taka fram að þar sem þriggja ára framfærslustyrkur frá Vísindasjóði 
Háskóla Íslands var uppurin síðla árs 2012. 
Frá og með janúar 2014 þá hef ég síðan verið í fullu starfi utan Háskóla 
Íslands. 
Ég hef stundað námið því í hjáverkum í næstum þrjú ár, en tekið rispur í 
sumarfríum og þegar tími hefur gefst hjá atvinnurekanda mínum.

\mycaption{Ágrip}
Doktorsverkefnið hefur vinnuheitið ALICE: Analysis \& Learning Iterative 
Consecutive Executions.

\clearpage

\mycaption{Námskeið}
Alls hef ég lokið 34 ECTS einingum í tengslum við námið. 
Þar að auki brautskráðist ég úr kennslufræði 
háskóla, sem er 30 ECTS viðbótardiplóma frá Menntavísindasviði. Kennslufræðin 
var gerð í samráði við deildina, þar sem það var í tengslum við kennslu mína á 
námskeiðinu IÐN401G Aðgerðagreining vormisserin 2011 og 2012. 
Námskeiðin voru eftirfarandi:
\begin{center}
\begin{tabular}{llrcc}
\toprule
Námskeiðsnúmer & Námskeiðsheiti & Einingar & Einkunn & Prófmisseri \\
\midrule
REI101F & Forritun ofurtölva A & 6.0 & 8,0 & des 2009 \\
REI102F	& Forritun ofurtölva B & 6.0 & Staðið & des 2009 \\
STÆ313M	& Grundvöllur tölfræðinnar & 10.0 & 7,0	& jan 2010 \\
FÉL045F	& Rannsóknaráætlanir og umsóknarskrif & 2.0	& Staðið & des 2010 \\
IÐN015F	& Leadership Skills for Doctoral Students & 2.0	& Staðið & des 2010\\
IÐN016F	& Communication Skills for Doctoral Students & 2.0 & Staðið	& maí 2011\\
SIÐ803F	& Siðfræði vísinda og rannsókna	& 6.0 & 7,5	& maí 2012\\
\midrule \multicolumn{2}{r}{alls:} & 34.0 & \\ \midrule
KEN212F	& Inngangur að kennslufræði á háskólastigi & 10.0 & Staðið & des 2010 \\
KEN103F	& Skipulag námskeiða, námsmat og mat á eigin kennslu & 10.0	& Staðið	
& maí 2011\\
KEN004F	& Kennsluþróun og starfendarannsóknir & 10.0 & Staðið & maí 2012 \\
\midrule \multicolumn{2}{r}{alls:} & 30.0 & \\
\bottomrule
\end{tabular}
\end{center}

\mycaption{Greinarskrif og framsögn}
Á námsferlinum hef ég fengið samþykktar fimm ráðstefnugreinar og flutt erindi 
þess efnis á fjórum þeirra. Ég gat ekki sótt ráðstefnuna sjálf fyrir InRu14, 
þannig í því tilfelli var leiðbeinandi minn með erindið, en ég samdi kynninguna 
sem hann flutti sjálf.

Rétt er að taka fram að síðasta ráðstefnugreinin mín, InRu15a, var ein af 
þremur tilnefningum fyrir \emph{Best Paper Award} á Learning and Intelligent 
Optimization ráðstefnunni. 

Nýbúið er að senda inn grein til \emph{Journal of Scheduling}, InRu15b, 
en hún er ekki enn samþykkt. 

Ein af ráðstefnugreininum var birt hjá IEEE, önnur hjá SCITEPRESS Digital 
Library, en restin hjá Springer. 
Í ljósi þess að ég hef ekki neinar samþykktar ISI birtingar þá hefur ritgerðin 
verður skrifuð sem ,,Monograph.``

Eftirfarandi er birtingar mínar í tengslum við doktorsverkefnið:
\begin{description}
    \item[InRu11a] H. Ingimundardottir, T.P. Runarsson. Supervised learning 
    linear priority dispatch rules for job-shop scheduling. In Proceedings of 
    the 5th international conference on Learning and Intelligent Optimization 
    (LION'05), Carlos Coello Coello (Ed.). Springer-Verlag, Berlin, Heidelberg, 
    263-277 (2011). %doi:10.1007/978-3-642-25566-3_20. 

    \item[InRu11b] H. Ingimundardottir, T.P. Runarsson. Sampling Strategies in 
    Ordinal Regression for Surrogate Assisted Evolutionary Optimization. In: 
    11th International Conference on. Intelligent Systems Design and 
    Applications (ISDA), November 22-24, 2011. %doi:10.1109/ISDA.2011.6121815. 

    \item[InRu12] H. Ingimundardottir, T.P. Runarsson. Determining the 
    Characteristic of Difficult Job Shop Scheduling Instances for a Heuristic 
    Solution Method. In: Learning and Intelligent OptimizatioN (LION6), January 
    16-20, 2012. %doi:10.1007/978-3-642-34413-8_36. 
       
    \item[InRu14] H. Ingimundardotir, T.P. Runarsson. Evolutionary Learning of 
    Weighted Linear Composite Dispatching Rules for Scheduling. In: 6th 
    International Conference on Evolutionary Computation Theory and 
    Applications (ECTA6), October 22-24, 2014. %doi:10.5220/0005077200590067. 
    
    \item[InRu15a] H. Ingimundardotir, T.P. Runarsson. Generating Training Data 
    for Learning Linear Composite Dispatching Rules for Scheduling. In: 
    Learning and Intelligent OptimizatioN (LION9), January 12-16, 2015. 
    
    \item[InRu15b] H. Ingimundardotir, T.P. Runarsson. Learning Linear 
    Composite Dispatch Rules for Scheduling. Journal of Scheduling. 
    Submitted 2015
\end{description}

Utan formlegra ráðstefna erlendis, þá hef ég flutt erindi í tengslum verið 
doktorsverkefnið á hinum ýmsum málþingum. 
Þau hafa verið eftirfarandi:

\begin{itemize}
\item Supervised Learning Linear Priority Dispatch Rules for Job-Shop 
Scheduling. Research Symposium, RVoN-2010, The School of Engineering and 
Natural Sciences, University of Iceland, Reykjavik, Iceland, October 9, 2010.

\item Supervised Learning Linear Priority Dispatch Rules for Job-Shop 
Scheduling. Silisian University, Gliwice, Poland, December 15, 2010.

\item Determining the Characteristic of Difficult Job Shop Scheduling Instances 
for a Heuristic Solution Methods. Stats colloquium, Reykjavik, Iceland, 
February 16, 2012. 

\item Creating Meaningful Training Data for Difficult Job Shop Scheduling 
Instances for Ordinal Regression. Seminar for Ph.D. students in School of 
Engineering and Natural Sciences at University of Iceland, Reykjavik, Iceland, 
March 28, 2012.

\item Generating Training Data for Learning Linear Composite Dispatching Rules 
for Scheduling. ReiDok12 Symposium on Computational PhD Projects, School of 
Engineering and Natural Sciences, University of Iceland, Reykjavik, December 3, 
2012. 

\item Supervising Learning Linear Composite Dispatch Rules for Scheduling. 
ReiDok13 Symposium on Computational PhD Projects, School of Engineering and 
Natural Sciences, University of Iceland, Reykjavik, April 22, 2013. 

\end{itemize}

Fundur með Waldemar Grzechca, Assistant Professor í Silesian University
of Technology var haldinn 19 maí 2010. Rannsóknarverkefni Waldemars um Assembly 
Line Balancing var rætt, og hvernig hægt væri að leysa vandamálið
með aðferðafræði doktorsverkefnisins. 
Waldemar lét af hendi MSc ritgerð og hugbúnað sem hægt væri að styðjast við. 
Jafnframt var rætt um áframhaldandi samstarf og heimsókn til Póllands. 
Sú heimsókn átti sér stað 15. desember 2010. 
Í framhaldinu á þeim fundi var unnið því að Assembly Line Balancing samtvinnað 
við kennslu mína á Aðgerðagreiningu vorönnina 2011.
Ekki varð úr frekari samstarfi. 

Jafnframt var hafist handa að samstarfi með Anne Auger hjá INRIA Saclay, 
Frakklandi. 
Á meðan LION6 ráðstefnunni stóð, þá var vinnufundur með Anne á grein sem myndi 
útvíkka ráðstefnunagreinina InRu11b. 
En þar sem fljótlega eftir vinnufundinn var ég komin í fulla vinnu utan 
háskólans þá vannst því miður ekki tími til að ljúka því verkfefni.

\mycaption{Framlag}
Síðan 2009 hef ég unnið ein í doktorsverkefninu mínu undir handleiðslu Tómas 
Philips. 
Ég hef séð um alla forritun og úrvinnslu gagna. Öll gögn og kóði er 
aðgengilegur á netinu af eftirfarandi slóð: \url{github.com/ALICE-InRu/}.

Varðandi greinarskrif þá hef ég séð um það að mestu, en Tómas hefur lesið yfir 
og hjálpað með flæði og orðalag. Einnig hafa hinir ýmsu vinir fengið þann 
heiður að prófarkalesa. Annars ber ég ábyrgð á öllum rituðum texta.
Einsog fram hefur áður komið þá hefur doktorsritgerðin veruð skrifuð sem 
,,monograph`` í huga og er að mestu leiti tilbúin. 
Eftir að er að skrifa formála og prófarkalesa. 

\clearpage
Í samráði við Tómas Philip þá leggum við til að andmælendur séu tveir að 
eftirtöldum sérfræðingum:

\begin{itemize} 
    \item Edmund Burke, Senior Deputy Principal \& Deputy Vice-Chancellor við  
    University of Stirling, Falkirk, Bretland
    \item Sigurður Ólafsson, associate prófessor hjá Industrial and 
    Manufacturing Systems Engineering við Iowa State University, Iowa, 
    Bandaríkin
    \item David Wolfe Corne, prófessor hjá School of Mathematical \& Computer 
    Sciences við Heriot-Watt University, Edinburgh, Bretland
    \item Kate Smith-Miles, prófessor hjá School of Mathematical Sciences við 
    Monash University, Melbourne, Ástralía
    \item Darrell Whitley, prófessor hjá Department of Computer Science við  
    Colorado State University, Colorado, Bandaríkin
    \item Marc Schoenauer, Directeur de Recherche hjá INRIA Saclay -- 
    Île-de-France, Paris, Frakkland
\end{itemize}

Með þessu að leiðarljósi, vil ég biðja deildina um að taka ritgerð mína til 
greinar fyrir vörn síðar á þessu ári. Með þá ósk að geta varið þann 4. desember 
næstkomandi, sem væri síðasti virki dagurinn fyrir þrítugsafmæli mitt. 

}{}{}
\end{document}

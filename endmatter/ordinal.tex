\HeaderQuote{What is the use of repeating all that stuff, if you don't explain it as you go on? It's by far the most confusing thing I ever heard!}{The Mock Turtle} 

\chapter{Ordinal Regression}\label{ch:ordinal} 
\FirstSentence{O}{rdinal regression has been} previously presented in \cite{Ru06:PPSN}, but given here for completeness. The preference learning task of linear classification presented there is based on the work proposed in \citep{liblinear,Lin08:newtontrustregion}. The modification relates to how the point pairs are selected and the fact that a $L2$-regularized logistic regression is used. \todo[inline]{Útskýra hvað er L2-regression?}

\section{Preference set}
The ranking problem is specified by a set $S_{0} = \{(\vec{x}_i,y_i)\}_{i=1}^N \subset X \times Y$ of $N$ (solution, rank)-pairs, where $Y=\{r_1,\ldots,r_N\}$ is the outcome space with ordered ranks $r_1> r_2,> \ldots > r_N$.  
Now consider the model space $\mathcal{H} = \{h(\cdot) : X \mapsto Y\}$ of mappings from solutions to ranks. Each such function $h$ induces an ordering $\succ$ on the solutions  by the following rule,
\begin{equation}\label{eq:linear}
\vec{x}_i \succ \vec{x}_j \quad \Leftrightarrow \quad h(\vec{x}_i) > h(\vec{x}_j)
\end{equation}
where the symbol $\succ$ denotes ``is preferred to.''  

In ordinal regression the task is to obtain function $h$ that can for a given pair $(\vec{x}_i,y_i)$ and $(\vec{x}_j,y_j)$ distinguish between two different outcomes: $y_i > y_j$ and $y_j > y_i$. The task is, therefore, transformed into the problem of predicting the relative ordering of all possible pairs of examples \citep{Herbrich00,Joachims02}.  However, it is sufficient to consider only all possible pairs of adjacent ranks (see also \cite{ShaweTaylor04:book} for yet an alternative formulation).  The preference set, composed of pairs, is then as follows,
\begin{equation}
S = \left\{(\vec{x}_k^{(1)}, \vec{x}_k^{(2)}),t_k=\text{sign}(y_k^{(1)} - y_k^{(2)})\right\}_{k=1}^{N'} \subset X\times Y  \label{eq:S}
\end{equation}
where $(y_k^{(1)} = r_i) \wedge (y_k^{(2)} = r_{i+1})$ (and vice versa $(y_k^{(1)} = r_{i+1}) \wedge (y_k^{(2)} = r_{i})$) resulting in $N'=2(N-1)$ possible adjacently ranked preference pairs. The rank difference is denoted by $t_k\in\{-1,1\}$.

In order to generalize the technique to different solution data types and model spaces an implicit kernel-defined feature space $\Phi\subset\mathbb{R}^d$ of dimension $d$, with corresponding feature mapping $\vphi:X\mapsto\Phi$ is applied, i.e., the feature vector $\vphi(\vec{x})=[\phi_1(\vec{x}),\ldots,\phi_d(\vec{x})]^T\in\Phi$. Thus the preference set defined by \cref{eq:S} is redefined as follows,
\begin{equation}
S = \left\{\left(\vphi(\vec{x}_k^{(1)}), \vphi(\vec{x}_k^{(2)})\right),t_k=\text{sign}(y_k^{(1)} - y_k^{(2)})\right\}_{k=1}^{N'} \subset \Phi \times Y \label{eq:Sfeat}.
\end{equation}


\section{Linear preference}\label{sec:ord:linpref}
The function used to induce the preference is defined by a linear function in the kernel-defined feature space,
\begin{equation} 
 h(\vec{x})=\sum_{i=1}^d w_i\phi_i(\vec{x})=\inner{\vec{w}}{\vphi(\vec{x})} \label{eq:linearpref}
\end{equation}
where $\vec{w}=[w_1,\ldots,w_d]\in\mathbb{R}^d$ has weight $w_i$ for feature $\phi_i$.


\begin{comment}
Let $\vec{z}$ denote either $\vphi(\vec{x}_k^{(1)})-\vphi(\vec{x}_k^{(2)})$ with \mbox{$t_k=+1$} or 
$\vphi(\vec{x}_k^{(2)})-\vphi(\vec{x}_k^{(1)})$ with \mbox{$t_k=-1$}, positive or negative example respectively.

Logistic regression learns the optimal parameters $\vec{w}\in\mathbb{R}^d$ determined by solving the following task,
\begin{equation}\label{eq:margin}
\min_{\vec{w}}\quad \tfrac{1}{2}\inner{\vec{w}}{\vec{w}} + C \sum_{i=1}^{N'} \log\left(1 + e^{-y_i \inner{\vec{w}}{\vec{z}_i}}\right) 
\end{equation}
where $C > 0$ is a penalty parameter, and the negative log-likelihood is due to the fact the given data point $\vec{z}_i$ and weights $\vec{w}$ are assumed to follow the probability model,
\begin{equation}\label{eq:prob}
\mathcal{P}\big(y=\pm1|\vec{z},\vec{w}\big)=\frac{1}{1+e^{-y\inner{\vec{w}}{\vec{z}_i}}}.
\end{equation}
The logistic regression defined in \cref{eq:margin} is solved iteratively, in particular using Trust Region Newton method \citep[cf.][]{Lin08:newtontrustregion}, which generates a sequence $\{\vec{w}^{(k)}\}_{k=1}^\infty$ converging to the optimal solution $\vec{w}^*$ of \cref{eq:margin}.
\end{comment}

The aim now is to find a function $h$ that encounters as few training errors as possible on $S'$. Applying the method of large margin rank boundaries of ordinal regression described in \cite{Herbrich00}, the optimal $\vec{w}^*$ is determined by solving the following task, 
\begin{equation}\label{eq:margin}
  \min_{\vec{w}}\quad \tfrac{1}{2}\inner{\vec{w}}{\vec{w}} + \tfrac{C}{2}\sum_{k=1}^{N'}\xi_k^2
\end{equation}
subject to $t_k\inner{\vec{w}}{(\vphi(\vec{x}_k^{(1)})-\vphi(\vec{x}_k^{(2)})}\ge 1 - \xi_k$ and $\xi_k \ge 0$, $k = 1,\ldots, N'$. The degree of constraint violation is given by the margin slack variable $\xi_k$ and when greater than $1$ the corresponding pair are incorrectly ranked. 
Note that,
\begin{equation}
h(\vec{x}_i)-h(\vec{x}_j)=\inner{\vec{w}}{\vphi(\vec{x}_i)-\vphi(\vec{x}_j)}
\end{equation}
and minimising $\inner{\vec{w}}{\vec{w}}$ in \cref{eq:margin} maximises the margin between rank boundaries, i.e., the distance between adjacently ranked pair $h(\vec{x}^{(1)})$ and $h(\vec{x}^{(2)})$.




\begin{comment}
\section{Non-linear preference}\label{sec:ord:nonlinpref}
In the case that the preference set $S$ defined by \cref{eq:Sfeat} is not linearly separable, a common way of coping with non-linearity is to apply the ``kernel-trick'' to transform $S$ onto a higher dimension. In which case, the dot product in \cref{eq:linear} is replaced by a kernel function $\kappa$.

In terms of training data, the optimal $\vec{w}^*$ can be expressed as, \todo{Finna uppr. heimild}
\begin{equation}
\vec{w}^*=\sum_{k=1}^{N'} \alpha^* t_k \left( \vphi(\vec{x}_k^{(1)})-\vphi(\vec{x}_k^{(2)}) \right)
\end{equation}
and the function $h$ may be reconstructed as follows,
\begin{eqnarray}\label{eq:nonlinear}
h(\vec{x})=\inner{\vec{w}^*}{\vphi(\vec{x})} &=& 
\sum_{k=1}^{N'} \alpha^* t_k \left( \inner{\vphi(\vec{x}_k^{(1)})}{\vphi(\vec{x})}-\inner{\vphi(\vec{x}_k^{(2)})}{\vphi(\vec{x})} ) \right) \nonumber \\ 
&=&\sum_{k=1}^{N'} \alpha^* t_k \left( \kappa(\vec{x}_k^{(1)},\vec{x})-\kappa(\vec{x}_k^{(2)},\vec{x}) \right)
\end{eqnarray}
where $\kappa(\vec{x},\vec{z})=\inner{\vphi(\vec{x})}{\vphi(\vec{z})}$ is the chosen kernel and $\alpha_k^*$ are the Lagrangian multi\-pliers for the constraints that can be determined by solving the dual quadratic programming problem,
\begin{equation}
\max_{\alpha} \sum_{k=1}^{N'} \alpha_k -\frac{1}{2} \sum_{i=1}^{N'}\sum_{j=1}^{N'} \alpha_i\alpha_jt_it_j\left(K(\vec{x}_i^{(1)},\vec{x}_i^{(2)},\vec{x}_j^{(1)},\vec{x}_j^{(2)}) +\frac{1}{C}\delta_{ij}\right)
\end{equation} 
subject to $\sum_{k=1}^{N'} \alpha_kt_k=0$ and $\alpha_k\geq0$ for all $k\in\{1,\ldots,N'\}$, and where,
\begin{eqnarray*}
K(\vec{x}_i^{(1)},\vec{x}_i^{(2)},\vec{x}_j^{(1)},\vec{x}_j^{(2)})  &=& 
\kappa(\vec{x}_i^{(1)},\vec{x}_j^{(1)})-\kappa(\vec{x}_i^{(1)},\vec{x}_j^{(2)})-\kappa(\vec{x}_i^{(2)},\vec{x}_j^{(1)})+\kappa(\vec{x}_i^{(2)},\vec{x}_j^{(2)})
\end{eqnarray*}
and $\delta_{ij}$ is the Kronecker $\delta$ defined to be 1 iff $i=j$ and 0 otherwise.

\subsection{Kernel functions}
There are several choices for a kernel $\kappa$, e.g., \emph{polynomial kernel},
\begin{eqnarray}
\kappa_{\text{poly}}(\vec{x}_i,\vec{x}_j)&=& \left(1+\inner{\vec{x}_i}{\vec{x}_j} \right)^p
\end{eqnarray}
of order $p$, or the most commonly used kernel in the literature which implements a Gauss\-ian radial basis function, the \emph{rbf kernel},
\begin{eqnarray}
\kappa_{\text{rbf}}(\vec{x}_i,\vec{x}_j)&=& e^{-\gamma \norm{\vec{x}_i-\vec{x}_j}^2}
\end{eqnarray}
for $\gamma>0$.
\end{comment}

\section{Parameter setting and tuning}
The regulation parameter $C$ in \cref{eq:margin}, controls the balance between model complexity and training errors, and must be chosen appropriately. A high value for $C$ gives greater emphasis on correctly distinguishing between different ranks, whereas a low $C$ value results in maximising the margin between classes.

\section{Scaling}
It is of paramount importance to scale the features $\vphi$ first, especially if implementing a kernel method. 
In the case of JSP (cf. \cref{ch:prefmodels}), scaling makes the features less sensitive to varying problem instances.
Moreover, for surrogate modelling (cf. \cref{ch:surrogates}), it is important to scale the features $\vphi$ as the evolutionary search zooms in on a particular region of the search space. 

A standard method of doing so is by scaling the preference set such that all points are in some range, typically $[-1,1]$. That is, scaled $\tilde{\vphi}$ is,
\begin{equation}\label{eq:scale}
\tilde{\phi}_i = 2 (\phi_i - \underline{\phi}_i) / (\overline{\phi}_i - \underline{\phi}_i) - 1 
\quad\quad \forall i\in\{1,\ldots,d\}
\end{equation}
where $\underline{\phi}_i$, $\overline{\phi}_i$ are the maximum and minimum $i$-th component of all the feature variables in set $\Phi$, namely,
\begin{equation}
\underline{\phi}_i=\min\{\phi_i\;|\;\forall\vphi\in \Phi\} \quad\textrm{and}\quad \overline{\phi}_i=\max\{\phi_i\;|\;\forall\vphi\in \Phi\}
\end{equation}
where $i\in\{1\ldots d\}$. 



% the abstract

Over the years there have been many approaches to create \dr s for scheduling.
Recent past efforts have focused on direct search methods (e.g. genetic 
programming) or training on data (e.g. supervised learning).
The dissertation will examine the latter and give a framework called 
\fullnameAlice\ (\Alice) on how to do it effectively. 
Defining training data as 
\mbox{$\{\vphi(\vec{x}_i(k)),y_i(k)\}_{k=1}^K\in\mathcal{D}$} the dissertation 
will show
\begin{enumerate*}
    \item samples $\vphi(\vec{x}_i)$ should represent the induced data 
    distribution $\mathcal{D}$. This done by updating the learned model in an 
    active imitation learning fashion
    \item $y_i$ is labelled using an expert policy via a solver
    \item data needs to be balanced, as the set is unbalanced w.r.t. 
    dispatching step $k$
    \item to improve upon localised stepwise features $\vphi$, it's possible to 
    incorporate $(K-k)$ roll-outs where the learned model can be construed as a 
    deterministic pilot heuristic
\end{enumerate*}

When querying an expert policy, there is an abundance of valuable information 
that can be utilised for learning new models.
For instance, it's possible to seek out when the scheduling process is most 
susceptible to failure.
Furthermore, generally stepwise optimality (or classification accuracy) 
implies good end performance, here minimising the final makespan. 
However, as the impact of suboptimal moves is not fully understood, then the 
measure needs to be adjusted for its intended trajectory.

Using these guidelines, it becomes easier to create custom \dr s for one's 
particular application. For this several different distributions of \jsp\ 
will be considered.
Moreover, the machine learning approach is based on preference learning 
which determines what feature states are preferable to others. 
However, that could easily be substituted for other learning methods or applied 
to other shop-constraints or family of scheduling problems that are based on 
iteratively applying \dr s.

%\keywords{Scheduling \and Composite \dr s \and Performance Analysis \and 
% Machine Learning \and Imitation Learning \and DAgger \and Preference Learning}
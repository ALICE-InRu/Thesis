Til eru margar aðferðir við að búa til ákvarðanareglur fyrir áætlanagerð. 
Undanfarið hefur áherslan í fræðunum verið á beina leit (t.d. gentíska bestun) 
eða gagnaþjálfun, en ein aðferð að því síðarnefnda er stýrður lærdómur.
Í ritgerðinni verður sú aðferð skoðuð nánar og sett fram líkan kallað 
„\fullnameAlice“ (\Alice) um hvernig megi 
framkvæma þessa greiningu á skilvirkan máta. 

Látum þjálfunargögnin vera 
\mbox{$\{\vphi(\vec{x}_i(k)),y_i(k)\}_{k=1}^K\in\mathcal{D}$} og ritgerðin mun 
sýna
\begin{enumerate*}[itemjoin*={{ og }}]
    \item úrtök $\vphi(\vec{x}_i)$ þurfa að samræmast við gagnadreifinguna 
    $\mathcal{D}$ sem verður unnin úr henni. Þetta er gert með því að uppfæra 
    lærða líkanið með virkum námsferli byggðu á eftirlíkingum
    \item $y_i$ er merkt með því að nota endurgjöf sérfræðings (gert með bestun)
    \item gögnin þurfa að vera í jafnvægi, þar sem gagnasettið er í ójafnvægi 
    með tilliti til skrefs $k$
    \item til að betrumbæta lýsingu á núverandi stöðu $\vphi$, þá er hægt að 
    nota útspilun fyrir næstu 
    $(K-k)$ skref, þ.e. að endalokum ákvarðanaferilsins. Þá er hægt að túlka 
    lærða líkanið sem fyrirframákveðna útspilunarreglu
\end{enumerate*}

Þegar sérfræðingur er spurður, verður til mikið af gagnlegum upplýsingum sem 
hægt er að nýta til að læra ný líkön. Til að mynda er hægt að komast að því 
hvenær í ákvarðanaferlinu er líklegast að mistök eigi sér stað. Yfirleitt gefa 
háar líkur á því að besta ákvörðun sé tekin (eða þjálfunar nákvæmni) til kynna 
góða lokaframmistöðu, þ.e. í þessu samhengi að lágmarka heildartíma fyrir allt 
ákvarðanaferlið. Þar sem afleiðingar rangra ákvarðana eru ekki alltaf þekktar, 
þá er betra að uppfæra matið með tilliti til ákvarðanatökunar sjálfrar.

Með þessari greiningu er einfaldara að búa til sérhæfðar ákvarðanareglur fyrir 
hverja nýja notkun. Í ritgerðinni verða skoðaðar nokkrar mismunandi tegundir af 
verkniðurröðun á vélar. 
Þar að auki verður vélnámið byggt á ákjósanlegri bestun, þar sem gerður er 
greinarmunur á því hvaða stöður eru betri kostur en aðrar. Ákjósanlegri bestun 
væri þó hægt að skipta út fyrir aðrar námsaðferðir, hægt væri að bæta við 
fleiri skorðum á verkefnið eða beita sömu námsaðferð á aðra tegund af verkefnum 
af svipuðum toga. 


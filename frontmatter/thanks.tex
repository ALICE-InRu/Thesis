% the acknowledgments section

\makeatletter 
This thesis owes its existence to the help, support, and inspiration of many. 
Firstly, I would like to express my sincere appreciation and gratitude to my 
principal advisor, \@advisor, who was abundantly patient and offered invaluable 
assistance and guidance. I hope we continue our brainstorming sessions after my 
defence. Also, thanks for being perfectly fine with me taking off to travel the 
world in between submission deadlines.
Deepest gratitude are also due to my doctoral committee: \@committeeA\ \& 
\@committeeB.
I would also want to give thanks to my opponents, \@opponentA\ \& \@opponentB, 
for taking the time to review my work.
\makeatother

I would like to convey thanks to the Research Fund of the University of 
Iceland for granting me a stipend during the first three years of my 
dissertation. 
A big thank you to my previous co-workers at Valka, especially Einar and Helgi, 
for accommodating my doctoral work during my three years balancing a 
career and pursuing a Ph.D.
I also thank my current co-workers at AGR Dynamics for being so patient with me 
leading up to my defence.

Special thanks to all my post-graduate friends \emph{Helgurnar og 
Tæknigarðsmennirnir} in room \#217
\begin{enumerate*}[label={{\!}}, after={{}}]
    \item Óli Palli for helping with my tax woes and sharing notes during our 
    years still doing regular course work
    \item Anna Helga and Sigrún Helga for encouraging me when my teaching had 
    the better of me
    \item Warsha [hans] Helga for inviting me to the most remarkable three day 
    Indian wedding in Fiji -- an absolutely unforgettable experience, and for 
    always being willing to quit early and head out for Happy-Hour
    \item Erla (an honorary Helga) for being interested to talk about 
    handicrafts and other passions close to my heart in the break room
    \item Gunnar Geir for being my best pupil in Operations Research and for 
    not graduating before me
    \item Bjarki for getting me over to the dark side: 
    the \texttt{ggplot2} package on its own was worth the cross from 
    \textsc{matlab} to \textsc{r}, but \texttt{dplyr} sealed the deal
    \item Chris for believing in me that I could graduate from lower-case 
    \texttt{r} to capital \textsc{r} capabilities. For a non-statistician, my 
    Shiny-app is pretty impressive!
\end{enumerate*}

Equally, thank you to Marta, Morgane, Snjólaug and especially Þorbjörg for 
including me in all engineering Ph.D. activities going on in VR-II. 
May we finally stitch'n'bitch about something other than our Ph.D. projects 
from now on.  
I'm truly blessed for having known the doctoral students in both neighbouring 
buildings, which despite their close proximity are quite divided.

I owe special gratitude to my beloved family for continuous and unconditional 
support of all my undertakings, scholastic or otherwise. 
I thank my uncle Árni for subtly planting the seed that how good it would be 
for the family to have doctor in our midst, and more importantly my aunt Gurrý 
for making the annotation that a doctor could easily be a Ph.D. in my chosen 
field instead of the proposed geriatrics.
I'm grateful for my parents and grandparents always being vocal on how they 
were proud of me pursuing an academic career. I'm blessed that both of my 
grandmothers are here to witness the fruit of my labours after all these years. 

Furthermore, I thank Jóna for helping me translate my abstract and all things 
Icelandic and being my non Ph.D. affiliated friend that understands the appeal 
of this time- and energy-consuming process. I hope you find the time to pursue 
your own doctoral degree in the future.
I appreciate Jake for continually reminding me that I'm a great aspiring 
academic whenever I felt succumbing to imposter syndrome.
Brennan thanks me for helping him to make contributions to science via 
proofreading my conference papers. You are a tiny potato and you believe in me, 
I \emph{can} do the thing.
In addition, I thank Gummi, Kristín \& Sverrir for our after-work special 
\emph{Föstudagur til fjár} where we, the perpetual post-graduate students, 
tried to make time to finish off our studies after punching out after our 9-5pm 
work regime. 
Without Kristín's influence and weekly reminders to keep my head in the game, 
this study would not have been successful (read: taken infinitely longer). 
On that note, I thank the developers at \textsc{github} for creating the simple 
contribution streak feature; a simple yet exceptionally effective carrot to 
keep me motivated.
I also thank Beyoncé for releasing the feminist anthem \emph{***Flawless} ft. 
Chimamanda Ngozi Adichie that played on repeat for most part of writing this 
dissertation. Also, thank you Anna Margrét, for helping me to channel my inner 
Queen B. This diamond, flawless. My diamond, flawless.

This dissertation pays homage to the mathematician Charles Lutwidge Dodgson, 
better known by his pen name Lewis \citeauthor{alice}, and author of the 
literary classics \emph{Alice's Adventures in Wonderland} \citeyearpar{alice} 
and \emph{Through the Looking-Glass, and What Alice Found There} 
\citeyearpar{lookingglass}.

On to my main cheerleaders. 
I would like to thank my brother, Árni Heimir, for reluctantly proofreading my 
manuscript (any grammatical mistakes can be blamed on him, as he has a proper 
degree in English) and teaching me \emph{the way of the showgirl}.
Not forgetting my best friend, Birna, who has always been there for me, and 
probably won't ever read this thesis, for I've burdened her enough with my 
emotional journey to get to this point. 
Finally, I would like to acknowledge my amazing mother, Þóra, for encouraging 
and believing in me throughout my entire education and giving me the support 
and mentality that gave me the opportunity to pursue a doctoral degree. This 
thesis is for you \emph{mamma}.

